\section{Wednesday for MAT4002}\index{Monday_lecture}
\paragraph{Revewing}
Homotopy relative to $\{0,1\}$.

It's \emph{essential} to study homotopy relative to $\{0,1\}$.

Given a torus with a loop $\ell_1(t)$ and a base point $b$,
if without restriction of the homotopy to $\{0,1\}$, and suppose that $X$ is path-connected,
then all loops are homotopic to the constant map $c_b(t)=b$.

In other words, $\ell\simeq c_b,\forall\ell$, i.e., there is only one element $\{[c_b]\}$ in $\pi_1(X,b)$.

Recall that $\pi_1(X,b)$

\begin{proposition}
$[\ell]$ denote the homotopy class of loops relative to $\{0,1\}$ based at $b$, and define
\begin{align*}
[\ell]*[\ell']&=[\ell\cdot\ell']\\
\end{align*}
Then this gives a group structure on
\[
\pi_1(X,b) = \{[\ell]\mid \ell:[0,1]\to X\text{ denotes loops based at $b$}\} 
\]
\end{proposition}
\begin{proof}
\begin{enumerate}
\item
Well-definedness:
Suppose $u\sim u'$ and $v\sim v'$, it suffices to show $u\cdot v\simeq u'\cdot v'$.

Consider $H:u\simeq u'$, $K:v\simeq v'$, where 
$H,K:I\times I\to X$.

Consider $L:I\times I\to X$ with
\[
L(t,s)=\left\{
\begin{aligned}
H(2t,s),&\quad 0\le t\le 1/2\\
K(2t-1,s),&\quad 1/2\le t\le 1
\end{aligned}
\right.
\] 
Therefore, $u\cdot v\simeq u'\cdot v'$.
\item
Associate: $(u\cdot v)\cdot w\simeq u\cdot(v\cdot w)$

They are essentially different loops! They are with different speed.

Consider $H:I\times I\to X$ with
\[
H(t,s) = \left\{
\begin{aligned}
u(4t/(2-s)),&\quad 0\le t\le 1/2-1/4s\\
v(4t - 2+s),&\quad 1/2 - 1/4s\le t\le 3/4 - 1/4s\\
w(4t - 3+s/(1+s)),&\quad 3/4 - 1/4s\le t\le 1
\end{aligned}
\right.
\]
Therefore,
\[
[u]*([v]*[w])=([u]*[v])*[w]
\]
\item
Identity: let $c_b:I\to X$ by $c_b(t) = b,\forall t$, and let $\ell = [c_b]$. WTS:
\[
[c_b]*[\ell]=[\ell]*[c_b]=[\ell]
\]
Or equivalently,
\[
[c_b\cdot\ell]=[\ell\cdot c_b]=[\ell]
\]
Or equivalently,
\[
c_b\cdot\ell\simeq\ell,\quad
\ell\cdot c_b\simeq \ell
\]
Figure
\item
Inverse: for all $u:I\to X$ (loop based at $b$), define $u^{-1}:I\to X$ by $u^{-1}(t)=u(1-t)$.

Note that
\[
[u]*[u^{-1}]=[u\cdot u^{-1}]=[c_b] = e,
\]
i.e., $u\cdot u^{-1}\simeq c_b$:

in fact, let
\[
H(t,s)=\left\{
\begin{aligned}
u(2t(1-s)),&\quad 0\le t\le 1/2\\
u((2-2t)(1-s)),&\quad 1/2\le t\le 1
\end{aligned}
\right.
\]
As a result,
\[
H(t,0)=\left\{
\begin{aligned}
u(2t),&\quad 0\le t\le 1/2\\
u(2-2t),&\quad 1/2\le t\le 1
\end{aligned}
\right.
=
u\cdot u^{-1}
\]
and
\[
H(t,1) = c_b.
\]
Similarly, $u^{-1}\cdot u\simeq c_b$.
\end{enumerate}
\end{proof}

Question:
does the figure below give a homotopy $u\cdot u^{-1}\simeq c_b$?
Answer: no.

\begin{example}
$\pi_1(\mathbb{R}^2,b) = \{e\}$ is trivial.

Indeed, for all $u:I\to\mathbb{R}^2$ with $u(0)=u(1)=b$, consider (pick any $s$, then $H(0,s)=H(1,s)=b$)
\[
H(t,s) = (1-s)u(t)+sb.
\]
Therefore, $u\simeq c_b$ for any lop $u$ based at $b$.

Therefore, $[u] = [c_b] = e$.

More generally, if $X\simeq\{x\}$ is contractible, then $\pi_1(X,b)\simeq\{e\}$ is trivial.

However, $\pi_1(S^1,1)$ is not trivial.
You cannot continuously deform the red loop to a constant loop.

In fact, $\pi_1(S^1,1) \cong\mathbb{Z}$.
\end{example}

\begin{proposition}
If $b,b'$ are path-connected in $X$, then $\pi_1(X,b)\cong\pi_1(X,b')$.
\end{proposition}
\begin{proof}
Let $w$ be a path from $b$ to $b'$, and define
\[
\begin{array}{ll}
w_{\#}:&\pi_1(X,b)\to\pi_1(X,b')\\
\text{with}&[\ell]\mapsto[w^{-1}\ell w]
\end{array}
\]
Check that if $\ell\simeq\ell'$, then $w^{-1}\ell w\simeq w^{-1}\ell' w$.

It is the same proof as in the definition/proposition of the well-definedness of $\pi_1$.

\end{proof}



















