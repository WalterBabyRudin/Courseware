
\section{Wednesday for MAT3040}

\paragraph{Reviewing}
Unitary Operators
\[
\inp{T\bm v}{T\bm w}=\inp{\bm v}{\bm w},\ \forall \bm v,\bm w\in V.
\]
\subsection{Unitary Operator}
\begin{example}
Let $V=\mathbb{R}^n$ with usual inner product.
For the linear operator $T(\bm v) =\bm A\bm v$, $T$ is orthogonal if and only if $\bm A\trans\bm A=\bm I$.

Let $V=\mathbb{C}^n$ with usual inner product.
For the linear operator $T(\bm v) =\bm A\bm v$, $T$ is unitary if and only if $\bm A\Her\bm A=\bm I$.
\end{example}

\begin{proposition}
Let $T:V\to V$ be a linear operator on a vector space over $\mathbb{K}$ satisfying $T'T=I$.
Then for all eigenvalues $\lambda$ of $T$, we have $|\lambda|=1$.
\end{proposition}

\begin{proof}
Suppose we have the eigen-pair $(\lambda,\bm v)$, then
\begin{align*}
\inp{T\bm v}{T\bm v}&=\inp{\bm v}{\bm v}\\
\Longleftrightarrow\inp{\lambda\bm v}{\lambda\bm v}&=\inp{\bm v}{\bm v}\\
\Longleftrightarrow\bar{\lambda}\lambda \inp{\bm v}{\bm v}&=\inp{\bm v}{\bm v}
\end{align*}
Since $\inp{\bm v}{\bm v}\ne0$ $(\bm v\ne\bm0)$, we imply $|\lambda|^2=1$, i.e., $|\lambda|=1$.
\end{proof}


\begin{proposition}
Let $T:V\to V$ be an operator on a finite dimension $V$ over $\mathbb{K}$ satisfying $T'T=I$.
If $U\le V$ is $T$-invariant, then $U$ is also $T^{-1}$-invariant.
\end{proposition}
\begin{proof}
Since $T'T=I$, i.e., $T$ is invertible, we imply $0$ is not a root of $\mathcal{X}_T(x)$, i.e., $0$ is not a root of $m_T(x)$.
Since $m_T(0)\ne0$, $m_T(x)$ has the form
\[
m_T(x) = x^m+\cdots+a_1x+a_0,\ a_0\ne0,
\]
which follows that
\[
m_T(T)=T^m+\cdots+a_0I=0
\implies
T(T^{m-1}+\cdots+a_1I)=-a_0I
\]
Or equivalently,
\[
T\left(
-\frac{1}{a_0}
(T^{m-1}+\cdots+a_1I)
\right)=I
\]
Therefore, 
\[
T^{-1} = -\frac{1}{a_0}T^{m-1}-\cdots-\frac{a_2}{a_0}T-\frac{a_1}{a_0}I,
\]
i.e., the inverse $T^{-1}$ can be expressed as a polynomial involving $T$ only.

Sicne $U$ is $T$-invariant, we imply $U$ is $T^m$-invariant for $m\in\mathbb{N}$, and therefore $U$ is $T^{-1}$-invariant since $T^{-1}$ is a polynomial of $T$.
\end{proof}

\begin{proposition}
Let $T:V\to V$ satisfies $T'T=I$ $(\dim(V)<\infty)$, then
$U\le V$ is $T$-invariant implies $U^\perp$ is $T$-invariant.
\end{proposition}
\begin{proof}
Let $v\in U^\perp$, it suffices to show $T(v)\in U^\perp$.

For all $u\in U$, we have
\begin{align*}
\inp{u}{T(v)}&=\inp{T'(u)}{v}=\inp{T^{-1}(u)}{v}
\end{align*}
Since $U$ is $T^{-1}$-invaraint, we imply $T^{-1}(u)\in U$, and therefore 
\[
\inp{u}{T(v)}=\inp{T^{-1}(u)}{v}=0\implies
T(v)\in U^\perp.
\]
\end{proof}

\begin{theorem}
Let $T:V\to V$ be a unitary operator on finite dimension $V$ (over $\mathbb{C}$), then there exists an orthonormal basis $\mathcal{A}$ such that
\[
(T)_{\mathcal{A},\mathcal{A}} = \diag(\lambda_1,\dots,\lambda_n),\ |\lambda_i|=1,\ \forall i.
\]
\end{theorem}

\begin{proof}[Proof Outline]
Note that $\mathcal{X}_T(x)$ always admits a root in $\mathbb{C}$, so we can always find an eigenvector $\bm v\in V$ of $T$.

Then the theorem follows by the same argument before on seld-adjoint operators.
\begin{itemize}
\item
Consider $U=\Span\{\bm v\}$
\item
$V=U\oplus U^\perp$ and $U^\perp$ is $T$-invariant
\item
Use induction on the unitary operator $T\mid_{U^\perp}:U^\perp\to U^\perp$
\end{itemize}
\end{proof}
\begin{remark}
\begin{itemize}
\item
The argument fails for orthogonal operators
\[
\begin{array}{ll}
T&:\mathbb{R}\to\mathbb{R}^2,\\
\text{with}&T(\bm v)=\bm A\bm v\\
\text{where}&\bm A=\begin{pmatrix}
\cos\theta&-\sin\theta\\
\sin\theta&\cos\theta
\end{pmatrix}
\end{array}
\]
The matrix $\bm A$ is not diagonalizable over $\mathbb{R}$.
It has no real eigenvalues.

However, if we treat $\bm A$ as $T:\mathbb{C}^2\to\mathbb{C}^2$ with $T(\bm v)=\bm A\bm v$, then $\bm A\Her\bm A=\bm I$, and therefore $T$ is unitary.
Then $\bm A$ is diagonalizable over $\mathbb{C}$ with eigenvalues $e^{i\theta},e^{-i\theta}$
\item
As a corollary of the theorem, for all $\bm A\in M_{n\times n}(\mathbb{C})$ satisfying $\bm A\Her\bm A=\bm I$, there exists $P\in M_{n\times n}(\mathbb{C})$ such that 
\[
P^{-1}AP=\diag(\lambda_1,\dots,\lambda_n),\quad
|\lambda_i|=1,
\]
where $P=(\bm u_1,\dots,\bm u_n)$, with $\{\bm u_1,\dots,\bm u_n\}$ forming orthonormal basis of $\mathbb{C}^n$.

In fact, 
\[
P\Her P=\begin{pmatrix}
\bm u_1\Her
\\
\vdots\\
\bm u_n\Her
\end{pmatrix}\begin{pmatrix}
\bm u_1&\cdots&\bm u_n
\end{pmatrix}
=
\begin{pmatrix}
\inp{\bm u_1}{\bm u_1}&\cdots&\inp{\bm u_1}{\bm u_n}\\
\vdots&\ddots&\vdots\\
\inp{\bm u_n}{\bm u_1}&\cdots&\inp{\bm u_n}{\bm u_n}
\end{pmatrix}
\]

Conclusion: all matrices $\bm A\in M_{n\times n}(\mathbb{C})$ with $\bm A\Her\bm A=\bm I$ can be written as
\[
\bm A = \bm P^{-1}\diag(\lambda_1,\dots,\lambda_n)\bm P,
\]
with some $\bm P$ satisfying $\bm P\Her\bm P=\bm I$.
\end{itemize}
\end{remark}
\paragraph{Notation}
Let $U(n)=\{\bm A\in M_{n\times n}(\mathbb{C})\mid \bm A\Her\bm A=\bm I)\}$ be the unitary group, then all $\bm A\in U(n)$ can be diagonalized by 
\[
A=P^{-1}\diag(\lambda_1,\dots,\lambda_n)P,\quad
P\in U(n).
\]

\subsection{Normal Operators}

\begin{definition}[Normal]
Let $T:V\to V$ be a linear operator over a $\mathbb{C}$ inner product vector space $V$.
We say $T$ is \emph{normal}, if
\[
T'T=TT'
\]
\end{definition}

\begin{example}
\begin{itemize}
\item
All self-adjoint operators are normal:
\[
T=T'\implies TT'=T'T=T^2
\]
\item
All (finite-dimensional) unitary operators are normal:
\[
T'T=TT'=I
\]
\end{itemize}
\end{example}


\begin{proposition}
Let $T$ be a normal operator on $V$.
Then
\begin{enumerate}
\item
$\|T(\bm v)\| = \|T'(\bm v)\|,\forall\bm v\in V$.

In particular, $T(\bm v)=0$ if and only if $T'(\bm v)=0$
\item
$(T-\lambda I)$ is also a normal operator, for any $\lambda\in\mathbb{C}$
\item
$T(\bm v) = \lambda\bm v$ if and only if $T'(\bm v)=\bar{\lambda}\bm v$.
\end{enumerate}
\end{proposition}
\begin{proof}
\begin{enumerate}
\item
\begin{align*}
\inp{T\bm v}{T\bm v}&=\inp{T'T\bm v}{\bm v}\\
&=\inp{TT'\bm v}{\bm v}\\
&=\overline{\inp{\bm v}{TT'\bm v}}\\
&=\overline{\inp{T'\bm v}{T'\bm v}}\\
&=\inp{T'\bm v}{T'\bm v}
\end{align*}
Therefore, $\|T(\bm v)\|^2=\|T'(\bm v)\|^2$, i.e.,
$\|T(\bm v)\|=\|T'(\bm v)\|$.
\item
By hw4, $(T-\lambda I)' = T'-\overline{\lambda} I$.
It suffices to check
\[
(T-\lambda I)'(T-\lambda I)=(T-\lambda I)(T-\lambda I)',
\]
Expanding both sides out gives the desired result, i.e.,
\[
(T-\lambda I)'(T-\lambda I)
=(T'-\bar{\lambda}I)(T-\lambda I)
=T'T-\bar{\lambda}T-\lambda T'+\lambda\bar{\lambda}I
\]
and
\[
(T-\lambda I)(T-\lambda I)'
=
(T-\lambda I)(T'-\bar{\lambda}I)
=
TT'-\bar{\lambda}T-\lambda T'+\lambda\bar{\lambda}I
\]
\item
The proof for (3) will be discussed in the next lecture.
\end{enumerate}
\end{proof}




















