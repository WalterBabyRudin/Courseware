\chapter{Week10}
\section{Monday for MAT3040}\index{Monday_lecture}
\subsection{Inner Product Space}
\begin{itemize}
\item
Symmetric: $F(\bm u,\bm w)=F(\bm w,\bm u),\forall\bm u,\bm w$
\item
Non-degenerate: $F(\bm u,\bm w)=0,\forall\bm w$ implies $\bm u=\bm0$
\item
Positive definite: $F(\bm v,\bm v)>0,\forall\bm v\ne\bm0$
\end{itemize}
\paragraph{Classification}
When we say $V$ be a vector space over $\mathbb{F}$, we treat $\alpha\in\mathbb{F}$ as a scalar.
\begin{definition}[Sesqui-linear Form]
Let $V$ be a vector space over $\mathbb{C}$.
A \emph{sesquilinear form} on $V$ is a function $F:V\times V\to\mathbb{C}$ such that
\begin{enumerate}
\item
$F(\bm u+\bm v,\bm w)=F(\bm u,\bm w)+F(\bm v,\bm w)$
\item
$F(\bm u,\bm v+\bm w)=F(\bm u,\bm v)+F(\bm u,\bm w)$
\item
$F(\overline{\lambda}\bm v,\bm w)=F(\bm v,\lambda\bm w)=\lambda F(\bm v,\bm w),\forall\lambda\in\mathbb{C}$
\end{enumerate}
In this case, we say $F$ is \emph{conjugate symmetric} if
\[
F(\bm v,\bm w)=\overline{F(\bm w,\bm v)},\quad\forall \bm v,\bm w\in V.
\]
The definition for non-degenerateness, and positve definiteness is the same as that in bilinear form.
\end{definition}
\begin{remark}
In the sesquilinear form, why there is a $\bar{\lambda}$ shown in condition~(3)?

Partial Answer: We want our $F$ to be positive definite in many cases:
\begin{itemize}
\item
Suppose that $F(\bm v,\bm v)>0$ and we do not have $\bar{\lambda}$ in sesquilinear form $F$, it follows that
\[F(i\bm v,i\bm v) = i^2F(\bm v,\bm v)=-F(\bm v,\bm v)<0\]
As a result, there will be no positive bilinear form for vector space over $\mathbb{C}$.
\end{itemize}
Therefore, $\bar{\lambda}$ is essential to guarantee that we have a positive definite form on vector space over $\mathbb{C}$, i.e.,
\[
F(i\bm v,i\bm v)=\bar{i}iF(\bm v,\bm v)=F(\bm v,\bm v)
\]
\end{remark}

\begin{example}
Consider $V=\mathbb{C}^n$, and a basic sesquilinear form is the Hermitian inner product:
\[
F(\bm v,\bm u)=\bm v\Her\bm u
=
\begin{pmatrix}
\bar{v_1}&\cdots&\bar{v_n}
\end{pmatrix}\begin{pmatrix}
w_1\\\vdots\\ w_n
\end{pmatrix}
=
\sum_{i=1}^n\bar{v_i}w_i
\]
In this case, we do not have symmetric property $F(\bm v,\bm w)=F(\bm w,\bm v)$ any more, instead, we have the conjugate symmetric property $F(\bm v,\bm w)=\overline{F(\bm w,\bm v)}$.
\end{example}

\begin{definition}[Inner Product]
A real~(complex) vector space $V$ with a bilinear~(sesquilinear) form with symmetric~(conjugate symmetric) and positive definite property is called an \emph{inner product} on $V$.
Any vector space equipped with inner product is called an \emph{inner product space}.
\end{definition}
\paragraph{Notation}
We write $\inp{\cdot}{\cdot}$ instead of $F(\cdot,\cdot)$ to denote inner product.
\begin{definition}[Norm]
The \emph{norm} of a vector $\bm v$ is
$
\|\bm v\|=\sqrt{\inp{\bm v}{\bm v}}.
$
\end{definition}
\begin{remark}
As a result, $\|\alpha\bm v\|=\sqrt{\inp{\alpha\bm v}{\alpha\bm v}}=\sqrt{\bar{\alpha}\alpha\inp{\bm v}{\bm v}}
=
\sqrt{|\alpha|^2\inp{\bm v}{\bm v}}=|\alpha|\|\bm v\|$.

The norm is well-defined since $\inp{\bm v}{\bm v}\ge0$ (positive definiteness of inner product).
\end{remark}
\begin{definition}[Orthogonal]
We say a family of vectors $S=\{\bm v_i\mid i\in I\}$ is \emph{orthogonal} if 
\[
\inp{\bm v_i}{\bm v_j}=0,\ \forall i\ne j
\]
If furthermore $\inp{\bm v_i}{\bm v_i}=1,\forall i$, then we say $S$ is an \emph{orthonormal} set.
\end{definition}

\begin{remark}
\begin{enumerate}
\item
The Cauchy-Scharwz inequality holds for inner product space:
\[
|\inp{\bm u}{\bm v}|\le\|\bm u\|\|\bm v\|,\ \forall\bm u,\bm v\in V.
\]
\begin{proof}
The proof for $\inp{\bm u}{\bm v}\in\mathbb{R}$ is the same as in MAT2040 course.
Check Theorem~(6.1) in the note
\[
\text{\url{https://walterbabyrudin.github.io/information/Notes/MAT2040.pdf}}
\]
However, for $\inp{\bm u}{\bm v}\in\mathbb{C}\setminus\mathbb{R}$, we need the re-scaling technique:

Let $\bm w=\frac{1}{\overline{\inp{\bm u}{\bm v}}}\bm u$, then $\inp{\bm w}{\bm v}\in\mathbb{R}$:
\[
\inp{\bm w}{\bm v}=\inp{\frac{1}{\overline{\inp{\bm u}{\bm v}}}\bm u}{\bm v}
=
\overline{\left(\frac{1}{\overline{\inp{\bm u}{\bm v}}}\right)}
\inp{\bm u}{\bm v}
=
\frac{1}{\inp{\bm u}{\bm v}}\inp{\bm u}{\bm v}=
1.
\]
Applying the Cauchy-Scharwz inequality for $\inp{\bm w}{\bm v}\in\mathbb{R}$ gives
\begin{align*}
\left|\inp{\frac{1}{\overline{\inp{\bm u}{\bm v}}}\bm u}{\bm v}\right|&=|\inp{\bm w}{\bm v}|\\
&\le \|\bm w\|\|\bm v\|=\left\|\frac{1}{\overline{\inp{\bm u}{\bm v}}}\bm u\right\|\|\bm v\|
\end{align*}
Or equivalently,
\[
\left|\frac{1}{\inp{\bm u}{\bm v}}\right||\inp{\bm u}{\bm v}|\le\left|
\frac{1}{\overline{\inp{\bm u}{\bm v}}}\right|\|\bm u\|\|\bm v\|
\]
Since $\left|\frac{1}{\inp{\bm u}{\bm v}}\right|=\left|
\frac{1}{\overline{\inp{\bm u}{\bm v}}}\right|$, we imply 
\[
|\inp{\bm u}{\bm v}|\le\|\bm u\|\|\bm v\|
\]
\end{proof}
\item
The triangle inequality also holds for inner product process:
\[
\|\bm u+\bm v\|\le\|\bm u\|+\|\bm v\|
\]
\item
The Gram-Schmidt process holds for finite set of vectors: let $S=\{\bm v_1,\dots,\bm v_n\}$ be (finite) linearly independent. Then we can construct an orthonormal set from $S$:
\[
\bm w_1=\bm v_1,\quad
\bm w_{i+1}=\bm v_{i+1}-\frac{\inp{\bm v_{i+1}}{\bm w_1}}{\|\bm w_1\|^2}-\frac{\inp{\bm v_{i+1}}{\bm w_2}}{\|\bm w_2\|^2}-\cdots-\frac{\inp{\bm v_{i+1}}{\bm w_i}}{\|\bm w_i\|^2},\ i=1,\dots,n-1
\]
Then after normalization, we obtain the constructed orthonormal set.

Consequently, every finite dimensional inner product space has an orthonormal basis.
\end{enumerate}
\end{remark}
\subsection{Dual spaces}

\begin{theorem}[Riesz Representation]
Consider the mapping
\[
\begin{array}{ll}
\phi:&V\to V^*\\
\text{with}&\bm v\mapsto\phi_{\bm v}\\
\text{where}&\phi_{\bm v}(w)=\inp{\bm v}{w},\ \forall w\in V
\end{array}
\]

Then the mapping $\phi$ is well-defined and it is an $\mathbb{R}$-linear transformation.

Moreover, if $V$ is finite dimensional, then $\phi$ is an isomorphism.
\end{theorem}

The $\mathbb{R}$-linear transformation $V\to V^*$ means that, when $V,V^*$ are vector space over $\mathbb{R}$, the $\mathbb{R}$-linear transformation deduces into exactly the linear transformation.

\begin{remark}
The $\mathbb{R}$-linear transformation $V\to V^*$ is \emph{not} necessarily linear if $V,V^*$ are vector spaces over $\mathbb{C}$.

However, we can transform a vector space over $\mathbb{C}$ into a vector space over $\mathbb{R}$:
\begin{itemize}
\item
For example, suppose that $\{\bm v_1,\dots,\bm v_n\}$ is a basis of $V$ over $\mathbb{C}$, i.e.,
\[
\bm v=\sum_{j=1}^n\alpha_j\bm v_j
\]
where $\alpha_j=p_j+iq_j,\forall p_j,q_j\in\mathbb{R}$, then
\[
\bm v=\sum_jp_j\bm v_j+\sum_jq_j(i\bm v_j),\ p_j,q_j\in\mathbb{R}
\]
Therefore, $\{\bm v_1,\dots,\bm v_n,i\bm v_1,\dots,i\bm v_n\}$ forms a basis of $V$ over $\mathbb{R}$.
\end{itemize}
Note that $i\bm v_1$ cannot be considered as a linear combination of $\bm v_1$ over $\mathbb{R}$, but a linear combination of $\bm v_1$ over $\mathbb{C}$.

In particular, if $\phi:V\to V^*$ is a $\mathbb{R}$-linear transformation, then 
\[
\phi(i\bm v)\ne i\phi(\bm v),\text{ but }\phi(2\bm v)=2\phi(\bm v).
\]
\end{remark}
\begin{proof}
\begin{enumerate}
\item
Well-definedness:
We need to show $\phi_{\bm v}\in V^*$, i.e., for scalars $a,b$,
\[
\phi_{\bm v}(a\bm w_1+b\bm w_2)
=
\inp{\bm v}{a\bm w_1+b\bm w_2}
=
a\inp{\bm v}{\bm w_1}+b\inp{\bm v}{\bm w_2}
=
a\phi_{\bm v}(\bm w_1)+b\phi_{\bm v}(\bm w_2)
\]
Therefore, $\phi_{\bm v}\in V^*$.
\item
$\mathbb{R}$-linearity of $\phi$: it suffices to show 
\[
\phi(c\bm v_1+d\bm v_2)=c\phi(\bm v_1)+d\phi(\bm v_2),\quad
\forall c,d\in\mathbb{R},\bm v_1,\bm v_2\in V.
\]
For all $\bm w\in V$, we have
\[
\phi_{c\bm v_1+d\bm v_2}(\bm w)
=
\inp{c\bm v_1+d\bm v_2}{\bm w}
=
c\inp{\bm v_1}{\bm w}
+
d\inp{\bm v_2}{\bm w}
=
c\phi_{\bm v_1}(\bm w)+d\phi_{\bm v_2}(\bm w)
\]
where the second equality holds because $c,d\in\mathbb{R}$.

Therefore,
\[
\phi(c\bm v_1+d\bm v_2)=c\phi(\bm v_1)+d\phi(\bm v_2).
\]
\end{enumerate}
\end{proof}

















