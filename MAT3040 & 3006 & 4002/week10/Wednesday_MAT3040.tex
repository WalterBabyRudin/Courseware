
\section{Wednesday for MAT3040}

\paragraph{Reviewing}
Consider the mapping
\[
\begin{array}{ll}
\phi:&V\to V^*\\
\text{with}&\phi(\bm v) = \phi_{\bm v}\\
\text{where}&\phi_{\bm v}(\bm w) = \inp{\bm v}{\bm w}
\end{array}
\]
The Riesz Representation Theorem claims that 
\begin{enumerate}
\item
$\phi$ is a $\mathbb{R}$-linear transformation.
\item
$\phi$ is injective.
\item
If $\dim(V)<\infty$, then $\phi$ is an isomorphism.
\end{enumerate}
\begin{proof}[Proof for Claim~(2)]
Consider the equality $\phi(\bm v) = \phi_{\bm v}=0_{V^*}$, which implies
\[
\phi_{\bm v}(\bm w)=\inp{\bm v}{\bm w}=0,\forall\bm w\in V
\]
By the non-degenercy property, $\bm v=0_{\bm v}$, i.e., $\phi$ is injective.
\end{proof}
\begin{proof}[Proof for Claim~(3)]
Since $\dim_{\mathbb{R}}(V)=\dim_{\mathbb{R}}(V^*)$, 
and $\phi$ is injective as a $\mathbb{R}$-linear transformation, 
we imply $\phi$ is an isomorphism from $V$ to $V^*$, where $V,V^*$ are treated as vector spaces over $\mathbb{R}$.
\end{proof}

\subsection{Orthogonal Complement}
\begin{definition}[Orthogonal Complement]
Let $U\le V$ be a subspace of an inner product space.
Then the \emph{orthogonal complement} of $U$ is
\[
U^\perp
=
\{\bm v\in V\mid \inp{\bm v}{\bm u}=0,\forall \bm u\in U\}
\]
\end{definition}

The analysis for orthogonal complement for vector spaces over $\mathbb{C}$ is quite similar as what we have studied in MAT2040.
\begin{proposition}
\begin{enumerate}
\item
$U^\perp$ is a subspace of $V$
\item
$U\cap U^\perp = \{0\}$
\item
$U_1\subseteq U_2$ implies $U_2^\perp\le U_1^\perp$.
\end{enumerate}
\end{proposition}

\begin{proof}
\begin{enumerate}
\item
Suppose that $\bm v_1,\bm v_2\in U^\perp$, where $a,b\in K$ ($K=\mathbb{C}$ or $\mathbb{R}$), 
then for all $\bm u\in U$,
\begin{align*}
\inp{a\bm v_1+b\bm v_2}{\bm u}
&=
\bar{a}\inp{\bm v_1}{\bm u}
+
\bar{b}\inp{\bm v_2}{\bm u}\\
&=\bar{a}\cdot0+\bar{b}\cdot0=0
\end{align*}
Therefore, $a\bm v_1+b\bm v_2\in U^\perp$.
\item
Suppose that $\bm u\in U\cap U^\perp$, then we imply $\inp{\bm u}{\bm u}=0$.
By the positive-definiteness of inner product, $\bm u=\bm0$.
\item
The statement~(3) is easy.
\end{enumerate}
\end{proof}

\begin{proposition}\label{pro:10:2}
\begin{enumerate}
\item
If $\dim(V)<\infty$ and $U\le V$, then $V=U\oplus U^\perp$
\item
If $U,W\le V$, then
\begin{align*}
(U+W)^\perp&=U^\perp\cap W^\perp\\
(U\cap W)^\perp&\supseteq U^\perp +W^\perp\\
(U^\perp)^\perp&\supseteq U
\end{align*}
Moreover, if $\dim(V)<\infty$, then these are equalities.
\end{enumerate}
\end{proposition}
\begin{proof}
\begin{enumerate}
\item
Suppose that $\{\bm v_1,\dots,\bm v_k\}$ forms a basis for $U$, and by basis extension, we obtain $\{\bm v_1,\dots,\bm v_k,\bm v_{k+1},\dots,\bm v_n\}$ is a basis for $V$.

By Gram-Schmidt Process, any finite basis induces an orthonormal basis. 

Therefore, suppose that $\{\bm e_1,\dots,\bm e_k\}$ forms an orthonormal basis for $U$, and $\{\bm e_{k+1},\dots,\bm e_{n}\}$ forms an orthonormal basis for $U^\perp$.

It's easy to show $V=U+U^\perp$ using orthonormal basis.
\item
\begin{enumerate}
\item
The reverse part $(U+W)^\perp\supseteq U^\perp\cap W^\perp$ is trivial;
for the forward part, suppose $\bm v\in(U+W)^\perp$, then 
\[
\inp{\bm v}{\bm u+\bm w}=0,\ \forall\bm u\in U,\ \bm w\in W
\]  
Taking $\bm u\equiv\bm0$ in the equality above gives $\inp{\bm v}{\bm w}=0$, i.e., $\bm v\in U^\perp$.
Similarly, $\bm v\in W^\perp$.
\item
Follow the similar argument as in (2a).
If $\dim(V)<\infty$, then write down the orthonormal basis for $U^\perp+W^\perp$ and $(U\cap W)^\perp$.
\item
Follow the similar argument as in (2a).
If $\dim(V)<\infty$, then 
\[
V=U^\perp\oplus (U^\perp)^\perp=U\oplus U^\perp.
\]
Therefore, $(U^\perp)^\perp=U$.
\end{enumerate}
\end{enumerate}
\end{proof}

\begin{proposition}
The mapping $\phi:V\to V^*$ maps $U^\perp\le V$ injectively to $\text{Ann}(U)\le V^*$.
If $\dim(V)<\infty$, then $U^\perp\cong\text{Ann}(U)$ as $\mathbb{R}$-vector spaces
\end{proposition}
\begin{proof}
The injectivity of $\phi$ has been shown at the beginning of this lecture.
For any $\bm v\in U^\perp$, we imply $\phi_{\bm v}(\bm u)=0,\forall\bm u\in U$, i.e., $\phi_{\bm v}\in\text{Ann}(U)$.

Therefore, $\phi(U^\perp)\le\text{Ann}(U)$. 

Provided that $\dim(V)<\infty$, by (1) in proposition~(\ref{pro:10:2}), 
\[
\dim(U)+\dim(U^\perp)=\dim(V)
\]
Since $\dim(U)+\dim(\text{Ann}(U))=\dim(V)$, we imply $\dim(U^\perp) = \dim(\text{Ann}(U))$.

Moreover, 
\[
\phi:U^\perp\to\text{Ann}(U)
\]
is an isomorphism between $\mathbb{R}$-vector spaces $U^\perp$ and $\text{Ann}(U)$.
\end{proof}

\subsection{Adjoint Map}
\paragraph{Motivation}
Then we study the induced mapping based on a given linear operator $T$, denoted as $T'$.
This induced mapping essentially plays the similar role as taking the Hermitian for a complex matrix.
\paragraph{Notation}
Previously we have studied the \emph{adjoint} of $T:V\to W$, denoted as $T^*:W^*\to V^*$.
However, from now on, we use the same terminalogy but with different meaning.
If $T:V\to V$ is a linear operator, then the \emph{adjoint} of $T$ is the linear operator $T':V\to V$ defined as follows.

\begin{definition}[Adjoint]
Let $T:V\to V$ be a linear operator between inner product spaces.
The \emph{adjoint} of $T$ is defined as 
$T':V\to V$ satisfying
\begin{equation}\label{Eq:10:1}
\inp{T'(\bm v)}{\bm w}=\inp{\bm v}{T(\bm w)},\ \forall \bm w\in V
\end{equation}
\end{definition}

\begin{proposition}
If $\dim(V)<\infty$, then $T'$ exists, and it is unique.
Moreove, $T'$ is a linear map.
\end{proposition}

\begin{proof}
Fix any $\bm v\in V$. Consider the mapping
\[
\alpha_{\bm v}: \bm w\xrightarrow{T}T(\bm w)\xrightarrow{\phi_{\bm v}}\inp{\bm v}{T(\bm w)}
\]
This is a linear transformation from $V$ to $\mathbb{F}$, i.e., $\alpha_{\bm v}\in V^*$

By Riesz representation theorem, $\phi$ is an isomorphism from $V$ to $V^*$. 
Therefore, for any $\alpha_{\bm v}\in V^*$, 
there exists a vector $T'(\bm v)\in V$ such that
\[
\phi(T'(\bm v))=\alpha_{\bm v}\in V^*
\]

Or equivalently, $\phi_{T'(\bm v)}(\bm w)=\alpha_{\bm v}(\bm w),\forall \bm w\in V$, i.e., $\inp{T'(\bm v)}{\bm w}=\inp{\bm v}{T(\bm w)}$.

Therefore, from $\bm v$ we have constructed $T'(\bm v)$ satisfying (\ref{Eq:10:1}).
Now define $T':V\to V$ by $\bm v\mapsto T'(\bm v)$.

\begin{itemize}
\item
Since the choice of $T'(\bm v)$ is unique by the injectivity of $\phi$, $T'$ is well-defined.
\item
Now we show $T'$ is a linear transformation:
Let $\bm v_1,\bm v_2\in V,a,b\in K$. For all $\bm w\in V$, we have
\begin{align*}
\inp{T'(a\bm v_1+b\bm v_2)}{\bm w}&=\inp{a\bm v_1+b\bm v_2}{T(\bm w)}\\
&=\bar{a}\inp{\bm v_1}{T(\bm w)}+\bar{b}\inp{\bm v_2}{T(\bm w)}\\
&=\bar{a}\inp{T'(\bm v_1)}{\bm w}+\bar{b}\inp{T'(\bm v_2)}{\bm w}\\
&=\inp{aT'(\bm v_1)+bT'(\bm v_2)}{\bm w}
\end{align*}
Therfore,
\[
\inp{T'(a\bm v_1+b\bm v_2) - [aT'(\bm v_1)+bT'(\bm v_2)]}{\bm w}=0,\ \forall \bm w\in V
\]
By the non-degeneracy of inner product,
\[
T'(a\bm v_1+b\bm v_2) - [aT'(\bm v_1)+bT'(\bm v_2)]=\bm0,
\]
i.e., $T'(a\bm v_1+b\bm v_2)=aT'(\bm v_1)+bT'(\bm v_2)$
\end{itemize}
\end{proof}


\begin{example}
Let $V=\mathbb{R}^n$, $\inp{\cdot}{\cdot}$ as the usual inner product. Consider the matrix-multiplication mapping
\[
\begin{array}{ll}
T:&V\to V\\
&T(\bm v)=A\bm v
\end{array}
\]
Then $\inp{T'(\bm v)}{\bm w}=\inp{\bm v}{T(\bm w)}$ implies
\begin{align*}
(T'(\bm v))\trans\bm w&=\inp{\bm v}{\bm A\bm w}\\
&=\bm v\trans\bm A\bm w\\
&=(\bm A\trans\bm v)\trans\bm w
\end{align*}
Therfore, $T'(\bm v) = A\trans\bm v$.
\end{example}

\begin{proposition}\label{Pro:10:5}
Let $T:V\to V$ be a linear transformation, $V$ a inner product space.
Suppose that $\mathcal{B}=\{\bm e_1,\dots,\bm e_n\}$ is an orthonormal basis of $V$, then
\[
(T')_{\mathcal{B},\mathcal{B}}=\overline{((T)_{\mathcal{B},\mathcal{B}})\trans}
\]
\end{proposition}

\begin{proof}
Suppose that $(T)_{\mathcal{B},\mathcal{B}}=(a_{ij})$, where $T(\bm e_j)=\sum_{k=1}^na_{kj}\bm e_k$, then
\begin{align*}
\inp{\bm e_i}{T(\bm e_j)}&=\inp{\bm e_i}{\sum_{k=1}^na_{kj}\bm e_k}\\
&=\sum_{k=1}^na_{kj}\inp{\bm e_i}{\bm e_k}\\
&=a_{ij}
\end{align*}
Also, suppose $(T')_{\mathcal{B},\mathcal{B}}=(b_{ij})$, we imply $T'(\bm e_j) = \sum_{k=1}^nb_{ij}\bm e_k$, which follows that
\[
\inp{\bm e_i}{T'(\bm e_j)}=b_{ij}
\implies
\overline{\inp{T'(\bm e_j)}{\bm e_i}}=b_{ij}
\implies
\overline{\inp{\bm e_j}{T(\bm e_i)}}=b_{ij},
\]
i.e., $\overline{a_{ji}} = b_{ij}$.
\end{proof}
\begin{remark}
Proposition~(\ref{Pro:10:5}) does not hold if $\mathcal{B}$ is not an orthonormal basis.
\end{remark}












