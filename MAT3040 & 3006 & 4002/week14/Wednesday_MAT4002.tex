\section{Wednesday for MAT4002}\index{Monday_lecture}
\begin{theorem}
\[
\pi_1(S^1)\cong(\mathbb{Z},+)
\]
\end{theorem}
\begin{proof}
It's clear that $\pi_1(S^1)\cong E(K,1)$, where $K$ is the triangle illustrated in the figure below.

For every edge loop based at $1$,
\[
\alpha\sim(123\cdots1231),\quad
\text{or}
\alpha\sim(132\cdots1321)
\]
Question: can $\alpha$ be equivalent to both $(123\cdots1231)$ and $(132\cdots1321)$?

Define the winding number of $\alpha$ by
\[
\begin{array}{l}
\text{number of 23 appearing in $\alpha$}-\\
\text{number of 32 appearing in $\alpha$}
\end{array}
\]
It follows that
\begin{enumerate}
\item
Then if $\alpha\sim\alpha'$, then $\alpha$ and $\alpha'$ have the same winding number, e.g.,
\[
\alpha = (1231231),\ \alpha'=(123231231)
\]
\item
In particular, w.n. of $(123123\cdots1231)$ ($23$ shows for $m$ times) is $m$;
w.n. of $(132132\cdots1321)$ ($32$ shows for $n$ times) is $-n$
\end{enumerate}
In conclusion, $\alpha$ is equivalent to a unique $(123123\cdots1231)$ or $(132\cdots1321)$, since otherwise will have different winding numbers.

Define
\[
\begin{array}{ll}\phi:
&E(K,1)\to (\mathbb{Z},+)\\\text{with}&[\alpha]\mapsto
\text{winding number of $\alpha$}
\end{array}
\]
It follows that for $[\alpha]=[(1bc1bc\cdots1bc1)]$ and $[\beta]=[(1pq1pq\cdots1pq1)]$,
\[
\phi([\alpha]\cdot[\beta])=\phi([\alpha\cdot\beta])
=
[(1bc1bc\cdots1bc11pq1pq\cdots1pq1)]
\]
If $\phi([\alpha])=j,\phi([\beta])=k$, and $jk>0$, then ...
if $jk<0$, then $12311321$ reduces into $1$, then ...
\[
\phi([\alpha]\cdot[\beta])=j+k=\phi([\alpha])+\phi([\beta])
\]
$\phi$ is a homomorphism.

$\phi$ is bijective: surjective is clear; suppose $\phi([\alpha])=0$, then by definition of $\phi$, $[\alpha]=[(1)]=e$, which is the trivial element in $E(K,1)$.

Therefore, $\phi$ is an isomorphism.
\end{proof}

\begin{remark}
The loop based at $1$ given by:
\[
\begin{array}{ll}
\ell&I\to S^1\\\text{with}&t\mapsto e^{2\pi i t}
\end{array}
\]
satisfies $\phi([\ell])=1$ in $\phi:\pi_1(S^1,1)\cong\mathbb{Z}$.

Therefore,
\[
\phi(\underbrace{[\ell]\cdots[\ell]}_{\text{$m$ times}})=m
\]
More precisely, the loop
\[
\begin{array}{ll}
\ell^m:&I\to S^1\quad m\in\mathbb{Z}\\
\text{with}&\ell^m(t) = e^{2\pi imt}
\end{array}
\]
gives $\phi([\ell^m])=m$
\end{remark}

\begin{corollary}[Fundamental Theorem of Algebra]
All non-constant polynomials in $\mathbb{C}$ has a root in $\mathbb{C}$
\end{corollary}

\begin{proof}
Suppose on the contrary that 
\[
p(x)=a_nx^n+\cdots+a_1x+a_0\ a_n\ne0
\]
has no roots, then 
\[
p:\mathbb{C}\to\mathbb{C}\setminus\{0\}
\]
Since $\mathbb{C}\setminus\{0\}\simeq\mathcal{O}=\{z\in\mathbb{C}\mid|z|=1\}$, we imply
\[
\pi_1(\mathbb{C}\setminus\{0\})=\pi_1(S^1)\cong\mathbb{Z}.
\]
Consider the map (since $\mathbb{C}\simeq\{0\}$ is contractible)
\[
\begin{array}{ll}
p_*:&\pi_1(\mathbb{C})\to \pi_1(\mathbb{C}\setminus\{0\})\\
&\{e\}\mapsto\mathbb{Z}
\end{array}
\]
Note that $p_*(e)=0$.

Consider $C_r=\{z\in\mathbb{C}\mid |z|=r\}$ and the map
\[
\begin{array}{ll}
i:&C_r\to\mathbb{C}\\
\text{with}&z\mapsto z
\end{array}
\]
such that
\[
\begin{array}{ll}
i_*:&\pi_1(C_r)\to\pi_1(\mathbb{C})\\
\text{with}&(p\mid_{c_r})_* = p_*\circ i_*,\\
&\text{$p_*$ maps $m$ to $0$}
\end{array}
\]
Study $p\mid_{c_r}:c_r\to\mathbb{C}\setminus\{0\}$.
\[
\left\{
\begin{aligned}
q(z)&=k\cdot z^n,\quad k:=\frac{p(r)}{r^n}\text{is a constant}\\
p(z)&=a_nz^n+\cdots+a_1z+a_0
\end{aligned}
\right.
\]
Note that $p(r)=q(r)$, and $p,q:C_r\to\mathbb{C}\setminus\{0\}$.
\begin{itemize}
\item
We claim that $p\simeq q$ for large $r$:
let $H(z,t)=tp(z)+(1-t)q(z)$, then 
\[
\begin{array}{ll}
H:&c_r\times[0,1]\to\mathbb{C}\\
\text{with}&H(z,0)=q(z), \ H(z,1) = p(z)
\end{array}
\]
Indeed, $H:c_r\times[0,1]\to\mathbb{C}\setminus\{0\}$:
SUppose on the contrary that there exists $(z,t)$ such that
\[
(1-t)p(z)+tq(z)=0,\quad |z|=r, t\in[0,1]
\]
then 
\[
(1-t)(a_nz^n+\cdots+a_1z+a_0)+t\cdot kz^n=0
\]
Substituting $k=p(r)/r^n$ into it,
\[
a_nz^n+\cdots+a_1z+a_0
=
t(a_{n-1}z^{n-1}+\cdots+a_0 - (a_{n-1}\frac{z^n}{r}+\cdots+a_1\frac{z^n}{r^{n-1}}+a_0\frac{z^n}{r^n}))
\]
Therefore, LHS has leading order $n$, the RHS has leading order less or equal to $n-1$.
As $r=|z|\to\infty$, then $t\to\infty$.
In order to make the equality hold, if we pick $r=|z|$ large enough, the equality holds only for some $t>1$.

For this choice of $r=|z|$, 
\[
H:c_r\times[0,1]\to\mathbb{C}\setminus\{0\}
\]
gives the homotopy $p(z)\simeq q(z)$.
\item
Therefore, $q:c_r\to\mathbb{C}\setminus\{0\}$ with $q(z)=kz^n$ satisfies 
\[
(p\mid_{c_r})_*=q_*:\pi_1(c_r)\to\pi_1(\mathbb{C}\setminus\{0\})
\]
such that 
figrues.
\end{itemize}
\end{proof}













