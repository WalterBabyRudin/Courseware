\chapter{Week15}
\section{Monday for MAT3040}\index{Monday_lecture}
\subsection{More on Exterior Power}
\paragraph{Reviewing}
Let $\text{Obj}:=\{\phi:V\times\cdots\times V\to W\mid\text{$\phi$ is alternating}\}$,
then there exists
\[
\{\Lambda:V\times\cdots\times V\to E\}\in\text{Obj}
\]
such that
\[
\phi = \bar{\phi}\circ\Lambda,\quad\text{where $\bar{\phi}:E\to W$ is the unique linear transformation}
\]
Here we give one way for constructing $E$:
\[
E = V^{\otimes p}/U,
\]
where $U$ is spanned by vectors of the form 
\[
v_1\otimes\cdots\otimes v_p\in V^{\otimes p},
\quad
v_i=v_j\text{ where for some $i\ne j$.}
\]
For instance, $v\otimes v\otimes\cdots\otimes v_p\in U$.

\begin{definition}[Wedge Product]
Define the wedge product space
\[
\wedge^pV:=V^{\otimes p}/U =E,
\]
with the wedge product among vectors
\[
v_1\wedge\cdots\wedge v_p = v_1\otimes\cdots\otimes v_p+U\in \wedge^pV
\]
\end{definition}
As a result, the mapping
\[
\begin{array}{ll}
\wedge:&V\times\cdots\times V\to E:=\wedge^pV\\
&(v_1,\dots,v_p)\mapsto v_1\wedge\cdots\wedge v_p
\end{array}	
\]
will satisfy the universal property of exterior power.

\begin{proposition}\label{pro:15:1}
\begin{enumerate}
\item
We have the $p$-linearity for $\wedge^pV$, i.e.,
\[
v_1\wedge\cdots\wedge(av_i+bv_i')\wedge\cdots\wedge v_p
=
a
(v_1\wedge\cdots\wedge v_i\wedge\cdots\wedge v_p)
+
b
(v_1\wedge\cdots\wedge v_i'\wedge\cdots\wedge v_p)
\]
for $i=1,\dots,p$.
\item
The wedge product is alternating:
\begin{align*}
v_1\wedge\cdots\wedge v\wedge\cdots \wedge v\wedge\cdots\wedge v_p&:=v_1\otimes\cdots\otimes v\otimes\cdots \otimes v\otimes\cdots\otimes v_p+U\\
&=0+U\\
&=0_{\wedge^pV}
\end{align*}
\item
The wedge product reverses sign reversal property:
\[
v_1\wedge\cdots\wedge v\wedge\cdots\wedge w\wedge\cdots\wedge v_p
=
-
v_1\wedge\cdots\wedge w\wedge\cdots\wedge v\wedge\cdots\wedge v_p
\]
Reason: $(v+w)\wedge(v+w)=0$, which implies $v\wedge w+w\wedge v=0$.
\end{enumerate}
\end{proposition}

\begin{proposition}
\begin{enumerate}
\item
If $\dim(V)=n$, and $0\le p\le n$, then
\[
\dim(\wedge^pV)=\binom{n}{p}
\]
\item
For all linear operators $T:V\to V$, there is an unique linear operator
from $\wedge^pV$ to $\wedge^pV$:
\[
\begin{array}{ll}
T^{\wedge^p}:&\wedge^pV\to \wedge^pV\\
\text{with}&v_1\wedge\cdots\wedge v_p\mapsto
T(v_1)\wedge\cdots\wedge T(v_p)
\end{array}
\]
\end{enumerate}
\end{proposition}

\begin{proof}
\begin{enumerate}
\item
Let $\{v_1,\dots,v_n\}$ be basis of $V$, 
then $\{v_{i_1}\otimes\cdots\otimes v_{i_p}\mid 1\le i_{k}\le n\}$ forms basis of $V^{\otimes p}$.
Note that $\{v_{i_1}\wedge\cdots\wedge v_{i_p}\mid 1\le i_{k}\le n\}$ spans $\wedge^pV$, since $\pi_V:V\to V/U$ is surjective.
We claim that
\[
\mathcal{B}=\{v_{i_1}\wedge\cdots\wedge v_{i_p}\mid 1\le i_1<i_2<\cdots<i_p\le n\}
\]
is a basis of $\wedge^pV$
\begin{itemize}
\item
$\mathcal{B}$ spans $\wedge^pV$:
we can use (3) in proposition~(\ref{pro:15:1}) to ``rearrange'' the indices $j_1,\dots,j_p$ into ascending order, and $\Span(\mathcal{B})=\Span\{v_{i_1}\wedge\cdots\wedge v_{i_p}\mid 1\le i_{k}\le n\}$.
\item
We omit the proof that $\mathcal{B}$ is linear independent due to time limit.
\end{itemize}
The numbre of vectors in $\mathcal{B}$ is equal to $\binom{n}{p}$.

\end{enumerate}
\end{proof}

\subsection{Determinant}
\paragraph{Previous Approach for defining determinant}
We define the determinant for $\bm A = M_{n\times n}(\mathbb{F})$ directly.
From such complicated definition, we come up with $\det(\bm A\bm B)=\det(\bm A)\det(\bm B)$, which implies that the similar matrices share with the same determinant, then we define the determinant for any linear operator $T:V\to V$ as
\[
\det(T) = \det((T)_{\mathcal{B},\mathcal{B}}),\quad
\text{for some basis $\mathcal{B}$ of $T$}
\]

\paragraph{New Approach}
We will define $\det(T)$ for linear operators without fixing a basis,
and then we will imply $\det(T\circ S) = \det(T)\det(S)$ easily.
Then $\det(\bm A)$ for $\bm A\in M_{n\times n}(\mathbb{F})$ belongs to our special case.

\begin{definition}[Determinant for Linear Operators]
\begin{enumerate}
\item
Suppose that $\dim(V)=n$, then
\[
\dim(\wedge^nV)=\binom{n}{n}=1
\]
More precisely, for any basis $\{v_1,\dots,v_n\}$ of $V$, we have $\wedge^n(V)=\Span\{v_1\wedge\dots\wedge v_n\}$.
\item
Note tht $T^{\wedge^n}:\wedge^nV\to \wedge^nV$
is a linear operator on $\wedge^nV\cong\mathbb{F}$.
Therefore, for all $\tau\in \wedge^nV$, there exists $\alpha_{T}\in\mathbb{F}$ such that
\[
T^{\wedge^n}(\tau) = \alpha_{T}\tau
\]
\item
Now we define
\[
\det(T) = \alpha_{T}
\]
This definition of determinant does not depend on any choice of basis of $V$.
\end{enumerate}
\end{definition}




\begin{example}\label{exp:15:1}
\begin{enumerate}
\item
Suppose that $T=I:V\to V$ be identity.
Take a basis $\{v_1,\dots,v_n\}$ of $V$, then
\[
T^{\wedge^n} (v_1\wedge\cdots\wedge v_n)
=
T(v_1)\wedge\cdots\wedge T(v_n)
\]
Or equivalently,
\[
\det(T)\cdot (v_1\wedge\cdots\wedge v_n) = v_1\wedge\cdots\wedge v_n
\]
Therefore, $\det(T)=1$.
\item
Suppose that $T:V\to V$ is diagonalizable with
$\{w_1,\dots,w_n\}$ forming eigen-basis of $T$.

As a result,
\[
T^{\wedge^n}(w_1\wedge\cdots\wedge w_n)
=
T(w_1)\wedge T(w_2)\cdots\wedge T(w_n),
\]
which implies
\[
\det(T)(w_1\wedge\cdots\wedge w_n)
=
(\lambda_1w_1)\wedge\cdots\wedge(\lambda_nw_n),
\]
which implies
\[
\det(T)w_1\wedge\cdots\wedge w_n=(\lambda_1\cdots\lambda_n)w_1\wedge\cdots\wedge w_n,
\]
i.e., $\det(T)=\lambda_1\cdots\lambda_n$.
\end{enumerate}
\end{example}


\begin{proposition}
Let $T,S:V\to V$ be linear transformations, then
\[
\begin{array}{ll}
(T\circ S)^{\wedge^p}:&\wedge^pV\to \wedge^pV\\
\text{with}&T^{\wedge^p},S^{\wedge^p}:\wedge^pV\to \wedge^pV
\end{array}
\]
satisfies
\[
(T\circ S)^{\wedge^p} = (T^{\wedge^p})\circ(S^{\wedge^p})
\]
\end{proposition}

\begin{proof}
Pick any basis $\{v_{i_1}\wedge\cdots\wedge v_{i_p}\mid 1\le i_1<\cdots<i_p\le n\}$ of $\wedge^pV$.
Then
\begin{align*}
(T\circ S)^{\wedge^p}(v_{i_1}\wedge\cdots\wedge v_{i_p})&=
(T\circ S)(v_{i_1})\wedge\cdots\wedge(T\circ S)(v_{i_p})
\end{align*}
On  the other hand,
\begin{align*}
(T^{\wedge^p})\circ(S^{\wedge^p})(v_{i_1}\wedge\cdots\wedge v_{i_p})
&=
(T^{\wedge^p})(S(v_{i_1})\wedge\cdots\wedge S(v_{i_p}))\\
&=
(T\circ S)(v_{i_1})\wedge\cdots (T\circ S)(v_{i_p})\\
\end{align*}


\end{proof}

\begin{corollary}
\[
\det(T\circ S) = \det(T)\det(S)
\]
\end{corollary}

\begin{proof}
Pick any basis $\{v_1\wedge\cdots\wedge v_n\}$ of $\wedge^nv$, then
\begin{align*}
\det(T\circ S)v_1\wedge\cdots\wedge v_n
&=(T\circ S)^{\wedge^n}v_1\wedge\cdots\wedge v_n\\
&=(T^{\wedge^n})\circ((S^{\wedge^n})v_1\wedge\cdots\wedge v_n)\\
&=(T^{\wedge^n})(\det(S)v_1\wedge\cdots\wedge v_n)\\
&=\det(S)T^{\wedge^n}(v_1\wedge\cdots\wedge v_n)\\
&=\det(S)\det(T)v_1\wedge\cdots\wedge v_n
\end{align*}
Therefore, $\det(T\circ S) = \det(T)\det(S)$.


\end{proof}

\begin{theorem}
Let $V=\mathbb{F}^n$, and 
\[
\begin{array}{ll}
T:&V\to V\\
\text{with}&T(\bm v)=\bm A\bm v,\quad\bm A\in M_{n\times n}(\mathbb{F})
\end{array}
\]
Then $\det(T)= \det(\bm A)$
\end{theorem}
\begin{proof}
Take $\{ e_1,\dots, e_n\}$ as the usual basis of $V\equiv\mathbb{F}^n$, then
\begin{align*}
\det(T)e_1\wedge\cdots\wedge e_n &= T( e_1)\wedge\cdots T(e_n)\\
&=a_1\wedge\cdots\wedge a_n
\end{align*}
where $a_i$ denotes the $i$-th column of $\bm A$.

As we have studied before [c.f. p141 in MAT2040 Notebook], the previous definition of determinant is based on three basic properties. It suffices to show these three basis properties:
\begin{enumerate}
\item
The determinant of the $n$ by $n$ identity matrix is 1:
See part~(1) in Example~(\ref{exp:15:1})
\item
The determianant changes sign when two columns~(w.l.o.g., ``rows'' are relaced with 
``columns'') are exchanged:
due to the sign reversal property for wedge product
\item
The determinant is a linear function of each column separately, i.e.,
\[
a_1\wedge\cdots\wedge(ta_i)\wedge\cdots\wedge a_n
=
t(a_1\wedge\cdots\wedge a_i\wedge\cdots\wedge a_n)
\]
\end{enumerate}
Once we verify these three properties, we conclude that the explicit formula for $\det(\bm A)$ is a special case for our new definition.
\end{proof}

Or we can come into the previous definition for determinant directly. 
For instance, consider the mapping
\[
\begin{array}{ll}
T:&\mathbb{R}^2\to\mathbb{R}^2\\
\text{with}&T\begin{pmatrix}
x\\y	
\end{pmatrix}
=
\begin{pmatrix}
a&b\\c&d
\end{pmatrix}\begin{pmatrix}
x\\y
\end{pmatrix}
\end{array}
\]
Then we imply
\begin{subequations}
\begin{align*}
\det(T)(e_1\wedge e_2)&=\begin{pmatrix}
a\\c
\end{pmatrix}\wedge\begin{pmatrix}
b\\d
\end{pmatrix}\\
&=(ae_1)\wedge(be_1)+(ae_1)\wedge(de_2)+(ce_2)\wedge(de_1)+(ce_2)\wedge(de_2)\\
&=(ad)e_1\wedge e_2 + (bc)e_2\wedge e_1\\
&=(ad-bc)e_1\wedge e_2
\end{align*}
\end{subequations}
Therefore, we imply $\det(T) = ad-bc$.









