\section{Wednesday for MAT3006}\index{Monday_lecture}

The quiz will be held on Wednesday.

\paragraph{Reviewing}Let's go through the proof for Weierstrass Theorem quickly.
\begin{itemize}
\item
Study $Q_n(x) = c_n(1-x^2)^n$ and construct the approximate function
\[
p_n(x) = \int_{-1}^1Q_n(t)f(x+t)\diff t
\]
\item
Show that 
\begin{align*}
|p_n(x) - f(x)|&\le\int_{-1}^1|f(x+t) - f(x)|Q_n(t)\diff t\\
&=
\left(\int_{\delta}^1+\int_{\delta}^{-\delta}+ \int_{-1}^{-\delta}\right)|f(x+t)-f(x)|Q_n(t)\diff t\\
&\le 4M\sqrt{n}(1-\delta^2)^n + \int_{\delta}^{-\delta}|f(x+t)-f(x)|Q_n(t)\diff t\\
&\le 4M\sqrt{n}(1-\delta^2)^n + \varepsilon\cdot\int_{\delta}^{-\delta}Q_n(t)\diff t\\
&\le 4M\sqrt{n}(1-\delta^2)^n + \varepsilon
\end{align*}
Therefore, $\|p_n-f\|_\infty\to0$ as $n\to\infty$.
\item
Generalization for $\forall g\in\mathcal{C}[0,1]$:
Recall that we have assumed $f(0)=f(1)=0$. Now consider the general case, say
\[
\begin{array}{ll}
g(0)=a,
&g(1)=b.
\end{array}
\]
Consider $f(x):=g(x) - l(x)$, where $l$ is the line segment from $(0,a)$ to $(1,b)$. Then we imply $|f(x) - p_n(x)|<\varepsilon$, i.e.,
\[
|g(x) - (p_n(x)+l(x))|<\varepsilon,\quad \forall x.
\]
\item
Generlization for $\forall h\in\mathcal{C}[a,b]$: Recall that we have restrict $f$ is continuous on $[0,1]$. For any $h\in\mathcal{C}[a,b]$, define $g(x) = h((b-a)x + a)$ for $x\in[0,1]$. Therefore, $g\in\mathcal{C}[0,1]$, i.e., $|g(y) - p_n(y)|<\varepsilon,\forall y\in[0,1]$, which implies
\[
|h((b-a)y+a) - p_n(y)|<\varepsilon,\quad \forall y\in[0,1]
\]
Applying change of variables with $x=(b-a)y+a$, we imply
\[
\left|h(x) - p_n\left(\frac{x-a}{b-a}\right)\right|<\varepsilon,\quad \forall x\in[a,b],
\]
where $p_n(\cdot)$ is a polynomial function.


\end{itemize}

\subsection{Stone-Weierstrass Theorem}
The motivation is to generalize the Weierstrass approximation into the space $\mathcal{C}(X)$, where $(X,d)$ is a general compact space. 
Here $\mathcal{C}(X):=\{f:X\to\mathbb{R}\text{ is continuous}\}$. Note that
\begin{itemize}
\item
$\mathcal{C}(X)$ has a norm:
\[
\|f\|_\infty :=\sup\{f(x)\mid x\in X\}
\]
This is well-defined, since $f(X)\subseteq\mathbb{R}$ is compact, i.e., closed and bounded.
\item
$(\mathcal{C}(X),d_\infty)$ is complete. The proof follows similarly from the proof that $\mathcal{C}[a,b]$ is complete (see Example~(\ref{exp:2:15})).
\end{itemize}
\begin{remark}
If $X$ is not compact, then the norm $\|\cdot\|_\infty$ is \emph{not} well-defined on $\mathcal{C}(X)$, but this norm is still well-defined on the space
\[
\mathcal{C}_{b}(X)=\{f:X\to\mathbb{R}\mid \text{$f$ is continous and bounded}\}.
\]
If $X$ is compact, then $\mathcal{C}(X)=\mathcal{C}_{b}(X)$.
\end{remark}

\begin{definition}[Separation Property]
Let $(X,d)$ be any metric space, and $\mathcal{A}\subseteq\mathcal{C}_b(X)$ is an algebra (closed under linear combination and pointwise product), then 
\begin{enumerate}
\item
$\mathcal{A}$ is said to be equipped with the \emph{separation property} if for any $x_1\ne x_2\in X$, there exists $f\in\mathcal{A}$ such that $f(x_1)\ne f(x_2)$
\item
$\mathcal{A}$ is said to be equipped with the \emph{nonvanishing property} if for any $x\in X$, there exists $f\in\mathcal{A}$ such that $f(x)\ne0$.
\end{enumerate}
\end{definition}
\begin{example}
Suppose that $X:=S^1:=\{e^{i\theta}\mid\theta\in[0,2\pi]\}\subseteq\mathbb{C}\cong\mathbb{R}^2$, and consider the algebra
\[
\mathcal{A}=\langle g\rangle:=\Span\{1,g,g^2,\dots\}
\]
Define $g:S^1\to\mathbb{R}$ as $g(e^{i\theta}) = \cos\theta$. Note that
\begin{enumerate}
\item
$\mathcal{A}$ does not satisfy the separation property: take $e^{i\theta},e^{i(2\pi-\theta)}$
\item
However, $\mathcal{A}$ satisfies the nonvanishing property. Consider the special element of $\mathcal{A}$: $f\equiv1$.
\end{enumerate}
\end{example}


\begin{theorem}[Stone-Weierstrass Theorem]
Let $(X,d)$ be a compact space, and $\mathcal{A}\subseteq\mathcal{C}(X)$ is an algebra.
Then $\overline{A}= \mathcal{C}(X)$ iff $A$ satisfies both the \emph{nonvanishing} and \emph{separation} property.
\end{theorem}
Before going through the proof, we establish two lemmas below:
\begin{proposition}\label{pro:4:12}
If both $f,g$ belong to the algebra $\mathcal{A}$, then
$\max\{f,g\}\in\mathcal{A}$ and $\min\{f,g\}\in\overline{\mathcal{A}}$.
\end{proposition}
\begin{proof}
Since
\begin{align*}
\max\{f,g\}&=\frac{1}{2}(f+g)+\frac{1}{2}|f-g|\\
\min\{f,g\}&=\frac{1}{2}(f+g)-\frac{1}{2}|f-g|,
\end{align*}
it suffices to show $|h|\in\overline{\mathcal{A}}$ given that $h\in\mathcal{A}$. 

Let $M=\max\{|h(x)|\mid x\in X\}$. Consider the function (w.r.t. $t$) $|t|\in\mathcal{C}[-M,M]$. By Weierstrass approximation, there exists a polynomial $p$ such that $||t| - p(t)|<\varepsilon$, which implies 
\[
||h(x)| - p(h(t))|<\varepsilon.
\]
Note that $p(h(t))$ is a polynomial of $h(t)$, and therefore an element from the algebra $\mathcal{A}$. 
Therefore, $|h|$ can be approximated by some element from $\mathcal{A}$, i.e., $|h|\in\overline{\mathcal{A}}$.
\end{proof}


\begin{proposition}\label{pro:4:13}
Let $\mathcal{A}\subseteq\mathcal{C}(X)$ be an algebra satisfying the separation property and non-vanishing property. 
Then for all $x_1\ne x_2\in X$, and any $\alpha,\beta\in\mathbb{R}$, 
there exists $f\in\mathcal{A}$ such that 
\[
\left\{
\begin{aligned}
f(x_1)&=\alpha\\
f(x_2)&=\beta
\end{aligned}
\right.
\]
\end{proposition}
\begin{proof}
By separation property, there exists $h\in\mathcal{A}$ such that 
$h(x_1)\ne h(x_2)$. 
\begin{enumerate}
\item
We claim that we can construct a new $h$ such that 
\begin{equation}\label{Eq:4:5}
\begin{array}{lll}
h(x_1)\ne h(x_2),& h(x_1)\ne0,&h(x_2)\ne0
\end{array}
\end{equation}
\begin{enumerate}
\item
If both $h(x_1),h(x_2)\ne0$, we have done.
\item
If not, suppose $h(x_1)=0$. By non-vanishing property, there eixsts $p\in\mathcal{A}$ such that $p(x_1)\ne0$. Then some linear transformation of $h$ and $p$ will do the trick. (hint: construct $t$ such that $h\leftarrow h + t\cdot p$ gives the desired result.)
\end{enumerate}
\item
Now suppose the requirement~(\ref{Eq:4:5}) is met. Consider the function 
\[
f(x)=ah(x)+bh^2(x)\in\mathcal{A},
\]
where $a,b$ are two parameters to be determined. 

Indeed, it suffices to find $a,b$ such that $f(x_1)=\alpha,f(x_2)=\beta$, or equivalently, solve the linear system
\begin{align*}
f(x_1)&=ah(x_1)+bh^2(x_1)=\alpha\\
f(x_2)&=ah(x_2)+bh^2(x_2)=\beta
\end{align*}
Since the determinant of the linear system is not equal to $0$, $a,b$ can be clearly found.
\end{enumerate}
The proof is complete.
\end{proof}

\begin{proof}[Necessity part of the proof]
Given that $\mathcal{A}$ has separation and non-vanishing,
we aim to show $\overline{\mathcal{A}}=\mathcal{C}(X)$.
\begin{enumerate}
\item
Take any $f\in\mathcal{C}(X)$. 
By proposition~(\ref{pro:4:13}), for any $x,y\in X$, there exists $\phi_{x,y}\in\mathcal{A}$ such that
\[
\left\{
\begin{aligned}
\phi_{x,y}(x)&=f(x)\\
\phi_{x,y}(y)&=f(y)
\end{aligned}
\right..
\]
Construct the open set $U_{x,y} = (f-\phi_{x,y})^{-1}((-\varepsilon,\varepsilon))$, i.e., 
\[
U_{x,y} = \{t\in X\mid \phi_{x,y}(t)-\varepsilon
<f(t)<
\phi_{x,y}(t)+\varepsilon
\}.
\]
\item
It's clear that $x,y\in U_{x,y}$. For fixed $y\in X$, the collection $\{U_{x,y}\}_{x\in X}$ forms an open cover of $X$. By the compactness of $X$, there exists the finite subcover
\[
\{U_{x_1,y},\dots,U_{x_N,y}\}\supseteq X.
\]
By proposition~(\ref{pro:4:12}), the function $\phi_y:=\max\{\phi_{x_1,y},\dots,\phi_{x_N,y}\}\in\overline{\mathcal{A}}$. Furthermore, for $\forall x\in X$, we imply there exists some $U_{x_i,y}\ni x$, i.e.,
\[
f(x)<\phi_{x_i,y}(x)+\varepsilon\implies
f(x)<\phi_y(x)+\varepsilon, \ \forall x\in X.
\]
\item
Also, consider $V_y = \cap_{i=1}^NU_{x_i,y}$, which is the open set containing $y$, and $\{V_y\}_{y\in X}$ covers $X$ (why?). Note that for any $x\in V_y$, we imply $x\in U_{x_i,y},\forall i$, i.e.,
\[
\phi_{x_i,y}(x)-\varepsilon<f(x),\quad \forall i\implies
\phi_y(x)-\varepsilon<f(x),\ \forall x\in V_y.
\]
By the compactness of $X$ again, we take finite subcover $\{V_{y_j}\}_{j=1}^M$ and define
\[
\phi(x):=\min\{\phi_{y_1}(x),\dots,\phi_{y_M}(x)\}\in\overline{\mathcal{A}}.
\]
Therefore, for any $x\in X$ we imply $x\in V_{y_m}$, i.e.,
\begin{equation}\label{Eq:4:6}
\phi_{y_m}(x)-\varepsilon<f(x)\implies\phi(x)-\varepsilon<f(x)
\end{equation}
\item
Also, from (2) we have obtained $f(x)<\phi_y(x)+\varepsilon$ for $\forall y\in X$. In particular,
\begin{equation}\label{Eq:4:7}
f(x)<\phi_{y_m}(x)+\varepsilon,\ \forall m=1,\dots,M
\end{equation}
Combining (\ref{Eq:4:6}) and (\ref{Eq:4:7}), we imply $|\phi(x)-f(x)|<\varepsilon$.
\end{enumerate}
Therefore, we have constructed a function $\phi\in\overline{\mathcal{A}}$ such that $|\phi(x)-f(x)|<\varepsilon$, which implies $f\in\overline{\overline{\mathcal{A}}}=\overline{\mathcal{A}}$. The proof is complete.
\end{proof}























