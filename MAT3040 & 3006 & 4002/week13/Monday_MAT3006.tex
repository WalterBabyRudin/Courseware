
\section{Monday for MAT3006}\index{Monday_lecture}

\paragraph{Notations}
In this lecture, we let $\int_If(x,y)\diff x$ denote the Lebesgue integral.

\begin{theorem}
Let $I,J$ be intervals in $\mathbb{R}$, and $f:I\times J\to\mathbb{R}$ be a function such that
\begin{enumerate}
\item
For fixed $y\in J$, the function $f(x):=f(x,y)$ is integrable on $I$
\item
$\frac{\partial f}{\partial y}$ exists for any $(x,y)\in I\times J$
\item
$\left|\frac{\partial f}{\partial y}(x,y)\right|\le g(x)$ for some integrable function $g(x)$ on $I$.
\end{enumerate}
Then $F(y):=\int_If(x,y)\diff x$ is differentiable on $J$, with
\[
F'(y)=\int_I\frac{\partial f}{\partial y}(x,y)\diff x
\]
\end{theorem}
\begin{proof}
Fix $y\in J$, and consider any sequence $\{y_n\}$ (with $y_n\ne y$) in $J$ converging to $y$.

Construct the function
\[
g_n(x):=\frac{f(x,y_n) - f(x,y)}{y_n-y}
\]
which follows that 
\begin{enumerate}
\item
The function $g_n$ is integrable by hypothesis (1)
\item
The function $g_n(x)$ converges to $\frac{\partial f}{\partial y}(x,y)$ as $n\to\infty$
\item
By MVT, $|g_n(x)|=|\frac{\partial f}{\partial y}(x,\xi)|$, which is bounded by $g(x)$ by hypothesis (3).
\end{enumerate}
Therefore, the DCT applies, and 
\[
\int_Ig_n(x)\diff x
=
\frac{1}{y_n-y}\left[\int f(x,y_n)\diff x - \int f(x,y)\diff x\right]
\to\int_I\frac{\partial f}{\partial y}(x,y)\diff x
\]
In other words, for all sequences $\{y_n\}\to y$ with $y_n\ne y$,
\[
\lim_{n\to\infty}\frac{F(y_n) - F(y)}{y_n-y}=\int_I\frac{\partial f}{\partial y}(x,y)\diff x
\]
From the elementary analysis knowledge, in particular, $\lim_{y'\to y}H(y')$ exists~(equal to $L$) if and only if 
$\lim_{n\to\infty}H(y_n)=L$
for all sequences $\{y_n\}\to y$ with $y_n\ne y$.
Therefore,
\[
F'(y):=\lim_{y'\to y}\frac{F(y')-F(y)}{y'-y}=\int_I\frac{\partial f}{\partial y}(x,y)\diff x.
\]
\end{proof}

\subsection{Double Integral}
\begin{definition}[Measure in $\mathbb{R}^2$]
In $\mathbb{R}^2$, we can define the \emph{measure} of the rectangle 
$A\times B\subseteq\mathbb{R}^2$ with $A,B\in\mathcal{M}$ by
\[
m^*(A\times B) = m(A)m(B)
\]
In particular, we define
\[
x\cdot\infty=\infty\cdot x=(-x)\cdot(-\infty)=\left\{
\begin{aligned}
\infty,&\quad\text{if $x>0$}\\
-\infty,&\quad\text{if $x<0$}\\
0,&\quad\text{if $x=0$}
\end{aligned}
\right.
\]
\end{definition}
\begin{definition}[Outer Measure in $\mathbb{R}^2$]
Then the outer measure of any $E\subseteq\mathbb{R}^2$ is defined as
\[
m^*(E):=\inf\left\{
\sum_{i=1}^\infty m(R_i)\middle|
E\subseteq\bigcup_{i=1}^\infty R_i,
R_i = A_i\times B_i, A_i,B_i\in\mathcal{M}
\right\}
\]
\end{definition}
\begin{definition}[Lebesgue Measurable in $\mathbb{R}^2$]
A subset $E\subseteq\mathbb{R}^2$ is Lebesgue measurable if $E$ satisfies the Carathedory Property:
\[
m^*(A) = m^*(A\cap E)+m^*(A\setminus E),
\]
for any subset $A\subseteq\mathbb{R}^2$.
\end{definition}

\paragraph{Product Space of $\mathbb{R}^2$}
Given two measurable spaces $(X,\mathcal{A},\mu)$ and $(Y,\mathcal{B},\lambda)$, in particular, we are interested in 
\[
(X,\mathcal{A},\mu)=(Y,\mathcal{B},\lambda)=(\mathbb{R},\mathcal{M},m).
\]
Now we want to construct another measurable space in $X\times Y:=\mathbb{R}^2$.
\begin{enumerate}
\item
Start from the ``measurable rectangles''
\[
\mathcal{A}\times\mathcal{B}
=
\{A\times B\mid A\in\mathcal{A},B\in\mathcal{B}\}
\]
\item
Define the function $\pi:\mathcal{A}\times\mathcal{B}\to[0,\infty]$ by
\[
\pi(A\times B)= \mu(A)\lambda(B).
\]
\item
Let $\mathcal{A}\otimes \mathcal{B}$ be the smallest $\sigma$-algebra containing $\mathcal{A}\times\mathcal{B}$.
Then by Caratheodory extension theorem, we can extend $\pi:\mathcal{A}\times\mathcal{B}\to[0,\infty]$ to $\tilde{\pi}:\mathcal{A}\otimes \mathcal{B}\to[0,\infty]$ such that 
\begin{enumerate}
\item
$(X\times Y,\mathcal{A}\otimes\mathcal{B},\tilde{\pi})$ is a measurable space
\item
$\tilde{\pi}\mid_{\mathcal{A}\times\mathcal{B}}=\pi$.
\end{enumerate}
\end{enumerate}

\begin{remark}
\begin{itemize}
\item
If further we have $\mathcal{A}$ and $\mathcal{B}$ are $\sigma$-finite, i.e., there exists $E_i\in\mathcal{A}$ such that $X=\cup_{i=1}^\infty E_i$, $\mu(E_i)<\infty,\forall i$, then we can imply the extension $\tilde{\pi}$ is unique.

(For instance, $\mathbb{R}=\cup_{n\in\mathbb{Z}}[n,n+1]$ and $m([n,n+1]) = 1<\infty$, i.e., $(\mathbb{R},\mu,m)$ is $\sigma$-finite.)
\item
Question: 
we have constructed two measurable space $(\mathbb{R}\times\mathbb{R},\mathcal{M}\otimes\mathcal{M},\tilde{\pi})$
and
$(\mathbb{R}^2,\mathcal{M}_{\mathbb{R}^2},m)$.
Are they the same?

Answer : no, but the latter can be obtained from the former by completion process.
In particular,
\[
m\mid_{\mathcal{M}\otimes\mathcal{M}}=\tilde{\pi}.
\]
\end{itemize}
\end{remark}

Let's study the measurable space $(\mathbb{R}\times\mathbb{R},\mathcal{M}\otimes\mathcal{M},\pi)$ first,
where $f:\mathbb{R}^2\to[-\infty,\infty]$ is a measurable function, i.e.,
$f^{-1}((a,\infty])\in\mathcal{A}\otimes\mathcal{B}$.
In particular, we say $E\subseteq \mathbb{R}\times\mathbb{R}$ is measurable if $E\in\mathcal{M}\otimes\mathcal{M}$ for the moment being (but we will generalize the notion of measurable into $\mathcal{M}_{\mathbb{R}^2}$ in the future).


\begin{definition}[$x$-section and $y$-section]
Let $E\subseteq X\times Y$, with $(x,y)\in E$.
Define
\begin{itemize}
\item
the $x$-section $E_x = \{y\in Y\mid (x,y)\in E\}$, for fixed $x\in X$
\item
the $y$-section $E_y = \{x\in X\mid (x,y)\in E\}$, for fixed $y\in Y$.
\end{itemize}
\end{definition}

\begin{proposition}
Suppose that $E\subseteq X\times Y$ is measurable (i.e., $E\in\mathcal{A}\otimes\mathcal{B}$), then $E_x\in\mathcal{B}$ and $E_y\in\mathcal{A}$.
\end{proposition}
\begin{proof}
Construct the set $\mathfrak{A}=\{E\in\mathcal{A}\otimes\mathcal{B}\mid E_x\in\mathcal{B}\}.$
It suffices to show $\mathfrak{A}=\mathcal{A}\otimes\mathcal{B}$.
We claim that
\begin{enumerate}
\item
$\mathfrak{A}$ is a $\sigma$-algebra
\item
$\mathfrak{A}$ contains all $A\times B\in\mathcal{A}\times\mathcal{B}$
\end{enumerate}
If the claim~(1) and (2) hold, and since $\mathcal{A}\otimes\mathcal{B}$ is the smallest-$\sigma$-algebra containing $\mathcal{A}\times\mathcal{B}$,
we imply $\mathcal{A}\otimes\mathcal{B}\subseteq\mathfrak{A}\subseteq\mathcal{A}\otimes\mathcal{B}$, i.e., the proof is complete.



\begin{enumerate}
\item
\begin{enumerate}
\item
Note that $\emptyset\in\mathfrak{A}$, and 
$X\times Y\in\mathfrak{A}$
 since 
$(X\times Y)_x = Y\in\mathcal{B}$.
\item
Suppose that $E_i\in\mathfrak{A}$, $i\ge1$, i.e., $(E_i)_x\in\mathcal{B}$.
Observe that
\[
\left(
\bigcup_{i=1}^\infty E_i
\right)_x
=
\bigcup_{i=1}^\infty
(E_i)_x\in\mathcal{B},
\]
since $\mathcal{B}$ is a $\sigma$-algebra.
Therefore, $\cup_{i=1}^\infty E_i\in\mathfrak{A}$.
\item
Suppose that $E\in\mathfrak{A}$, i.e., $(E)_x\in\mathcal{B}$, then
\begin{align*}
(E^c)_x &=\{y\mid (x,y)\in E^c\}\\
&=\{y\mid (x,y)\notin E\}\\
&=(E_x)^c\in\mathcal{B}
\end{align*}
which implies $E^c\in\mathfrak{A}$.
\end{enumerate}
\item
For any $A\times B\in\mathcal{A}\times\mathcal{B}$, 
since $(A\times B)_x=B\in\mathcal{B}$, we imply $(A\times B)\in\mathfrak{A}$.
\end{enumerate}


In conclusion, $\mathfrak{A}=\mathcal{A}\otimes\mathcal{B}$.
For all $E\in \mathcal{A}\otimes\mathcal{B}$, we imply $E\in \mathfrak{A}$, i.e., $E_x\in\mathcal{B}$.
\end{proof}


\begin{proposition}
Sippose that $f:X\times Y\to[-\infty,\infty]$ is measurable.
(i.e., $f^{-1}((a,\infty])\in\mathcal{A}\otimes\mathcal{B}$), then the maps 
\[\left\{
\begin{array}{ll}
f_x:&Y\to[-\infty,\infty]\\
\text{with}&f_x(y):=f(x,y)
\end{array}\right.,\qquad\left\{
\begin{array}{ll}
f_y:&X\to[-\infty,\infty]\\
\text{with}&f_y(x):=f(x,y)
\end{array}\right.
\]
are measurable.
More precisely, $f_x^{-1}((a,\infty])\in\mathcal{B}$ and $f_y^{-1}((a,\infty])\in\mathcal{A}$.
\end{proposition}

\begin{proof}
\begin{align*}
f_x^{-1}((a,\infty])&=\{y\in Y\mid f_x(y)\in(a,\infty]\}\\
&=\{y\in Y\mid f(x,y)>a\}\\
&=\{(u,y)\in X\times Y\mid f(u,y)>a\}_x\\
&=(f^{-1}((a,\infty]))_x\in\mathcal{B}
\end{align*}
\end{proof}



















