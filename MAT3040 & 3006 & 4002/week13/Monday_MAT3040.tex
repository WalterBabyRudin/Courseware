\chapter{Week13}
\section{Monday for MAT3040}\index{Monday_lecture}
\paragraph{Reviewing}


\begin{enumerate}
\item
Define $S=\{(\bm v,\bm w)\mid \bm v\in V,\bm w\in W\}$ and $\mathfrak{X}=\Span(S)$.
In $\mathfrak{X}$, there are no relations between distinct elements of $S$, e.g.,
\[
2(\bm v,0)+3(0,\bm w)\ne1(2\bm v,3\bm w)
\]
General element in $\mathfrak{X}$:
\[
a_1(\bm v_1,\bm w_1)+\cdots+a_n(\bm v_n,\bm w_n), 
\]
where $(\bm v_i,\bm w_i)$ are distinct.
\item
Define the space $V\otimes W=\mathfrak{X}/y$, with 
\[
\bm v\otimes\bm w = 1(\bm v,\bm w)+y\in V\otimes W.
\]
General element in $\mathfrak{X}/y:=V\otimes W$:
\begin{align*}
a_1(\bm v_1,\bm w_1)+\cdots+a_n(\bm v_n,\bm w_n)+y
&=
a_1((\bm v_1,\bm w_1)+y)+\cdots+a_n((\bm v_n,\bm w_n)+y)\\
&=
a_1(\bm v_1\otimes\bm w_1)+\cdots+a_n(\bm v_n\otimes\bm w_n)\\
&=
(a_1\bm v_1)\otimes\bm w_1 + \cdots+(a_n\bm v_n)\otimes\bm w_n
\end{align*}
Therefore, a general element in $V\otimes W$ is of the form
\begin{equation}\label{Eq:13:1}
\bm v_1'\otimes\bm w_1+\cdots+\bm v_n'\otimes\bm w_n,\ 
\bm v_i'\in V, \bm w_i\in W.
\end{equation}
Note that 
$V\otimes W$ is different from $V\times W$, where all elements in $V\times W$ can be expressed as $(\bm v,\bm w)$.
\item
The tensor product mapping 
\[
\begin{array}{ll}
i:&V\times W\to V\otimes W\\
\text{with}&(\bm v,\bm w)\mapsto\bm v\otimes\bm w
\end{array}
\]
satisfies the universal property.
\end{enumerate}

Here we present an example for computing tensor product by making use of the rules below:
\begin{align*}
(\bm v_1+\bm v_2)\otimes\bm w&=\bm v_1\otimes\bm w + \bm v_2\otimes\bm w\\
\bm v\otimes(\bm w_1+\bm w_2)&=(\bm v\otimes\bm w_1)+(\bm v\otimes\bm w_2)\\
(k\bm v)\otimes\bm w&=k(\bm v\otimes\bm w)\\
\bm v\otimes(k\bm w)&=k(\bm v\otimes\bm w)
\end{align*}

\begin{example}
Let $V=W=\mathbb{R}^2$, with
\[
\bm e_1=\begin{pmatrix}
1\\0
\end{pmatrix},
\quad
\bm e_2=\begin{pmatrix}
0\\1
\end{pmatrix}.
\]
Here we have
\begin{align*}
\begin{pmatrix}
3\\1
\end{pmatrix}\otimes\begin{pmatrix}
-4\\2
\end{pmatrix}
&=(3\bm e_1+2\bm e_2)\otimes(-4\bm e_1+2\bm e_2)\\
&=(3\bm e_1)\otimes (-4\bm e_1+2\bm e_2)
+
(\bm e_2)\otimes(-4\bm e_1+2\bm e_2)\\
&=(3\bm e_1)\otimes(-4\bm e_1)
+
(3\bm e_1)\otimes(2\bm e_2)
+
(\bm e_2)\otimes(-4\bm e_1)
+
\bm e_2\otimes(2\bm e_2)\\
&=-12(\bm e_1\otimes\bm e_1)
+
6(\bm e_1\otimes\bm e_2)
-4
(\bm e_2\otimes\bm e_1)
+
2(\bm e_2\otimes\bm e_2)
\end{align*}
\end{example}

Exercise:
Check that $\bm e_1\otimes\bm e_2+\bm e_2\otimes\bm e_1$ cannot be re-written as 
\[
\begin{array}{ll}
(a\bm e_1+b\bm e_2)\otimes(c\bm e_1+d\bm e_2),
&
a,b,c,d\in\mathbb{R}.
\end{array}
\]

\subsection{Basis of $V\otimes W$}
\paragraph{Motivation}
Given that $\{\bm v_1,\dots,\bm v_n\}$ is a basis of $V$,
and 
$\{\bm w_1,\dots,\bm w_m\}$ a basis of $W$,
we aim to find a basis of $V\otimes W$ using $\bm v_i$'s and $\bm w_i$'s.

\begin{proposition}\label{pro:13:1}
The set 
$\{\bm v_i\otimes\bm w_j\mid 1\le i\le n, 1\le j\le m\}$
spans 
the tensor product space
$V\otimes W$.
\end{proposition}
\begin{proof}
Consider any $\bm v\in V$ and $\bm w\in W$, and we want to express $\bm v\otimes\bm w$ in terms of $\bm v_i,\bm w_j$.
Suppose that
$\bm v=\alpha_1\bm v_1+\cdots+\alpha_n\bm v_n$ and $\bm w=\beta_1\bm w_1+\cdots+\beta_m\bm w_m$.

Substituting $\bm v=\alpha_1\bm v_1+\cdots+\alpha_n\bm v_n$ into the expression 
$\bm v\otimes\bm w$, we imply
\begin{align*}
\bm v\otimes\bm w&=
(\alpha_1\bm v_1+\cdots+\alpha_n\bm v_n)\otimes\bm w\\
&=(\alpha_1\bm v_1)\otimes\bm w_1
+
\cdots
+
(\alpha_n\bm v_n)\otimes\bm w_n\\
&=\alpha_1(\bm v_1\otimes\bm w)
+
\cdots
+
\alpha_n(\bm v_n\otimes\bm w)
\end{align*}
For each $\bm v_i\otimes\bm w$, $i=1,\dots,n$, similarly, 
\begin{equation*}
\bm v_i\otimes\bm w
=
\beta_1(\bm v_i\otimes\bm w_1)
+
\cdots
+
\beta_m(\bm v_i\otimes\bm w_m).
\end{equation*}
Therefore, 
\begin{equation}\label{Eq:13:2}
\bm v\otimes\bm w
=
\sum_{i=1}^n\sum_{j=1}^m
\alpha_i\beta_j
(\bm v_i\otimes\bm w_j)
\end{equation}
By (\ref{Eq:13:1}), any vector in $V\otimes W$ is of the form 
\[
\bm v^{(1)}\otimes\bm w^{(1)}+\cdots
+
\bm v^{(\ell)}\otimes\bm w^{(\ell)}
\]
By (\ref{Eq:13:2}), each $\bm v^{(k)}\otimes\bm w^{(k)}, k=1,\dots,\ell$,
can be expressed as
\[
\bm v^{(k)}\otimes\bm w^{(k)}
=
\sum_{i=1}^n\sum_{j=1}^m
\alpha_i^{(k)}\beta_j^{(k)}(\bm v_i\otimes\bm w_j)
\]
Therefore,
\begin{align*}
\bm v^{(1)}\otimes\bm w^{(1)}+\cdots
+
\bm v^{(\ell)}\otimes\bm w^{(\ell)}
&=
\sum_{k=1}^\ell\sum_{i=1}^n\sum_{j=1}^m\alpha_i^{(k)}\beta_j^{(k)}(\bm v_i\otimes\bm w_j)
\end{align*}
In other words, $\{\bm v_i\otimes\bm w_j\mid 1\le i\le n, 1\le j\le m\}$ spans $V\otimes W$.
\end{proof}

\begin{theorem}
A basis of $V\otimes W$ is $\{\bm v_i\otimes\bm w_j\mid 1\le i\le n, 1\le j\le m\}$
\end{theorem}

\begin{proof}
By proposition~(\ref{pro:13:1}), 
it suffices to show that the set $\{\bm v_i\otimes\bm w_j\mid 1\le i\le n, 1\le j\le m\}$ is linear independent.
Suppose that
\begin{equation}\label{Eq:13:3}
\sum_{i=1}^n\sum_{j=1}^n\alpha_{ij}(\bm v_i\otimes\bm w_j)=\bm0
\end{equation}
Suppose that $\{\phi_1,\dots,\phi_n\}$ is a dual basis of $V^*$, and $\{\psi_1,\dots,\psi_m\}$ is a dual basis of $W^*$.
Construct the mapping
\[
\begin{array}{ll}
\pi_{p,q}:&V\times W\to\mathbb{F}\\
\text{with}&\pi_{p,q} = \phi_p(\bm v)\psi_q(\bm w)
\end{array}
\]
\begin{itemize}
\item
The mapping $\pi_{p,q}$ is actually bilinear: for instance,
\begin{align*}
\pi_{p,q}(a\bm v_1+b\bm v_2,\bm w)
&=
\phi_p(a\bm v_1+b\bm v_2)\psi_q(\bm w)\\
&=(a\phi_p(\bm v_1)+b\phi_p(\bm v_2))\psi_q(\bm w)\\
&=a\phi_p(\bm v_1)\psi_q(\bm w)+b\phi_p(\bm v_2)\psi_q(\bm w)\\
&=a\pi_{p,q}(\bm v_1,\bm w)+b\pi_{p,q}(\bm v_2,\bm w).
\end{align*}
Following the similar ideas, we can check that $\pi_{p,q}(\bm v,a\bm w_1+b\bm w_2)=a\pi_{p,q}(\bm v,\bm w_1)+b\pi_{p,q}(\bm v,\bm w_2).$
\item
Therefore, $\pi_{p,q}\in\text{Obj}$. By the universal property of the tensor product, $\pi_{p,q}$ induces the unique linear transformation
\[
\begin{array}{ll}
\prod_{p,q}:&V\otimes W\to \mathbb{F}\\
\text{with}&\prod_{p,q}(\bm v\otimes\bm w) = \pi_{p,q}(\bm v,\bm w)
\end{array}
\]
In other words, $\prod_{p,q}(\bm v\otimes\bm w) = \phi_p(\bm v)\psi_q(\bm w)$.
\item
Applying the mapping $\Pi_{p,q}$ on both sides of (\ref{Eq:13:3}), we imply
\[
\Pi_{p,q}
\left(
\sum_{i=1}^n\sum_{j=1}^n\alpha_{ij}(\bm v_i\otimes\bm w_j)
\right)
=
\Pi_{p,q}(\bm0)
\]
Or equivalently,
\[
\sum_{i=1}^n\sum_{j=1}^n\alpha_{ij}\Pi_{p,q}(\bm v_i\otimes\bm w_j)=0,
\]
i.e.,
\[
\sum_{i=1}^n\sum_{j=1}^n\alpha_{ij}\phi_p(\bm v_i)\psi_q(\bm w_j)=\alpha_{p,q}=0
\]
Following this procedure, we can argue that $\alpha_{ij}=0,\forall i, \forall j$.
\end{itemize}
\end{proof}

\begin{corollary}
If $\dim(V),\dim(W)<\infty$, then $\dim(V\otimes W)=\dim(V)\dim(W)$
\end{corollary}
\begin{proof}
Check dimension of the basis of $V\otimes W$. 
\end{proof}


\begin{remark}
The universal property can be very helpful.
In particular, given a bilinear mapping, say $\phi: V\times W\to U$, we imply $\phi\in\text{Obj}$.
By theorem~(\ref{The:12:3}), since $i$ satisfies the universal property of tensor product, we can induce an unique linear transformation $\psi:V\otimes W\to U$.
\end{remark}

Let's try another example for making use of the universal property:
\begin{theorem}
For finite dimension $U$ and $V$,
\[
V\otimes U\cong U\otimes V
\]
\end{theorem}
\begin{proof}
Construct the mapping 
\[
\begin{array}{ll}
\phi:&V\times U\to U\otimes V\\
\text{with}&\phi(\bm v,\bm u) = \bm u\otimes \bm v
\end{array}
\]
Indeed, $\phi$ is bilinear: for instance,
\begin{align*}
\phi(a\bm v_1+b\bm v_2,\bm u)&=
u\otimes(a\bm v_1+b\bm v_2)\\
&=a(\bm u\otimes\bm v_1)+b(u\otimes\bm v_2)\\
&=a\phi(\bm v_1,\bm u)+b\phi(\bm v_2,\bm u)
\end{align*}

Therefore, $\phi\in\text{Obj}$. By the universal property of tensor product, we induce an unique linear transformation
\[
\begin{array}{ll}
\Phi:&V\otimes U\to U\otimes V\\
\text{with}&\Phi(\bm v\otimes\bm u)=\bm u\otimes\bm v
\end{array}
\]
Similarly, we may induce the linear transformation
\[
\begin{array}{ll}
\Psi:&U\otimes V\to V\otimes U\\
\text{with}&\Psi(\bm u\otimes\bm v)=\bm v\otimes\bm u
\end{array}
\]
Given any $\sum_i\bm u_i\otimes\bm v_i\in U\otimes V$, observe that 
\begin{align*}
(\Phi\circ\Psi)\left(\sum_i\bm u_i\otimes\bm v_i\right)
&=
\Phi\left(\sum_i\Psi(\bm u_i\otimes\bm v_i)\right)\\
&=
\Phi\left(\sum_i\bm v_i\otimes\bm u_i\right)\\
&=
\sum_i\Phi\left(\bm v_i\otimes\bm u_i\right)\\
&=
\sum_i\bm u_i\otimes\bm v_i
\end{align*}
Therefore, $\Phi\circ\Psi=\text{id}_{U\otimes V}$.
Similarly, $\Psi\circ\Phi = \text{id}_{V\otimes U}$.
Therefore,
\[
U\otimes V\cong
V\otimes U.
\]
\end{proof}

\subsection{Tensor Product of Linear Transformation}
\paragraph{Motivation}
Given two linear transformations $T:V\to V'$ and $S:W\to W'$, we want to construct the tensor product 
\[
T\otimes S:
V\otimes W
\to V'\otimes W'
\]
Question: is $T\otimes S$ a linear transformation?

Answer: Yes.
Universal property plays a role!














