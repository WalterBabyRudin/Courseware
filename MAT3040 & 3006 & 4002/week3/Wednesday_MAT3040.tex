
\section{Wednesday for MAT3040}\index{Wednesday_lecture}
\subsection{Remarks for the Change of Basis}
\paragraph{Reviewing}
\begin{itemize}
\item
$[\cdot]_{\mathcal{A}}:V\to\mathbb{F}^n$ denotes coordinate vector mapping
\item
Change of Basis matrix: $\mathcal{C}_{\mathcal{A}',\mathcal{A}}$
\item
$T:V\to W$, $\mathcal{A}=\{\bm v_1,\dots,\bm v_n\}$ and $\bm B=\{\bm w_1,\dots,\bm w_m\}$.

$\text{Hom}_{\mathbb{F}}(V,W)\to M_{m\times n}(\mathbb{F})$
\end{itemize}

\begin{example}
Let $V=\mathbb{P}_3[x]$ and $\mathcal{A}=\{1,x,x^2,x^3\}$.

Let $T:V\to V$ defined as $p(x)\mapsto p'(x)$:
\[
\left\{
\begin{aligned}
T(1)&=0\cdot1+0\cdot x+0\cdot x^2+0\cdot x^3\\
T(x)&=1\cdot1+0\cdot x+0\cdot x^2+0\cdot x^3\\
T(x^2)&=0\cdot1+2\cdot x+0\cdot x^2+0\cdot x^3\\
T(x^3)&=0\cdot1+0\cdot x+3\cdot x^2+0\cdot x^3\\
\end{aligned}
\right.
\]

We can define the change of basis matrix for a linear transformation $T$ as well, w.r.t. $\mathcal{A}$ and $\mathcal{A}$:
\[
\mathcal{C}_{\mathcal{A},\mathcal{A}}=\begin{pmatrix}
0&1&0&0\\0&0&2&0\\0&0&0&3\\0&0&0&0
\end{pmatrix}
\]

Also, we can define a different basis $\mathcal{A}'=\{x^3,x^2,x,1\}$ for the output space for $T$, say $T:V_\mathcal{A}\to V_{\mathcal{A}'}$:
\[
(T)_{\mathcal{A},\mathcal{A}'}
=\begin{pmatrix}
0&0&0&0\\0&0&0&3\\0&0&2&0\\0&1&0&0
\end{pmatrix}
\]

Our observation is that the corresponding coordinate vectors before and after linear transformation admits a matrix multiplication:

\begin{align*}
(2x^2+4x^3)&\xrightarrow{T} ((4x+12x^2))\\
(2x^2+4x^3)_{\mathcal{A}}
=
\begin{pmatrix}
0\\0\\2\\4
\end{pmatrix}&\qquad
(4x+12x^2)_{\mathcal{A}}
=
\begin{pmatrix}
0\\4\\12\\0
\end{pmatrix}\\
\begin{pmatrix}
0&1&0&0\\0&0&2&0\\0&0&0&3\\0&0&0&0
\end{pmatrix}\begin{pmatrix}
0\\0\\2\\4
\end{pmatrix}&=\begin{pmatrix}
0\\4\\12\\0
\end{pmatrix}\\
\mathcal{C}_{\mathcal{A}\mathcal{A}}\cdot
(2x^2+4x^3)_{\mathcal{A}}
&=
(4x+12x^2)_{\mathcal{A}}
\end{align*}
\end{example}

\begin{theorem}[Matrix Representation]\label{The:3:3}
Let $T:V\to W$ be a linear transformation of finite dimensional vector sapces. Let $\mathcal{A},\mathcal{B}$ the ordered basis of $V,W$, respectively.
Then the following diagram holds:
\begin{figure}[H]
\centering
\includegraphics[width=6cm]{week3/p_2}
\caption{Diagram for the matrix reprentation, where $n:=\dim(V)$ and $m:=\dim(W)$}
\end{figure}
namely, for any $\bm v\in V$,
\[
(T)_{\mathcal{B},\mathcal{A}}(\bm v)_{\mathcal{A}}
=
(T\bm v)_{\mathcal{B}}
\]
Therefore, we can compute $T\bm v$ by matrix multiplication.
\end{theorem}
Therefore, linear transformation corresponds to coordinate matrix multiplication.

\begin{proof}
Suppose $\mathcal{A}=\{\bm v_1,\dots,\bm v_n\}$ and $\mathcal{B}=\{\bm w_1,\dots,\bm w_n\}$.  The proof of this theorem follows the same procedure of that in Theorem~(\ref{The:3:1})
\begin{enumerate}
\item
We show this result for $\bm v=\bm v_j$ first:
\begin{align*}
\text{LHS}&=[\alpha_{ij}]\bm e_j=
\begin{pmatrix}
\alpha_{1j}\\\vdots\\\alpha_{nj}
\end{pmatrix}
\\
\text{RHS}&=(T\bm v_j)_{\mathcal{B}}
=
\left(
\sum_{i=1}^m\alpha_{ij}\bm w_i
\right)_{\mathcal{B}}
=
\begin{pmatrix}
\alpha_{1j}\\\vdots\\\alpha_{nj}
\end{pmatrix}
\end{align*}
\item
Then we show the theorem holds for any $\bm v:=\sum_{j=1}^nr_j\bm v_j$ in $V$:
\begin{subequations}
\begin{align}\label{Eq:3:8}
(T)_{\mathcal{B}\mathcal{A}}(\bm v)_{\mathcal{A}}
&=
(T)_{\mathcal{B}\mathcal{A}}
\left(
\sum_{j=1}^nr_j\bm v_j
\right)_{\mathcal{A}}\\
&=
(T)_{\mathcal{B}\mathcal{A}}
\left(
\sum_{j=1}^nr_j(\bm v_j)_{\mathcal{A}}
\right)\\
&=
\sum_{j=1}^nr_j(T)_{\mathcal{B}\mathcal{A}}(\bm v_j)_{\mathcal{A}}\\
&=
\sum_{j=1}^nr_j(T\bm v_j)_{\mathcal{B}}\\
&=
\left(
\sum_{j=1}^nr_j(T\bm v_j)
\right)_{\mathcal{B}}\\
&=
\left[
T
(
\sum_{j=1}^nr_j\bm v_j
)
\right]_{\mathcal{B}}\\
&=(T\bm v)_{\mathcal{B}}
\end{align}
\end{subequations}
The justification for (\ref{Eq:3:8}) is similar to that shown in Theorem~(\ref{The:3:1}). The proof is complete.
\end{enumerate}





\end{proof}

\begin{remark}
Consider a special case for Theorem~(\ref{The:3:3}), i.e., $T=\text{id}$ and $\mathcal{A},\mathcal{A}'$ are two ordered basis for the input and output space, respectively. Then the result in Theorem~(\ref{The:3:3}) implies
\[
\mathcal{C}_{\mathcal{A}',\mathcal{A}}(\bm v)_{\mathcal{A}}=(\bm v)_{\mathcal{A}'}
\]
i.e., the matrix representation theorem~(\ref{The:3:3}) is a general case for the change of basis theorem~(\ref{The:3:1})
\end{remark}

\begin{proposition}[Functoriality]\label{pro:3:6}
Suppose $V,W,U$ are finite dimensional vector spaces,
and let $\mathcal{A},\mathcal{B},\mathcal{C}$ be the ordered basis for $V,W,U$, respectively. 
Suppose that
\[
\begin{array}{ll}
T:V\to W,
&
S:W\to U
\end{array}
\]
are given two linear transformations, then
\[
(S\circ T)_{\mathcal{C},\mathcal{A}}
=
(S)_{\mathcal{C},\mathcal{B}}(T)_{\mathcal{B},\mathcal{A}}
\]

Composition of linear transformation corresponds to the multiplication of change of basis matrices.
\end{proposition}
\begin{proof}
Suppose the ordered basis $\mathcal{A}=\{\bm v_1,\dots,\bm v_n\}$, $\mathcal{B}=\{\bm w_1,\dots,\bm w_m\}$, $\mathcal{C}=\{\bm u_1,\dots,\bm u_p\}$. By defintion of change of basis matrices, 
\begin{align*}
T(\bm v_j)&=\sum_i(T_{\mathcal{B},\mathcal{A}})_{ij}\bm w_i\\
S(\bm w_i)&=\sum_k(S_{\mathcal{C},\mathcal{B}})_{ki}\bm u_k
\end{align*}

We start from the $j$-th column of $(S\circ T)_{\mathcal{C},\mathcal{A}}$ for $j=1,\dots,n$, namely
\begin{subequations}
\begin{align}
(S\circ T)_{\mathcal{C},\mathcal{A}}
(\bm v_j)_{\mathcal{A}}
&=
(S\circ T(\bm v_j))_{\mathcal{C}}\label{Eq:3:9:a}\\
&=
\left[S\circ
\left(
\sum_i(T_{\mathcal{B},\mathcal{A}})_{ij}\bm w_i\right)
\right]_{\mathcal{C}}\label{Eq:3:9:b}\\
&=
\sum_i(T_{\mathcal{B},\mathcal{A}})_{ij}
\left(
S(\bm w_i)
\right)_{\mathcal{C}}\label{Eq:3:9:c}\\
&=
\sum_i(T_{\mathcal{B},\mathcal{A}})_{ij}
\left(
\sum_k(S_{\mathcal{C},\mathcal{B}})_{ki}\bm u_k
\right)_{\mathcal{C}}\label{Eq:3:9:d}\\
&=\sum_k\sum_i(S_{\mathcal{C},\mathcal{B}})_{ki}(T_{\mathcal{B},\mathcal{A}})_{ij}(\bm u_k)_{\mathcal{C}}\label{Eq:3:9:e}\\
&=\sum_k(S_{\mathcal{C},\mathcal{B}}T_{\mathcal{B},\mathcal{A}})_{kj}(\bm u_k)_{\mathcal{C}}\label{Eq:3:9:f}
\\
&=
\sum_k(S_{\mathcal{C},\mathcal{B}}T_{\mathcal{B},\mathcal{A}})_{kj}\bm e_k\label{Eq:3:9:g}\\
&=\text{$j$-th column of }[S_{\mathcal{C}\mathcal{B}}T_{\mathcal{B},\mathcal{A}}]\label{Eq:3:9:h}
\end{align}
\end{subequations}
where (\ref{Eq:3:9:a}) is by the result in theorem~(\ref{The:3:3}); 
(\ref{Eq:3:9:b}) and (\ref{Eq:3:9:d}) follows from definitions of $T(\bm v_j)$ and $S(\bm w_i)$;
(\ref{Eq:3:9:c}) and (\ref{Eq:3:9:e}) follows from the linearity of $\mathcal{C}$;
(\ref{Eq:3:9:f}) follows from the matrix multiplication definition; 
(\ref{Eq:3:9:g}) is because $(\bm u_k)_{\mathcal{C}}=\bm e_k$.

Therefore, $(S\circ T)_{\mathcal{C}\mathcal{A}}$ and $(S_{\mathcal{C},\mathcal{B}})(T_{\mathcal{B},\mathcal{A}})$ share the same $j$-th column, and thus equal to each other.
\end{proof}

\begin{corollary}

Suppose that $S$ and $T$ are two identity mappings $V\to V$, and consider $(S)_{\mathcal{A}'\mathcal{A}}$ and $(T)_{\mathcal{A},\mathcal{A}'}$ in proposition~(\ref{pro:3:6}), then
\[
(S\circ T)_{\mathcal{A}',\mathcal{A}'}
=(S)_{\mathcal{A}'\mathcal{A}}
(T)_{\mathcal{A},\mathcal{A}'}
\]
Therefore,
\[
\text{Identity matrix}
=
\mathcal{C}_{\mathcal{A}',\mathcal{A}}
\mathcal{C}_{\mathcal{A},\mathcal{A}'}
\]
\end{corollary}

\begin{proposition}
Let $T:V\to W$ with $\dim(V)=n,\dim(W)=m$, and let
\begin{itemize}
\item
$\mathcal{A},\mathcal{A}'$ be ordered basis of $V$
\item
$\mathcal{B},\mathcal{B}'$ be ordered basis of $W$
\end{itemize}
then the change of basis matrices admit the relation
\begin{equation}\label{Eq:3:10}
(T)_{\mathcal{B}',\mathcal{A}'}
=
\mathcal{C}_{\mathcal{B}',\mathcal{B}}
(T)_{\mathcal{B},\mathcal{A}}
\mathcal{C}_{\mathcal{A}\mathcal{A}'}
\end{equation}
Here note that $(T)_{\mathcal{B}',\mathcal{A}'},(T)_{\mathcal{B},\mathcal{A}}\in\mathbb{F}^{m\times n}$; $\mathcal{C}_{\mathcal{B}',\mathcal{B}}\in\mathbb{F}^{m\times m}$; and $\mathcal{C}_{\mathcal{A}\mathcal{A}'}\in\mathbb{F}^{n\times n}$.
\end{proposition}

\begin{proof}
Let $\mathcal{A}=\{\bm v_1,\dots,\bm v_n\},\mathcal{A}'=\{\bm v_1',\dots,\bm v_n'\}$. Consider simplifying the $j$-th column for the LHS and RHS of (\ref{Eq:3:10}) and showing they are equal:
\begin{align*}
\text{LHS}&=(T)_{\mathcal{B}',\mathcal{A}'}\bm e_j\\
&=(T)_{\mathcal{B}',\mathcal{A}'}(\bm v_j')_{\mathcal{A}'}\\
&=(T\bm v_j')_{\mathcal{B}'}
\end{align*}
\begin{align*}
\text{RHS}&=\mathcal{C}_{\mathcal{B}',\mathcal{B}}
(T)_{\mathcal{B},\mathcal{A}}
\mathcal{C}_{\mathcal{A}\mathcal{A}'}\bm e_j\\
&=\mathcal{C}_{\mathcal{B}',\mathcal{B}}
(T)_{\mathcal{B},\mathcal{A}}
\mathcal{C}_{\mathcal{A}\mathcal{A}'}(\bm v_j')_{\mathcal{A}'}\\
&=\mathcal{C}_{\mathcal{B}',\mathcal{B}}
(T)_{\mathcal{B},\mathcal{A}}
(\bm v_j')_{\mathcal{A}}\\
&=\mathcal{C}_{\mathcal{B}',\mathcal{B}}
(T\bm v_j')_{\mathcal{B}}\\
&=(T\bm v_j')_{\mathcal{B}'}
\end{align*}
\end{proof}

\begin{remark}
Let $T:V\to V$ be a linear operator with $\mathcal{A},\mathcal{A}'$ being two ordered basisof $V$, then
\[
(T)_{\mathcal{A}'\mathcal{A}'}
=
\mathcal{C}_{\mathcal{A}',\mathcal{A}}
(T)_{\mathcal{A}\mathcal{A}}\mathcal{C}_{\mathcal{A},\mathcal{A}'}
=
(\mathcal{C}_{\mathcal{A},\mathcal{A}'})^{-1}
(T)_{\mathcal{A}\mathcal{A}}\mathcal{C}_{\mathcal{A},\mathcal{A}'}
\]
Therefore, the change of basis matrices $(T)_{\mathcal{A}'\mathcal{A}'}$ and $(T)_{\mathcal{A}\mathcal{A}}$ are similar to each other, which means they share the same eigenvalues, determinant, trace.

Therefore, two similar matrices cooresponds to same linear transformation using different basis.
\end{remark}















