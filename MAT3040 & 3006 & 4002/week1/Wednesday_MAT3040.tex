
\section{Wednesday for MAT3040}\index{Wednesday_lecture}
\subsection{Review}
\begin{enumerate}
\item
Vector Space: e.g., $\mathbb{R},M_{n\times n}(\mathbb{R}),\mathcal{C}(\mathbb{R}^n),\mathbb{R}[x]$.
\item
Vector Subspace: $W\le V$, e.g., 
\begin{enumerate}
\item
$V=\mathbb{R}^2$, the set $W:=\mathbb{R}^2_+$ is not a vector subspace since $W$ is not closed under scalar multiplication;
\item
the set $W=\mathbb{R}^2_+\bigcup\mathbb{R}^2_-$ is not a vector subspace since it is not closed under addition.
\item
For $V=\mathbb{M}_{3\times 3}(\mathbb{R})$, the set of invertible $3\times 3$ matrices is not a vector subspace, since we cannot define zero vector inside.
\item
Exercise: How about the set of all singular matrices? Answer: it is not a vector subspace since the vector addition does not necessarily hold.
\end{enumerate} 

\end{enumerate}
\subsection{Spanning Set}
\begin{definition}[Span]
Let $V$ be a vector space over $\mathbb{F}$:
\begin{enumerate}
\item
A linear combination of a subset $S$ in $V$ is of the form
\[
\sum_{i=1}^n\alpha_i\bm s_i,\quad
\alpha_i\in\mathbb{F},
\bm s_i\in S
\]
Note that the summation should be finite.
\item
The \emph{span} of a subset $S\subseteq V$ is
\[
\Span(S)=\left\{\sum_{i=1}^n\alpha_i\bm s_i\middle|\alpha_i\in\mathbb{F},
\bm s_i\in S\right\}
\] 
\item
$S$ is a spanning set of $V$, or say $S$ spans $V$, if
\[
\Span(S)=V.
\]
\end{enumerate}
\end{definition}
\begin{example}
For $V=\mathbb{R}[x]$, define the set
\[
S=\{1,x^2,x^4,\dots,x^6\},
\]
then $2+x^4+\pi x^{106}\in\Span(S)$, while the series $1+x^2+x^4+\cdots\notin\Span(S)$.

It is clear that $\Span(S)\ne V$, but $S$ is the spanning set of $W=\{p\in V\mid p(x)=p(-x)\}$.
\end{example}

\begin{example}
For $V=M_{3\times 3}(\mathbb{R})$, let $W_1=\{\bm A\in V\mid \bm A\trans=\bm A\}$ and $W_2=\{\bm B\in V\mid\bm B\trans=-\bm B\}$ (the set of skew-symmetric matrices) be two vector subspaces. Define the set
\[
\bm S:= W_1\bigcup W_2
\]
Exercise: $\bm S$ spans $V$.
\end{example}
\begin{proposition}\label{Pro:1:7}
Let $S$ be a subset in a vector space $V$.
\begin{enumerate}
\item
$S\subseteq\Span(S)$
\item
$\Span(S)=\Span(\Span(S))$
\item
If $\bm w\in\Span\{\bm v_1,\dots,\bm v_n\}\setminus\Span\{\bm v_2,\dots,\bm v_n\}$, then 
\[
\bm v_1\in\Span\{\bm w,\bm v_2,\dots,\bm v_n\}\setminus
\Span\{\bm v_2,\dots,\bm v_n\}
\]
\end{enumerate}
\end{proposition}
\begin{proof}
\begin{enumerate}
\item
For each $\bm s\in S$, we have 
\[
\bm s=1\cdot\bm s\in\Span(S)
\]
\item
From (1), it's clear that $\Span(S)\subseteq\Span(\Span(S))$, and therefore suffices to show $\Span(\Span(S))\subseteq\Span(S)$:

Pick $\bm v=\sum_{i=1}^n\alpha_i\bm v_i\in \Span(\Span(S))$, where $\bm v_i\in\Span(S)$. Rewrite 
\[
\bm v_i=\sum_{j=1}^{n_i}\beta_{ij}\bm s_j,\quad \bm s_j\in S,
\]
which implies
\begin{align*}
\bm v&=\sum_{i=1}^n\alpha_i\sum_{j=1}^{n_i}\beta_{ij}\bm s_j\\
&=\sum_{i=1}^n\sum_{j=1}^{n_i}(\alpha_i\beta_{ij})\bm s_j,
\end{align*}
i.e., $\bm v$ is the finite combination of elements in $S$, whcih implies $\bm v\in\Span(S)$.
\item
By hypothesis, $\bm w=\alpha_1\bm v_1+\cdots+\alpha_n\bm v_n$ with $\alpha_1\ne0$, which implies
\[
\bm v_1=-\frac{\alpha_2}{\alpha_1}\bm v_2+\cdots+\left(
-\frac{1}{\alpha_1}\bm w
\right)
\]
which implies $\bm v_1\in\Span\{\bm w,\bm v_2,\dots,\bm v_n\}$. It suffices to show $\bm v_1\notin\Span\{\bm v_2,\dots,\bm v_n\}$.

Suppose on the contrary that $\bm v_1\in \Span\{\bm v_2,\dots,\bm v_n\}$. It's clear that $\Span\{\bm v_1,\dots,\bm v_n\}=\Span\{\bm v_2,\dots,\bm v_n\}$. (left as exercise). Therefore,
\[
\emptyset=\Span\{\bm v_1,\dots,\bm v_n\}\setminus\Span\{\bm v_2,\dots,\bm v_n\},
\]
which is a contradiction.
\end{enumerate}
\end{proof}


\subsection{Linear Independence and Basis}
\begin{definition}[Linear Independence]
Let $S$ be a (not necessarily finite) subset of $V$. Then $S$ is \emph{linearly independent} (l.i.) on $V$ if for any finite subset $\{\bm s_1,\dots,\bm s_k\}$ in $S$,
\[
\sum_{i=1}^k\alpha_i\bm s_i=0\Longleftrightarrow
\alpha_i=0,\forall i
\]
\end{definition}

\begin{example}
For $V=\mathcal{C}(\mathbb{R})$, 
\begin{enumerate}
\item
let $S_1=\{\sin x,\cos x\}$, which is l.i., since
\[
\alpha\sin x+\beta\cos x=\bm0 (\mbox{means zero function})
\]
Taking $x=0$ both sides leads to $\beta=0$; taking $x=\frac{\pi}{2}$ both sides leads to $\alpha=0$.
\item
let $S_2=\{\sin^2x,\cos^2x,1\}$, which is linearly dependent, since
 \[
 1\cdot\sin^2x+1\cdot\cos^2x+(-1)\cdot 1=0,\forall x
 \]
\item
Exercise: For $V=\mathbb{R}[x]$, let $S=\{1,x,x^2,x^3,\dots,\}$, which is l.i.:

Pick $x^{k_1},\dots,x^{k_n}\in S$ with $k_1<\cdots<k_n$. Consider that the euqation
\[
\alpha_1x^{k_1}+\cdots+\alpha_nx^{k_n}=\bm0
\]
holds for all $x$, and try to solve for $\alpha_1,\dots,\alpha_n$ (one way is differentation.)
\end{enumerate}
\end{example}
\begin{definition}[Basis]
A subset $S$ is a \emph{basis} of $V$ if 
\begin{enumerate}
\item[(a)]
$S$ spans $V$;
\item[(b)]
$S$ is l.i.
\end{enumerate}
\end{definition}
\begin{example}
\begin{enumerate}
\item
For $V=\mathbb{R}^n$, $S=\{\bm e_1,\dots,\bm e_n\}$ is a basis of $V$
\item
For $V=\mathbb{R}[x]$, $S=\{1,x,x^2,\dots\}$ is a basis of $V$
\item
For $V=M_{2\times 2}(\mathbb{R})$, 
\[
S=\left\{
\begin{pmatrix}
1&0\\0&0
\end{pmatrix},
\begin{pmatrix}
0&1\\0&0
\end{pmatrix},
\begin{pmatrix}
0&0\\1&0
\end{pmatrix},
\begin{pmatrix}
0&0\\0&1
\end{pmatrix}
\right\}
\]
is a basis of $V$
\end{enumerate}
\end{example}
\begin{remark}
Note that there can be many basis for a vector space $V$.
\end{remark}
\begin{proposition}
Let $V=\Span\{\bm v_1,\dots,\bm v_m\}$, then there exists a subset of $\{\bm v_1,\dots,\bm v_m\}$, which is a basis of $V$.
\end{proposition}
\begin{proof}
If $\{\bm v_1,\dots,\bm v_m\}$ is l.i., the proof is complete.

Suppose not, then $\alpha_1\bm v_1+\cdots+\alpha_m\bm v_m=\bm0$ has a non-trivial solution. 
w.l.o.g., $\alpha_1\ne0$, which implies
\[
\bm v_1=-\frac{\alpha_2}{\alpha_1}\bm v_2+\cdots+\left(
\frac{\alpha_m}{\alpha_1}
\right)\bm v_m\implies
\bm v_1\in\Span\{\bm v_2,\dots,\bm v_m\}
\]

By the proof in (c), Proposition~(\ref{Pro:1:7}),
\[
\Span\{\bm v_1,\dots,\bm v_m\}=\Span\{\bm v_2,\dots,\bm v_m\},
\]
which implies $V=\Span\{\bm v_2,\dots,\bm v_m\}$.

Continuse this argument finitely many times to guarantee that $\{\bm v_i,\bm v_{i+1},\dots,\bm v_m\}$ is l.i., and spans $V$. The proof is complete.
\end{proof}

\begin{corollary}
If $V=\Span\{\bm v_1,\dots,\bm v_m\}$ (i.e., $V$ is finitely generated), then $V$ has a basis. (The same holds for non-finitely generated $V$).
\end{corollary}
\begin{proposition}
If $\{\bm v_1,\dots,\bm v_n\}$ is a basis of $V$, then every $\bm v\in V$ can be expressed uniquely as
\[
\bm v=\alpha_1\bm v_1+\cdots+\alpha_n\bm v_n
\]
\end{proposition}
\begin{proof}
Since $\{\bm v_1,\dots,\bm v_n\}$ spans $V$, so $\bm v\in V$ can be written as
\begin{equation}\label{Eq:1:1}
\bm v=\alpha_1\bm v_1+\cdots+\alpha_n\bm v_n
\end{equation}
Suppose further that 
\begin{equation}\label{Eq:1:2}
\bm v=\beta_1\bm v_1+\cdots+\beta_n\bm v_n,
\end{equation}
it suffices to show that $\alpha_i=\beta_i$ for $\forall i$:

Subtracting (\ref{Eq:1:1}) into (\ref{Eq:1:2}) leads to
\[
(\alpha_1-\beta_1)\bm v_1+\cdots+(\alpha_n-\beta_n)\bm v_n=0.
\]
By the hypothesis of linear independence, we have $\alpha_i-\beta_i=0$ for $\forall i$, i.e., $\alpha_i=\beta_i$.
\end{proof}











