\section{Wednesday for MAT3006}\index{Monday_lecture}
\subsection{Compactness}
This lecture will talk about the generalization of closeness and boundedness property in $\mathbb{R}^n$. First let's review some simple definitions:

\begin{definition}[Compact]
Let $(X,d)$ be a metric space, and $\{U_\alpha\}_{\alpha\in\mathcal{A}}$ a collection of open sets.
\begin{enumerate}
\item
$\{U_\alpha\}_{\alpha\in\mathcal{A}}$ is called an \emph{open cover} of $E\subseteq X$ if $E\subseteq\cup_{\alpha\in\mathcal{A}}U_\alpha$
\item
A \emph{finite subcover} of $\{U_\alpha\}_{\alpha\in\mathcal{A}}$ is a finite sub-collection 
$\{U_{\alpha_1},\dots,U_{\alpha_n}\}\subseteq\{U_\alpha\}$ covering $E$.
\item
The set $E\subseteq X$ is \emph{compact} if every open cover of $E$ has a finite subcover.
\end{enumerate}
\end{definition}

A well-known result is talked in MAT2006:

\begin{theorem}[Heine-Borel Theorem]
The set $E\subseteq\mathbb{R}^n$ is \emph{compact} if and only if $E$ is closed and bounded.
\end{theorem}

However, there's a notion of sequentially compact, and we haven't identify its gap and relation with compactness.

\begin{definition}[Sequentially Compact]
Let $(X,d)$ be a metric space. Then $E\subseteq X$ is \emph{sequentially compact} if every sequence in $E$ has a convergent subsequence with limit in $E$.
\end{definition}

A well-known result is talked in MAT2006:
\begin{theorem}[Bolzano-Weierstrass Theorem]
The set $E\subseteq\mathbb{R}^n$ is closed and bounded if and only if $E$ is sequentially compact.
\end{theorem}

Actually, the definitions of comapctness and the sequential compactness are equivalent under a metric space.

\begin{theorem}
Let $(X,d)$ be a metric space, then $E\subseteq X$ is compact if and only if $E$ is sequentially compact.
\end{theorem}

\begin{proof}

\textit{Necessity}

Suppose $\{x_n\}$ is a sequence in $E$, it suffices to show it has a convergent subsequence. 
Consider the tail of $\{x_n\}$, say 
\[
F_n=\overline{\{x_k\mid k\ge n\}}\implies F_1\supseteq F_2\supseteq\cdots.
\]
\begin{itemize}
\item
Note that $\cap_{i=1}^\infty F_i\ne\emptyset$. 
Assume not, then we imply $\cup_{i=1}^\infty(E\setminus F_i)=E$, i.e., $\{E\setminus F_i\}_{i=1}^\infty$ a open cover of $E$. 
By the compactness of $E$, we imply there exists a finite subcover of $E$:
\[
E=\bigcup_{j=1}^r(E\setminus F_{i_j})\implies \bigcap_{j=1}^rF_{i_j}=\emptyset\implies F_{i_r}=\emptyset,
\]
which is a contradiction, and there must exist an element $x\in\cap_{n=1}^\infty F_i$.
\item
For any $n\ge1$ and $x\in\cap_{n=1}^\infty F_i$, either $x\in\{x_k\mid k\ge n\}$ or $x\in\partial\{x_k\mid k\ge n\}$. In both cases, the open ball $B_{\varepsilon}(x)$ must intersect with the $n$-th tail of the sequence $\{x_n\}$ for any $\varepsilon>0$:
\[
B_{\varepsilon}(x)\cap\{x_k\mid k\ge n\}\ne\emptyset,\ \forall \varepsilon>0.
\]
Therefore, construct $x_{n_1}\in B_1(x)\cap\{x_k\mid k\ge1\}$ and for $r>1$, 
\[
x_{n_r}\in B_{1/r}(x)\cap\{x_{k}\mid k\ge n_{r-1}+1\}.
\]
Therefore, the subsequence $x_{n_r}\to x$ as $r\to\infty$. The proof for necessity is complete.
\end{itemize}

\textit{Sufficiency}

Firstly, let's assume the claim below hold (which will be shown later):
\begin{proposition}\label{pro:2:18}
If $E\subseteq X$ is sequentially compact, then for any$\varepsilon>0$, there exists finitely many open balls, say $\{B_{\varepsilon}(x_1),\dots,B_{\varepsilon}(x_n)\}$, covering $E$.
\end{proposition}

Suppose on the contrary that there exists an open cover $\{U_\alpha\}_{\alpha\in\mathcal{A}}$ of $E$, that has no finite subcover.
\begin{itemize}
\item
By proposition~(\ref{pro:2:18}), for $n\ge1$, there are finitely many balls of radius $1/n$ covering $E$. Due to our assumption, there exists a open ball $B_{1/n}(y_n)$ such that $B_{1/n}(y_n)\cap E$ cannot be covered by finitely many members in $\{U_\alpha\}_{\alpha\in\mathcal{A}}$.
\item
Pick $x_n\in B_{1/n}(y_n)$ to form a sequence. Due to the sequential compactness of $E$, there exists a subsequence $\{x_{n_j}\}\to x$ for some $x\in E$. 
\item
Since $\{U_\alpha\}_{\alpha\in\mathcal{A}}$ covers $E$, there exists a $U_\beta$ containing $x$.
Since $U_\beta$ is open and the radius of $B_{1/n_j}(y_{n_j})$ tends to 0, we imply that,
for sufficiently large $n_j$, the set $B_{1/n_j}(y_{n_j})\cap E$ is contained in $U_\beta$.

In oteher words, $U_\beta$ forms a \emph{single} subcover of $B_{1/n}(y)\cap E$, which contradicts to our choice of $B_{1/n_j}(y_{n_j})\cap E$. 
The proof for sufficiency is complete.
\end{itemize}
\end{proof}
\begin{proof}[Proof for proposition~(\ref{pro:2:18})]
Pick $B_\varepsilon(x_1)$ for some $x_1\in E$. 
Suppose $E\setminus B_\varepsilon(x_1)\ne\emptyset$.
We can find $x_2\notin B_\varepsilon(x_1)$ such that $d(x_2,x_1)\ge\varepsilon$.

Suppose $E\setminus(B_\varepsilon(x_1)\bigcup B_\varepsilon(x_2))$ is non-empty, 
then we can find $x_3\notin B_\varepsilon(x_1)\bigcup B_\varepsilon(x_2)$ 
so that $d(x_j,x_3)\ge\varepsilon$, $j=1,2$.

Keeping this procedure, we obtain a sequence $\{x_n\}$ in $E$ such that
\[
E\setminus\bigcup_{j=1}^nB_\varepsilon(x_j)\ne\emptyset,
\qquad
\text{and}
\qquad
d(x_j,x_n)\ge\varepsilon,j=1,2,\dots,n-1.
\]

By the sequential compactness of $E$, 
there exists $\{x_{n_j}\}$ and $x\in E$ so that $x_{n_j}\to x$ as $j\to\infty$.
But then $d(x_{n_j},x_{n_k})<d(x_{n_j},x)+d(x_{n_k},x)\to0$, 
which contradicts that $d(x_j,x_n)\ge\varepsilon$ for $\forall j<n$.

Therefore, one must have 
$E\setminus \bigcup_{j=1}^NB_\varepsilon(x_j)=\emptyset$ for some finite $N$. 

The proof is complete.
\end{proof}

\begin{remark}
\begin{enumerate}
\item
Given the condition metric space,
\[
\text{Sequential Compactness}
\Longleftrightarrow
\text{Compactness}
\]
\item
Given the condition metric space, we will show that 
\[
\text{Compactness}
\implies
\text{Closed and Bounded}
\]
However, the converse may not necessarily hold. Given the condition the metric space is $\mathbb{R}^n$, then
\[
\text{Compactness}
\Longleftrightarrow
\text{Closed and Bounded}
\]
\end{enumerate}


\end{remark}

\begin{proposition}
Let $(X,d)$ be a metric space. Then $E\subseteq X$ is compact implies that $E$ is closed and bounded.

We say a set $E$ if bounded if there exists $K\geq 0$ such that
\[
d(e_1,e_2) < K,\quad
\forall e_1, e_2 \in E
\]
\end{proposition}

\begin{proof}
\begin{enumerate}
\item
Let $\{x_n\}$ be a convergent sequence in $E$. By sequential compactness, $\{x_{n_j}\}\to x$ for some $x\in E$. By the uniqueness of limits, under metric space, $\{x_n\}\to x$ for $x\in E$. The closeness is shown
\item
Take $x\in E$ and consider the open cover $\bigcup_{n=1}^\infty B_n(x)$ of $E$. By compactness, 
\[
E\subseteq\bigcup_{i=1}^kB_{n_i}(x)=B_{n_k}(x),
\]
which implies that for any $y,z\in E$,
\[
d(y,z)\le d(y,x)+d(x,z)\le n_k+n_k=2n_k.
\]
The boundness is shown.
\end{enumerate}
\end{proof}

Here we raise several examples to show that the coverse does not necessarily hold under a metric space.

\begin{example}
Given the metric space $\mathcal{C}[0,1]$ and a set $E=\{f\in\mathcal{C}[0,1]\mid0\le f(x)\le 1\}$.

Notice that $E$ is closed and bounded:
\begin{itemize}
\item
$E=\cap_{x\in[0,1]}\Psi_x^{-1}([0,1])$, where $\Psi_x(f)=f(x)$, which implies that $E$ is closed.
\item
Note that $E\subseteq B_2(\bm0)=\{f\mid |f|<2\}$, i.e., $E$ is bounded.
\end{itemize}

However, $E$ may not be compact. Consider a sequence $\{f_n\}$ with
\[
f_n(x)=\left\{
\begin{aligned}
nx,&\quad0\le x\le\frac{1}{n}\\
1,&\quad\frac{1}{n}\le x\le 1
\end{aligned}
\right.
\]

Suppose on the contrary that $E$ is sequentially compact, therefore there exists a subsequence $\{f_{n_k}\}\to f$ under $d_\infty$ metric, which implies, $\{f_{n_k}\}$ uniformly converges to $f$.

By the definition of $f_n(x)$, we imply
\[
f(x)=\left\{
\begin{aligned}
0,&\quad x=0\\
1,&\quad x\in(0,1]
\end{aligned}
\right.
\]

However, since $d_\infty$ indicates uniform convergence, the limit for $\{f_{n_k}\}$, say $f$, must be continuous, which is a contradiction.

\end{example}



\begin{theorem}\label{The:2:7}
Let the set $E$ be compact in $(X,d)$ and the function $f:(X,d)\to (Y,\rho)$ is continuous. Then $f(E)$ is compact in $Y$.
\end{theorem}

Note that the technique to show compactness by using the sequential compactness is very useful. However, this technique only applies to the metric space, but fail in general topological spaces.
\begin{proof}
Let $\{y_n\}=\{f(x_n)\}$ be any sequence in $f(E)$.
\begin{itemize}
\item
By the compactness of $X$, $\{x_n\}$ has a convergent subsequence $\{x_{n_r}\}\to x$ as $r\to\infty$.
\item
Therefore, $\{y_{n_r}\}:=\{f(x_{n_r})\}\to f(x)$ by the continuity of $f$.
\item
Therefore, $f(E)$ is sequentially compact, i.e., compact.
\end{itemize}
\end{proof}

\begin{remark}
The Theorem~(\ref{The:2:7}) is a generalization of the statement that \textit{a continuous function on $\mathbb{R}^n$ admits its minimum and maximum}. Note that such an extreme value property no longer holds for arbitrary closed, bounded sets in a general metric space, but it continues to hold when the sets are strengthened to compact ones.

Another characterization of compactness in $\mathcal{C}[a,b]$ is shown in the Ascoli-Arzela Theorem (see Theorem~(14.1) in MAT2006 Notebook).
\end{remark}

\subsection{Completeness}
\begin{definition}[Complete]
Let $(X,d)$ be metric space.
\begin{enumerate}
\item
A sequence $\{x_n\}$ in $(X,d)$ is a \emph{Cauchy sequence} if for every $\varepsilon>0$, there exists some $N$ such that $d(x_n,x_m)<\varepsilon$ for all $n,m\ge N$.
\item
A subset $E\subseteq X$ is said to be \emph{complete} if every Cauchy sequence in $E$ is convergent.
\end{enumerate}
\end{definition}

\begin{example}\label{exp:2:15}
The set $X=\mathcal{C}[a,b]$ is complete:
\begin{itemize}
\item
Suppose $\{f_n\}$ is Cauchy in $\mathcal{C}[a,b]$, i.e., $\{f_n(x)\}$ is Cauchy in $\mathbb{R}$ for $\forall x\in[a,b]$.
\item
By the completeness of $\mathbb{R}$, the sequence $f_n(x)\to f(x)$ for some $f(x)\in\mathbb{R},\forall x\in[a,b]$. It suffices to show $f_n\to f$ uniformly:
\begin{itemize}
\item
For fixed $\varepsilon>0$, there exists $N>0$ such that
\[
d_\infty(f_n,f_{n+k})<\frac{\varepsilon}{2},\qquad
\forall n\ge N,k\in\mathbb{N}
\]
which implies that for $\forall x\in[a,b]$, $\forall n\ge N,k\in\mathbb{N}$,
\[
|f_n(x)-f_{n+k}(x)|<\frac{\varepsilon}{2}
\implies
\lim_{k\to\infty}|f_n(x)-f_{n+k}(x)|\le\frac{\varepsilon}{2}
\]
Therefore, we imply
\[
|f_n(x)-f(x)|=\lim_{k\to\infty}|f_n(x)-f_{n+k}(x)|\le\frac{\varepsilon}{2}<\varepsilon,\qquad
\forall n\ge N,x\in[a,b]
\]
The proof is complete.
\end{itemize}
\end{itemize}
\end{example}


















