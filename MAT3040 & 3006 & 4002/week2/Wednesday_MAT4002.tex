\section{Wednesday for MAT4002}\index{Monday_lecture}

\paragraph{Reviewing}
\begin{enumerate}
\item
Interior, Closure:
\[
\overline{A}=\{x\mid\forall U\ni x\text{ open}, U\bigcap A\ne\emptyset\}
\]
\item
Accumulation points
\end{enumerate}

\subsection{Remark on Closure}

\begin{definition}[Sequential Closure]
Let $A_S$ be the set of limits of any convergent sequence in $A$, then $A_S$ is called the \emph{sequential closure} of $A$.
\end{definition}
\begin{definition}[Accumulation/Cluster Points]
The set of accumulation (limit) points is defined as
\[
A'=
\{x\mid\forall U\ni x\text{ open },(U\setminus\{x\})\bigcap A\ne\emptyset\}
\]
\end{definition}
\begin{remark}
\begin{enumerate}
\item
\begin{enumerate}
\item
There exists some point in $A$ but not in $A'$:
\[
A=\{1,2,3,\dots,n,\dots\}
\]
Then any point in $A$ is not in $A'$
\item
There also exists some point in $A'$ but not in $A$:
\[
A=\{\frac{1}{n}\mid n\ge1\}
\]
Then the point $0$ is in $A'$ but not in $A$.
\end{enumerate}
\item
The closure $\overline{A}=A\bigcup A'$.
\item
The size of the sequentical closure $A_S$ is between $A$ and $\overline{A}$, i.e., $A\subseteq A_S\subseteq\overline{A}$:

It's clear that $A\subseteq A_S$, since the sequence $\{a_n:=a\}$ is convergent to $a$ for $\forall a\in A$.

For all $a\in A_S$, we have $\{a_n\}\to a$. 
Then for any open $U\ni a$, 
there exists $N$ such that 
$\{a_N,a_{N+1},\dots\}\subseteq U\bigcap A\ne\emptyset$. 
Therefore, $a\in\overline{A}$, i.e., $A_S\subseteq\overline{A}$.
\end{enumerate}
\end{remark}

Question: Is $A_S=\overline{A}$?

\begin{proposition}
Let $(X,d)$ be a metric space, then $A_S=\bar{A}$.
\end{proposition}

\begin{proof}
Let $a\in\overline{A}$, 
then there exists $a_n\in B_{1/n}(a)\bigcap A$, 
which implies $\{a_n\}\to a$, i.e., 
$a\in A_S$.
\end{proof}

\begin{remark}
If $(X,\mathcal{T})$ is metrizable, then $A_S=\overline A$. 
The same goes for first countable topological spaces.
However, $A_S$ is a proper subset of $\overline{A}$ in general:

Let $A\subseteq X$ be the set of continuous functions, where $X=\mathbb{R}^{\mathbb{R}}$ denotes the set of all real-valued functions on $\mathbb{R}$, with the topology of pointwise convergence.

Then $A_S=B_1$, the set of all functions of first Baire-Category on $\mathbb{R}$; and $[A_S]_S=B_2$, the set of all functions of second Baire-Category on $\mathbb{R}$. Since $B_1\ne B_2$, we have $[A_S]_S=A_S$. Note that $\overline{\overline{A}}=\overline{A}$. We conclude that $A_S$ cannot equal to $\overline{A}$, since the sequential closure operator cannot be idemotenet.
\end{remark}

\begin{definition}[Boundary]
The \emph{boundary} of $\bm A$ is defined as
\[
\partial\bm A=\overline{A}\setminus A^\circ
\]
\end{definition}

\begin{proposition}
Let $(X,\mathcal{T})$ be a topological space with $A,B\subseteq X$.
\[
\begin{array}{lll}
\overline{X\setminus A}=X\setminus A^\circ,
&
(X\setminus B)^\circ = X\setminus\overline{B}&
\partial A=\overline{A}\bigcap(\overline{X\setminus A})
\end{array}
\]
\end{proposition}

%\begin{subequations}
%\begin{align}
%X\setminus A^\circ&=X\setminus\left(\bigcup_{U\text{ is open, } U\subseteq A}U\right)\\
%&=\bigcap_{U\text{ is oepn, }(X\setminus U)\\
%&=\bigcap_{V\text{ is closed, }F\supseteq X\setminus A}F\\
%\end{align}
%\end{subequations}
\begin{proof}
\begin{subequations}
\begin{align}
X\setminus A^\circ&=X\setminus\left(\bigcup_{U\text{ is open, } U\subseteq A}U\right)\\
&=\bigcap_{U\text{ is open, }U\subseteq A}(X\setminus U)\\
&=\bigcap_{V\text{ is closed, }F\supseteq X\setminus A}F\\
&=\overline{X\setminus A}
\end{align}
\end{subequations}
Denoting $X\setminus A$ by $B$, we obtain:
\begin{subequations}
\begin{align}
(X\setminus B)^\circ&=A^\circ\\
&=X\setminus(X\setminus A^\circ)\\
&=X\setminus\overline{X\setminus A}\\
&=X\setminus\overline{B}
\end{align}
\end{subequations}
By definition of $\partial A$,
\begin{subequations}
\begin{align}
\partial A&=\overline{A}\setminus A^\circ\\
&=\overline{A}\bigcap(X\setminus A^\circ)\\
&=\overline{A}\bigcap(\overline{X\setminus A})
\end{align}
\end{subequations}
\end{proof}

\subsection{Functions on Topological Space}
\begin{definition}[Continuous]
Let $f:(X,\mathcal{T}_X)\to(Y,\mathcal{T}_Y)$ be a map. Then the function $f$ is continuous, if
\[
U\in\mathcal{T}_Y\implies
f^{-1}(U)\in\mathcal{T}_X
\]
\end{definition}
\begin{example}
\begin{enumerate}
\item
The identity map $\text{id}: (X,\mathcal{T})\to(X,\mathcal{T})$ defined as 
$x\mapsto x$ is continuous
\item
The identity map $\text{id}: (X,\mathcal{T}_{\text{discrete}})\to(X,\mathcal{T}_{\text{indiscrete}})$ defined as $x\mapsto x$ is continuous.

Since $\text{id}^{-1}(\emptyset)=\emptyset$ and $\text{id}^{-1}(X)=X$
\item
The identity map $\text{id}: (X,\mathcal{T}_{\text{indiscrete}})\to(X,\mathcal{T}_{\text{discrete}})$ defined as $x\mapsto x$ is not continuous.
\end{enumerate}
\end{example}

\begin{proposition}
If $f:X\to Y$, and $g:Y\to Z$ be continuous, then $g\circ f$ is continuous
\end{proposition}

\begin{proof}
For given $U\in\mathcal{T}_Z$, we imply
\[
g^{-1}(U)\in\mathcal{T}_Y
\implies
f^{-1}(g^{-1}(U))\in\mathcal{T}_X,
\]
i.e., $(g\circ f)^{-1}(U)\in\mathcal{T}_X$
\end{proof}

\begin{proposition}
Suppose $f:X\to Y$ is continuous between two topological spaces. Then $\{x_n\}\to x$ implies $\{f(x_n)\}\to f(x)$.
\end{proposition}
\begin{proof}
Take open $U\ni f(x)$, which implies $f^{-1}(U)\ni x$. Since $f^{-1}(U)$ is open, we imply there exists $N$ such that 
\[
\{x_n\mid n\ge N\}\subseteq f^{-1}(U),
\] 
i.e., $\{f(x_n)\mid n\ge N\}\subseteq U$
\end{proof}

We use the notion of Homeomorphism to describe the equivalence between two topological spaces.
\begin{definition}[Homeomorphism]
A \emph{homeomorphism} between spaces topological spaces $(X,\mathcal{T}_X)$ and $(Y,\mathcal{T}_Y)$ is a bijection 
\[
f:(X,\mathcal{T}_X)\to(Y,\mathcal{T}_Y),
\]
such that both $f$ and $f^{-1}$ are continuous.
\end{definition}

\subsection{Subspace Topology}
\begin{definition}\label{Def:2:16}
Let $A\subseteq X$ be a non-empty set. 
The \emph{subspace topology} of $A$ is defined as:
\begin{enumerate}
\item
$\mathcal{T}_A:=\{U\bigcap A\mid U\in\mathcal{T}_A\}$
\item
The \emph{coarsest} topology on $A$ such that the \emph{inclusion map}
\[
\begin{array}{ll}
i:(A,\mathcal{T}_A)\to(X,\mathcal{T}_X),
&
i(x)=x
\end{array}
\]
is continuous.

(We say the topology $\mathcal{T}_1$ is \emph{coarser} than $\mathcal{T}_2$, or $\mathcal{T}_2$ is \emph{finer} than $\mathcal{T}_1$, if $\mathcal{T}_1\subseteq\mathcal{T}_2$

e.g., $\mathcal{T}_{\text{discrete}}$ is the finest topology, and $\mathcal{T}_{\text{indiscrete}}$ is coarsest topology.)
\item
The (\emph{unique}) topology such that for any $(Y,\mathcal{T}_Y)$,
\[
f:(Y,\mathcal{T}_Y)\to(A,\mathcal{T}_A)
\]
is continuous iff $i\circ f:(Y,\mathcal{T}_Y)\to(X,\mathcal{T}_X)$ (where $i$ is the inclusion map) is continuous.
\end{enumerate}
\end{definition}
\begin{proposition}
The definition (1) and (2) in~(\ref{Def:2:16}) are equivalent.
\end{proposition}

\begin{proof}[Outline]
The proof is by applying 
\[
i^{-1}(S)=S\bigcap A,\quad
\forall S
\]
\end{proof}

\begin{example}
Let all English and numerical letters be subset of $\mathbb{R}^2$:
\[
\mbox{P},
\mbox{6}
\]
The homeomorphism can be constrcuted between these two English letters.
\end{example}

\begin{proposition}
The definition (2) and (3) in (\ref{Def:2:16}) are equivalent.
\end{proposition}
\begin{proof}
Necessity. 
\begin{itemize}
\item
For $\forall U\in\mathcal{T}_X$, consider that
\[
(i\circ f)^{-1}(U)=f^{-1}(i^{-1}(U))=f^{-1}(U\bigcap A)
\]
since $U\bigcap A\in\mathcal{T}_A$ and $f$ is continuous, we imply $(i\circ f)^{-1}(U)\in\mathcal{T}_Y$
\item
For $\forall U'\in\mathcal{T}_A$, we have $U'=U\bigcap A$ for some $U\in\mathcal{T}_X$. Therefore,
\[
f^{-1}(U')=f^{-1}(U\bigcap A)=f^{-1}(i^{-1}(U))=(i\circ f)^{-1}(U)\in\mathcal{T}_Y.
\]
\end{itemize}
The sufficiency is left as exercise.
\end{proof}


\begin{proposition}\label{Pro:2:22}
\begin{enumerate}
\item
The definition (1) in~(\ref{Def:2:16}) does define a topology of $A$
\item
Closed sets of $A$ under subspace topology are of the form $V\bigcap A$, where $V$ is closed in $X$
\end{enumerate}
\end{proposition}

\begin{proposition}
Suppose $(A,\mathcal{T}_A)\subseteq(X,\mathcal{T}_X)$ is a subspace topology, and $B\subseteq A\subseteq X$. Then 
\begin{enumerate}
\item
$\bar B^A=\bar B^X\bigcap A$.
\item
$B^{\circ A}\supseteq B^{\circ X}$
\end{enumerate}
\end{proposition}

\begin{proof}
By proposition~(\ref{Pro:2:22}), $\bar B^X\bigcap A$ is closed in $A$, and $\bar B^X\bigcap A\supset B$, which implies
\[
\bar B^A\subseteq \bar B^X\bigcap A
\] 

Note that $\bar B^A\supset B$ is closed in $A$, which implies $\bar B^A=V\bigcap A\subseteq V$, where $V$ is closed in $X$. Therefore,
\[
\bar B^X\subseteq V\implies
\bar B^X\bigcap A\subseteq V\bigcap A=\bar B^A
\]
Therefore, $\bar B^A=\bar B^X\subseteq V$

\end{proof}

Can we have $B^{\circ X}=B^{\circ A}$?

\subsection{Basis (Base) of a topology}

Roughly speaking, a basis of a topology is a family of ``generators'' of the topology.
\begin{definition}
Let $(X,\mathcal{T})$ be a topological space. A family of subsets $\mathcal{B}$ in $X$ is a \emph{basis} for $\mathcal{T}$ if
\begin{enumerate}
\item
$\mathcal{B}\subseteq\mathcal{T}$, i.e., everything in $\mathcal{B}$ is open
\item
Every $U\in\mathcal{T}$ can be written as union of elements in $\mathcal{B}$.
\end{enumerate}
\end{definition}

\begin{example}
\begin{enumerate}
\item
$\mathcal{B}=\mathcal{T}$ is a basis.
\item
For $X=\mathbb{R}^n$,
\[
\mathcal{B}=\{B_r(\bm x)\mid \bm x\in\mathbb{Q}^n,r\in\mathbb{Q}\bigcap(0,\infty)\}
\]
Exercise: every $(a,b)=\bigcup_{i\in I}(p_i,q_i)$ for $p_i,q_i\in\mathbb{Q}$.

Therefore, $\mathcal{B}$ is countable.
\end{enumerate}
\end{example}
\begin{proposition}
If $(X,\mathcal{T})$ has a countable basis, e.g., $\mathbb{R}^n$, then $(X,\mathcal{T})$ has a second-countable space.
\end{proposition}




















