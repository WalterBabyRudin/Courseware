\section{Monday for MAT4002}\index{Monday_lecture}
\paragraph{Reviewing}
\begin{enumerate}
\item
Homotopy: we denote the homotopic function pair as $f\simeq g$.
\item
If $Y\subseteq\mathbb{R}^n$ is convex, 
then the set of continuous functions $f:X\to Y$ form a single equivalence class, i.e., $\{\text{continuous functions $f:X\to Y$}\}/\sim$ has only one element
\end{enumerate}
\subsection{Remarks on Homotopy}
\begin{proposition}\label{pro:9:4}
Consider four continous mappings 
\[
\begin{array}{llll}
W\xrightarrow{f}X,
&
X\xrightarrow{g}Y,
&
X\xrightarrow{h}Y,
&
Y\xrightarrow{k} Z.
\end{array}
\]
If $g\simeq h$, then
\[
\begin{array}{ll}
g\circ f\simeq h\circ f,&
k\circ g\simeq k\circ h
\end{array}
\]
\end{proposition}
\begin{proof}
Suppose there exists the homotopy $H:g\simeq h$, 
then $k\circ H:X\times I\to Z$ 
gives the momotopy between $k\circ g$ and $k\circ h$.

Simiarly, 
$H\circ(f\times\text{id}_I):W\times I\to Y$ 
gives the homotopy $g\circ f\simeq h\circ f$.
\end{proof}

\begin{definition}[Homotopy Equivalence]
Two topological spaces $X$ and $Y$ are \emph{homotopy equivalent} if there are continuous maps $f:X\to Y$, and $g:Y\to X$ such that
\[
\begin{array}{l}
g\circ f\simeq \text{id}_{X\to X}\\
f\circ g\simeq\text{id}_{Y\to Y},
\end{array}
\]
which is denoted as $X\simeq Y$.
\end{definition}
\begin{remark}
\begin{enumerate}
\item
If $X\cong Y$ are homeomorphic, then they are homotopic equivalent.
\item
The homotopy equivalence $X\simeq Y$ gives a bijection between $\{\phi:\text{continuous }W\to X\}/\sim$ and $\{\phi:\text{continuous }W\to Y\}/\sim$, for any given topological space $W$.
\begin{proof}
Since $X\simeq Y$, we can find $f:X\to Y$ and $g:Y\to X$ such that $f\circ g\simeq\text{id}_Y$ and $g\circ f\simeq\text{id}_X$.
We construct a mapping
\[
\begin{array}{ll}
\phi:&\{\phi:\text{continuous }W\to X\}/\sim\to \{\phi:\text{continuous }W\to Y\}/\sim\\
\text{with}&[\phi]\mapsto[f\circ\phi]
\end{array}
\]
$\phi$ is well-defined since $\phi_1\sim\phi_2$ implies $f\circ\phi_1\sim f\circ\phi_2$

Also, we can construct a mapping
\[
\begin{array}{ll}
\beta:& \{\phi:\text{continuous }W\to Y\}/\sim\to\{\phi:\text{continuous }W\to X\}/\sim\\
\text{with}&[\psi]\mapsto[g\circ\phi]
\end{array}
\]
Similarly, $\beta$ is well-defined.

Also, we can check that $\alpha\circ\beta=\text{id}$ and $\beta\circ\alpha=\text{id}$.
For example,
\[
\alpha\circ\beta[\psi]=[f\circ g\circ\psi]=[\psi],
\]
where the last equality is because that $f\circ g\simeq\text{id}_Y$.
\end{proof}
\item
The homotopy equivalence $X\simeq Y$ forms an equivalence relation between topological spaces
\end{enumerate}
\end{remark}

Compared with homeomorphism, some properties are lost when consider the homotopy equivalence.

\begin{definition}[Contractible]
The topological space $X$ is \emph{contractible} 
if it is homotopy equivalent to any point $\{\bm c\}$.

\begin{remark}
In other words, there exists continuous mappings $f,g$ such that
\begin{align*}
\{\bm c\}\xrightarrow{f}X\xrightarrow{g}\{\bm c\},&\ 
g\circ f\simeq\text{id}_{\{\bm c\}}\\
X\xrightarrow{g}\{\bm c\}\xrightarrow{f}X,&\ 
f\circ g\simeq\text{id}_{X}
\end{align*}
Note that $g\circ f\simeq\text{id}_{\{\bm c\}}$ follows naturally; and since $X\cong X$, we can find $f,g$ such that $f\circ g=c_y$ for some $y\in X$, where $c_y:X\to X$ is a constant function $c_y(x)=y,\forall x\in X$.
Therefore, to check $X$ is contractible, it suffices to check $c_y\simeq\text{id}_{X},\forall y\in X$.

Therefore, $X$ is contractible if its identity map $\text{id}_X$ is homotopic to any constant map $c_y,\forall y\in X$.
\end{remark}
\end{definition}
\begin{proposition}
The definition for contractible can be simplified further:
\begin{enumerate}
\item
$X$ is contractible if it is homotopy equivalent to some point $\{c\}$
\item
$X$ is contractible if the identity map $\text{id}_X$ is homotopic to some constant map $c_y(x)=y$.
\end{enumerate}
\end{proposition}
\begin{proof}
The only thing is to show that $c_y\simeq c_{y'},\forall y,y'\in X$.
By hw 3, $X$ is path-connected, and therefore there exists continous $p(t)$ such that
\[
\begin{array}{ll}
p(0)=y,&
p(1)=y'
\end{array}
\]
Therefore, we construct the homotopy between $c_y$ and $c_{y'}$ as follows:
\[
H(x,t)=p(t).
\]
\end{proof}


\begin{example}
\begin{enumerate}
\item
$X=\mathbb{R}^2$ is contractible:

It suffices to show that the mapping $f(\bm x)=\bm x,\forall \bm x\in\mathbb{R}^2$ is homotopic to the constant function $g(x)=(0,0),\forall x\in\mathbb{R}^2$, i.e., $g=c_{(0,0)}$.

Consider the continuous mapping $H(\bm x,t) = tf(\bm x)$, with
\[
\begin{array}{ll}
H(\bm x,0)=c_{(0,0)},
&
H(\bm x,1)=\text{id}_X
\end{array}
\]
Therefore, $c_{(0,0)}\simeq \text{id}_X$.
Since $c_{(0,0)}\simeq c_{\bm y},\forall\bm y\in\mathbb{R}^2$, we imply $c_{\bm y}\simeq\text{id}_X$ for any $\bm y\in \mathbb{R}^2$.

Therefore, $X$ is contractible.

More generally, any convex $X\subseteq\mathbb{R}^n$ is contractible.
\end{enumerate}
\end{example}

\begin{remark}
$S^1$ is not contractible, and we will see it in 3 weeks' time. In particular, we are not able to construct the continuous mapping
\[
H:S^1\times[0,1]\to S^1
\]
such that 
\[
\begin{array}{ll}
H(e^{2\pi ix},0)=e^{2\pi i x},
&
H(e^{2\pi i x},1)=e^{2\pi i(0)} = 1
\end{array}
\]

How about the mapping $H(e^{2\pi ix},t)=e^{2\pi ixt}$?
Unfortunately, it is not well-defined, since
\[
H(e^{2\pi i(1)},t)=e^{2\pi it}=H(e^{2\pi i(0)},t)=1
\]
and the equality is not true for $t\ne0,1$.
\end{remark}


\begin{definition}[Homotopy Retract]
Let $A\subseteq X$ and $i:A\hookrightarrow X$ be an inclusion.
We say $A$ is a \emph{homotopy retract} of $X$ if there exists continuous mapping $r:X\to A$ such that 
\begin{align*}
r\circ i:&A\hookrightarrow X\xrightarrow{r}A=\text{id}_{A}\\
i\circ r:&X\xrightarrow{r}A\hookrightarrow X\simeq\text{id}_{X}
\end{align*}
In particualr, $A\simeq X$.
\end{definition}

\begin{example}
The 1-sphere $S^1$ is a homotopy retract of Mobius band $M$.

Let $M=[0,1]^2/\sim$ and $S^1=[0,1]/\sim$.
Define the inclusion $i$ and $r$ as:
\[
\begin{array}{ll}
i:&S^1\hookrightarrow M\\
\text{with}&[x]\mapsto [(x,\frac{1}{2})]
\end{array}
\]
\[
\begin{array}{ll}
r:&M\to S^1\\
\text{with}&[(x,y)]\mapsto[x]
\end{array}
\]
As a result, 
\[
r\circ i = \text{id}_{S^1},\qquad
i\circ r([(x,y)]) = [(x,1/2)]
\]

It suffices to show $i\circ r\simeq\text{id}_{M}$, where $\text{id}_M([(x,y)]) = [(x,y)]$.

Construct the continous mapping $H:M\times I\to M$ with
\[
H([(x,y)],t):=[(x,(1-t)y+t/2)]
\]
To show the well-definedness of $H$, we need to check 
\[
H([(0,y)],t)=H([(1,1-y)],t),\quad \forall y\in[0,1]
\]
It's clear that $H$ gives a homotopy between $i\circ r$ and $\text{id}_{M}$, i.e., $i\circ r\simeq\text{id}_M$
\end{example}
\begin{example}
The $n-1$-sphere $S^{n-1}$ is a homotopy retract of $\mathbb{R}^n\setminus\{\bm0\}$:

We have the inclusion $i: S^{n-1}\to\mathbb{R}^n\setminus\{0\}$ and
\[
\begin{array}{ll}
r:&\mathbb{R}^n\setminus\{0\}\to \mathbb{S}^{n-1}\\
\text{with}&x\mapsto\frac{x}{\|x\|}
\end{array}
\]
Therefore, $r\circ i=\text{id}_{S^{n-1}}$ and $i\circ r(x)= \frac{x}{\|x\|}$.

It suffices to show that $i\circ r\simeq\text{id}_{\mathbb{R}^n\setminus\{0\}}$.
Consider the homotopy $H(x,t)=t\bm x+(1-t)\bm x/\|\bm x\|$ such that 
\[
H(\bm x,0)=i\circ r(\bm x),\quad
H(\bm x,1)=\bm x=\text{id}(\bm x)
\]
To show the well-definedness of $H$, we need to check $H(x,t)\in\mathbb{R}^n\setminus\{\bm0\}$ for all $\bm x\in\mathbb{R}^n\setminus\{\bm0\}$ and $t\in[0,1]$.
\end{example}


\begin{definition}[Homotopic Relative]
Let $A\subseteq X$ be topological spaces.
We say $f,g:X\to Y$ are homotopic relative to $A$ if there eixsts $H:X\times I\to Y$ such that
\[
\left\{
\begin{aligned}
H(x,0)&=f(x)\\
H(x,1)&=g(x)
\end{aligned}
\right.
\qquad
\text{ and }
H(a,t)=f(a)=g(a),\forall a\in A
\]
\end{definition}

\begin{figure}[H]
\centering
\includegraphics[width=\textwidth]{week9/p_4}
\end{figure}















