
\section{Wednesday for MAT3040}\index{Wednesday_lecture}
\textbf{Reviewing: }Root Theorem: $p(\lambda)=0$ iff $(x-\lambda)$ divdes $p(x)$.
\begin{corollary}
A polynomial with degree $n$ has at most $n$ roots counting multiplicity.
\end{corollary}
For example, the polynomial $(x-3)^2$ has one root $x=3$ with multiplicity $2$. When counting multiplicity, we say the polynomial $(x-3)^2$ has two roots.

\begin{definition}[Algebraically Closed]
A field $\mathbb{F}$ is called \emph{algebraically closed} if every non-constant polynomial $p(x)\in\mathbb{F}[x]$ has a root $\lambda\in \mathbb{F}$.
\end{definition}
\begin{theorem}[Fundamental Theroem of Algebra]
The set of complex numbers $\mathbb{C}$ is algebraically closed.
\end{theorem}
\begin{proof}
One way is by complex analysis; Another way is by the topology on $\mathbb{C}\setminus\{0\}$.
\end{proof}
\begin{remark}
By induction, we can show that every polynomial with degree $n$ on algebraically closed field $\mathbb{F}$ has \emph{exactly} $n$ roots, counting multiplicity.
Therefore, for any $p(x)$ on algebraically closed field $\mathbb{F}$,
\begin{equation}\label{Eq:6:2}
p(x) = c(x-\lambda_1)\cdots(x-\lambda_n)
\end{equation}
for $c,\lambda_1,\dots,\lambda_n\in\mathbb{F}$.
\end{remark}
The polynomials on general field $\mathbb{F}$ may not necessarily be factorized as in~(\ref{Eq:6:2}) , but still admit unique factorization property:
\begin{theorem}[Unique Factorization]
Every $f(x) = a_nx^n+\cdots+a_0$ in $\mathbb{F}[x]$ can be factorized as
\[
f(x) = a_n[p_1(x)]^{e_1}\cdots[p_k(x)]^{e_k}
\]
where $p_i$'s are \emph{monic, irreducible,distinct}. Furthermore, this expression is unique up to the permutation of factors.
\end{theorem}

\begin{definition}[Factor]
If $p(x) = q(x)s(x)$ with $p,q,s\in\mathbb{F}[x]$, then we say
\begin{itemize}
\item
$p(x)$ is \emph{divisible} by $s(x)$;
\item
$s(x)$ is a \emph{factor} of $p(x)$;
\item
$s(x)|p(x)$
\item
$s(x)$ \emph{divides} $p(x)$
\item
$p(x)$ is \emph{multiple} of $s(x)$
\end{itemize}
\end{definition}

\begin{definition}[Common Factor]
\begin{enumerate}
\item
The polynomial $g(x)$ is said to be a \emph{common factor} of $f_1,\dots,f_k\in\mathbb{F}[x]$ if
\[
\begin{array}{ll}
g|f_i,
i=1,\dots,k
\end{array}
\]
\item
The polynomial $g(x)$ is said to be a \emph{greatest common divisor} of $f_1,\dots,f_k$ if
\begin{itemize}
\item
$g$ is \emph{monic}.
\item
$g$ is common factor of $f_1,\dots,f_k$
\item
$g$ is of largest possible (maximal) degree.
\end{itemize} 
\end{enumerate}
\end{definition}

\begin{remark}
\begin{itemize}
\item
$\text{gcd}(f_1,\dots,f_k) = \text{gcd}(\text{gcd}(f_1,f_2),f_3,\dots,f_k)$
$=\text{gcd}(\text{gcd}(f_1,f_2,f_3),\dots,f_k)$
\item
$\text{gcd}(f_1,\dots,f_k)$ is unique.
\item
If $\text{gcd}(f_1,\dots,f_k)=1$, we say $f_1,\dots,f_k$ is \emph{relatively prime}
\item
Polynomials $f_1,\dots,f_k$ are relatively prime does not necessarily mean $\text{gcd}(f_i,f_j)=1$ for any $i\ne j$.

Counter-example:
Let $a_1,\dots,a_n$ distinct irreducible polynomials, and
\[
f_i(x) = a_1(x)\cdots \hat{a}_i(x)\cdots a_n(x):=a_1\cdots a_{i-1}a_{i+1}\cdots a_n,
\]
then $\text{gcd}(f_1,\dots,f_n)=1$, but $\text{gcd}(f_i,f_j) = a_1\cdots\hat{a}_i\cdots\hat{a}_j\cdots a_n$, which does not necessarily equal to 1.
\end{itemize}
\end{remark}
\begin{example}
The $\text{gcd}(f_1,f_2)$ is easy to compute for factorized polynomials. For example, let $f_1(x) = (x^2+x+1)^3(x-3)^2x^4$ and
$f_2(x)=(x^2+1)(x-3)^4x^2$ in $\mathbb{R}[x]$, then
\[
\text{gcd}(f_1,f_2) = (x-3)^2x^2
\]
\end{example}
The question is how to find $\text{gcd}(f_1,f_2)$ for given un-factorized polynomials? 
\begin{theorem}[Bezout]\label{The:6:7}
Let $g = \text{gcd}(f_1,f_2)$, then there exists $r_1,r_2\in\mathbb{F}[x]$ such that
\[
g(x)  =r_1(x)f_1(x)+r_2(x)f_2(x)
\]
More generally, $g=\text{gcd}(f_1,\dots,f_k)$ implies there exists $r_1,\dots,r_k$ such that 
\[
g = r_1f_1+\cdots+r_kf_k
\]
\end{theorem}
The derivation of $r_i$'s is by applying \emph{Euclidean algorithm}. For example, given $x^3+6x+7$ and $x^2+3x+2$, we imply 
\[
x^3+6x+7 - (x-3)(x^2+3x+2) = 13x+13
\]
and
\[
x^2+3x+2 - \frac{x+2}{13}(13x+13)=0
\]
Therefore, $\text{gcd}(x^3+6x+7,x^2+3x+2) = \text{gcd}(x^2+3x+2,13x+13)=x+2$.

\subsection{Eigenvalues $\&$ Eigenvectors}
\begin{definition}[Eigenvalues]
Let $T:V\to V$ be a linear operator.
\begin{enumerate}
\item
We say $\bm v\in V\setminus\{\bm0\}$ is an eigenvector of $T$ with eigenvalue $\lambda$ if 
$T(\bm v) = \lambda\bm v$;
\item
Or equivalently, $\bm v\in\ker(T-\lambda I)$, the $\lambda$-eigenspace of $T$. Here the mapping $I:V\to V$ denotes identity map, i.e., $I(\bm v)=\bm v,\forall \bm v\in V$.
\end{enumerate}
\end{definition}
\begin{definition}
A vector $\bm v\in V\setminus\{\bm0\}$ is a \emph{generalized eigenvector} of $T$ with \emph{generalized eigenvalue} $\lambda$ 
if $\bm v\in\ker((T-\lambda I)^k)$ for some $k\in\mathbb{N}^+$.
\end{definition}
Note that an eigenvector is a generalized eigenvector of $T$; while the converse does not necessarily hold.
\begin{example}
Consider the linear transformation $A:\mathbb{R}^2\to\mathbb{R}^2$ with
\[
\begin{array}{ll}
A:&\mathbb{R}^2\to\mathbb{R}^2\\
\text{with}&\bm x\to \bm A\bm x\\
\text{where}&\bm A=\begin{pmatrix}
1&1\\0&1
\end{pmatrix}
\end{array}
\]
\begin{enumerate}
\item
Note that $[1,0]\trans$ is an eigenvector with eigenvalue $1$, since
\[
A\begin{pmatrix}
1\\0
\end{pmatrix}=\begin{pmatrix}
1\\0
\end{pmatrix}
\]
\item
However, $[0,1]\trans$ is not an eigenvector, since
\[
A\begin{pmatrix}
0\\1
\end{pmatrix}=\begin{pmatrix}
1\\0
\end{pmatrix}.
\]
Note that 
\[
(A-I)^2=\begin{pmatrix}
0&1\\0&0
\end{pmatrix},\quad
(A-I)^3=\begin{pmatrix}
0&0\\0&0
\end{pmatrix}
\]
and therefore
\[
\begin{pmatrix}
0\\1
\end{pmatrix}\in\ker(A-I)^2,
\]
i.e., a generalized eigenvector with generalized eigenvalue $1$.
\end{enumerate}
\end{example}
\begin{example}
Consider $V=\mathcal{C}^\infty(\mathbb{R})$, which is a set of all infinitely differentiable functions.

Define the linear operator $T:V\to V$ as $T(f) = f''$. 
Then the $(-1)$-eigenspace of $T$ has $f\in V$ satisfying
\[
f''=-f
\]
From ODE course, we imply $\{\sin x,\cos x\}$ forms a basis of $(-1)$-eigenspace.
\end{example}

\paragraph{Assumption}
From now on, we assume $V$ has finite dimension by default.

\begin{definition}[Determinant]
Let $T:V\to V$ be a linear operator.
The \emph{determinant} of $T$ is given by
\[
\det(T)=\det((T)_{\mathcal{A},\mathcal{A}})
\]
where $\mathcal{A}$ is some basis of $V$.
\end{definition}
\begin{remark}
Assume we have complete knowledge about $\det(M)$ for matrices for now.
The determinant is well-defined, i.e., independent of the choice of basis $\mathcal{A}$. For another basis $\mathcal{B}$, we imply
\[
\det(T_{\mathcal{B},\mathcal{B}})
=
\det(C_{\mathcal{B},\mathcal{A}}T_{\mathcal{A},\mathcal{A}}
C_{\mathcal{A},\mathcal{B}}
)
=
\det(C_{\mathcal{B},\mathcal{A}})
\det(T_{\mathcal{A},\mathcal{A}})
\det(C_{\mathcal{A},\mathcal{B}})=\det(T_{\mathcal{A},\mathcal{A}})
\]
\end{remark}


\begin{definition}[characteristic polynomial]
The \emph{characteristic polynomial} $\mathcal{X}_T(x)$
of $T:V\to V$ is defined as
\[
\mathcal{X}_T(x) = \det((T)_{\mathcal{A},\mathcal{A}} - xI)
\]
for any basis $\mathcal{A}$
\end{definition}
In the next few lectures, we will study
\begin{itemize}
\item
Cayley-Hamilton Theorem
\item
Jordan Canonical Form
\end{itemize}

These theorems can be stated using matrices, and
they both hold up to change of basis.
We have a unified statement of these theorem using vecotor space rather than $\mathbb{R}^n$.









