\chapter{Week6}
\section{Monday for MAT3040}\index{Monday_lecture}

\subsection{Polynomials}
We recall some useful properties of polynomial before studying eigenvalues/eigenvectors.
\begin{definition}[Polynomial]
\begin{enumerate}
\item
A polynomial over $\mathbb{F}$ has the form
\[
p(z)=a_mz^m+\cdots+a_1z+a_0,\quad (a_m\ne0).
\]
Here $a_mz^m$ is called the \emph{leading term} of $p(z)$; $m$ is called the degree; $a_m$ is called the \emph{leading coefficient}; $a_m,\cdots,a_0$ are called the coefficients of this polynomial.
\item
A polynomial over $\mathbb{F}$ is monic if its leading coefficient is $1_{\mathbb{F}}.$
\item
A polynomial $p(z)\in\mathbb{F}[z]$ is \emph{irreducible} if for any $a(z),b(z)\in\mathbb{F}[z]$, 
\[
p(z)=a(z)b(z)\implies
\text{either $a(z)$ or $b(z)$ is a constant polynomial}.
\]
Otherwise $p(z)$ is \emph{reducible}.
\end{enumerate}
\end{definition}
\begin{example}
For example, the polynomial $p(x)=x^2+1$ is irreducible over $\mathbb{R}$; but $p(x) = (x-i)(x+i)$ is \emph{reducible} over $\mathbb{C}$.
\end{example}

\begin{theorem}[Division Theorem]
For all $p,q\in\mathbb{F}[z]$ such that $p\ne0$, there exists unique $s,r\in\mathbb{F}[x]$ satisfying $\text{deg}(r)<\text{deg}(f)$, such that 
\[
p(z)=s(z)\cdot q(z)+r(z).
\]
Here $r(z)$ is called the \emph{remainder}.
\end{theorem}


\begin{example}
Given $p(x)=x^4+1$ and $q(x)=x^2+1$, the junior school knowledge tells us that uniquely
\[
x^4+1 = (x^2-1)(x^2+1)+2.
\]
\end{example}

\begin{theorem}[Root Theorem]
For $p(x)\in\mathbb{F}[x]$, and $\lambda\in\mathbb{F}$, $x-\lambda$ divides $p$ if and only if $p(\lambda)=0$.
\end{theorem}
\begin{proof}
\begin{enumerate}
\item
If $(x-\lambda)$ divides $p$, then $p=(x-\lambda)q$ for some $q\in\mathbb{F}[x]$. Thus clearly $p(\lambda)=0$.
\item
For the other direction, suppose that $p(\lambda)=0$. By division theorem, there exists $s,r\in\mathbb{F}[x]$ such that 
\begin{equation}\label{Eq:6:1}
p=(x-\lambda)s+r\quad \text{with deg$(r)<$deg$(x-\lambda)=1$.}
\end{equation}
Therefore, the polynomial $r$ must be  constant. 

Substituting $\lambda$ into $x$ both sides in (\ref{Eq:6:1}), we have
\[
0 = p(\lambda)=0\cdot s+r\implies r=0.
\]
Therefore, $p = (x-\lambda)\cdot s$, i.e., $(x-\lambda)$ divides $p$.
\end{enumerate}
\end{proof}













