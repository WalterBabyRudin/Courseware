
\chapter{Week1}

\section{Monday}\index{Monday_lecture}

\subsection{Metric Space}

\begin{definition}[Metric Space]
A metric space is a pair $(X,\textsf{d})$,
where $X$ is a set and $\textsf{d}$ is a metric on $X$,
i.e., a function defined on $X\times X$ such that for any $x,y,z\in X$,
\begin{itemize}
\item
$\textsf{d}$ is real-valued, non-negative, and finite;
\item
$\textsf{d}(x,y)=0$ if and only if $x=y$;
\item
$\textsf{d}(x,y)=\textsf{d}(y,x)$;
\item
$\textsf{d}(x,y)\le \textsf{d}(x,z) + \textsf{d}(z,y)$.
\end{itemize}
\end{definition}

\begin{definition}[Subspace]
A subspace $(Y, \tilde{\textsf{d}})$ of $(X,d)$ is obtained if we take $Y\subseteq X$ and restrict $\textsf{d}$ into $Y\times Y$, denoted as
\[
\tilde{\textsf{d}}=\textsf{d}\mid_{Y\times Y}.
\]
The metric $\tilde{\textsf{d}}$ is called the metric induced on $Y$ by $\textsf{d}$.
\end{definition}

\begin{example}
Examples about metric space:
\begin{itemize}
\item
Real line $\mathbb{R}$:
a set of all real numbers with the usual metric $\textsf{d}(x,y)=|x-y|$;
\item
Euclidean space $\mathbb{R}^2$:
the metric space $\mathbb{R}^2$ is obtained by defining the metric
\[
d(x,y) = \sqrt{(x_1-y_1)^2 + (x_2-y_2)^2},\quad
x=(x_1,x_2), y=(y_1,y_2).
\]
\item
More generally, the Euclidean space $\mathbb{R}^n$, unitary space $\mathbb{C}^n$, and complex plane $\mathbb{C}$ are metric spaces.
\item
Sequence space $\ell^\infty$: As a set $X$, we take the set of all bounded sequences of complex numbers, i.e.,
\[
X\ni x=(x_1,x_2, \ldots):=(x_i)
\]
such that for all $i$, $|\xi_i|\le C_x$, where $C_x$ is a real number which may be dependent on $x$ but does not depend on $i$.
We choose the metric
\[
\textsf{d}(x,y)=\sup_{i\in\mathbb{N}}|x_i-y_i|,
\]
with $x=(x_i), y=(y_i).$
\item
Function space $\mathcal{C}[a,b]$:
The set of all real-valued functions $x,y,\ldots$,
which are functions of an independent variable $t$,
and are defined and continuous on $\mathcal{J}=[a,b]$.
We choose the metric
\[
\textsf{d}(x,y) = \max_{t\in\mathcal{J}}|x(t) - y(t)|.
\]
\item
Space $\mathcal{B}(A)$ of bounded functions:
each element $x\in\mathcal{B}(A)$ is a function defined and bounded on a set $A$.
The metric is defined by 
\[
\textsf{d}(x,y)=\sup_{A}|x(t)-y(t)|.
\]
\begin{proof}
It's eash to see that $\textsf{d}(x,y)$ is real-valued, finite, non-negative,and $\textsf{d}(x,y)=\textsf{d}(y,x)$.
As for the second condition, when $d(x,y)=0$,
\[
x(t)-y(t)=0,\quad \forall t\in A,
\]
so that $x=y$.
As for the last condition, for all $t\in A$,
\begin{align*}
|x(t) - y(t)|
&\le
|x(t) - z(t)| + |z(t)-y(t)|\\
&\le
\sup_{t\in A}|x(t) - z(t)| + \sup_{t\in A}|z(t)-y(t)|
\end{align*}
This indicates that $|x(t) - y(t)|\le \textsf{d}(x,z)+\textsf{d}(z,y)$ for all $t\in A$. Taking the supremum on $A$ both sides gives the desired result.
\end{proof}
\item
Space $\ell^p$: 
Let $p\ge 1$ be a fixed number.
By definition, each element in $\ell^p$ is a sequence $x=(x_i)=(x_1,x_2,\ldots)$ of numbers such that 
\[
\sum_{i\ge1}|x_i|^p<\infty.
\]
The metric is defined by
\[
\textsf{d}(x,y)=\left(
\sum_i|x_i-y_i|^p
\right)^{1/p}.
\]
When $p=2$, $\textsf{d}$ reduces into the $2$-norm.
Then $\ell^2$ is called the Hilbert space.
\end{itemize}
\end{example}

\begin{proposition}
The space $\ell^p$ is a metric space.
\end{proposition}
We proof this result by four intermediate steps:
\begin{itemize}
\item
An auxiliary inequality.
\item
the 
Holder inequality.
\item
the Minkowski inequality.
\item
the triangle inequality.
\end{itemize}
\begin{proof}
\begin{itemize}
\item
Let $p>1$ and define $q$ by $1/p+1/q=1$.
Let $\alpha$ and $\beta$ be any non-negative numbers, then we have 
\[
\alpha\beta \le \frac{\alpha^p}{p}+\frac{\beta^q}{q}.
\]
The condition $1/p+1/q=1$ implies $1/(p-1)=q-1$, then
\[
u=t^{p-1}\implies t=u^{q-1}.
\]
It follows that
\begin{align*}
\alpha\beta&\le 
\int_0^{\alpha}t^{p-1}\diff t + \int_0^{\beta}u^{q-1}\diff u
\\
&=\frac{\alpha^p}{p}+\frac{\beta^q}{q}.
\end{align*}
\item
Take any non-zero $x=(x_i)\in\ell^p$ and $y=(y_i)\in\ell^q$, we have the Holder inequality:
\[
\sum_{i\ge1}|x_iy_i|\le 
\left(
\sum_{i\ge1}|x_i|^p
\right)^{1/p}
\left(
\sum_{i\ge1}|y_i|^q
\right)^{1/q}.
\]
Let $(\tilde{x}_i)$ and $(\tilde{y})$ be such that 
\[
\sum|\tilde{x}_i|^p=1, \sum|\tilde{y}_i|^q=1.
\]
By applying the Auxiliary inequality,
\[
|\tilde{x}_i\tilde{y}_i|\le \frac{1}{p}|\tilde{x}_i|^p + \frac{1}{q}|\tilde{y}_i|^q.
\]
Therefore, 
\[
\sum_i|\tilde{x}_i\tilde{y}_i|\le \frac{1}{p}+\frac{1}{q}=1.
\]
Now take any non-zero $x=(x_i)\in\ell^p$ and $y=(y_i)\in\ell^q$. Set
\[
\tilde{x}_i=\frac{x_i}{\left(
\sum_i|x_i|^p
\right)^{1/p}},\quad
\tilde{y}_i=\frac{y_i}{\left(
\sum_i|y_i|^q
\right)^{1/q}}
\]
It follows that 
\[
\sum_i\left|
\frac{x_i}{\left(
\sum_i|x_i|^p
\right)^{1/p}}
\frac{y_i}{\left(
\sum_i|y_i|^q
\right)^{1/q}}
\right|\le 1.
\]
The desired result follows.
\end{itemize}
\end{proof}








