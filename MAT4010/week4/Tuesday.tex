
\chapter{Week4}

\section{Tuesday}


\subsection{Martingales in Discrete Time}

\begin{example}
Let $\mathbb{F}\triangleq\{\mathcal{F}_n\}_{n\ge0}$ be a filteration and consider a random variable 
$\zeta\in\mathcal{L}^1$.
Define $X_n\triangleq\mathbb{E}[\zeta\mid\mathcal{F}_n]$, and we can check that 
$\{X_n\}_{n\ge}$ is a martingale with respect to $\mathbb{F}$:
\begin{itemize}
\item
Firstly we need to show the integrability of $X_n$ for any $n$:
\begin{align*}
\mathbb{E}[|X_n|]&=\mathbb{E}[|\mathbb{E}[\zeta\mid\mathcal{F}_n]|]\\
&\le \mathbb{E}[\mathbb{E}[|\zeta|\mid\mathcal{F}_n]]\\
&=\mathbb{E}[|\zeta|]<\infty
\end{align*}
where the first inequality is by the Jensen's inequality.
\item
Then we check that $\mathbb{E}[X_{n+1}\mid\mathcal{F}_n]=X_n$ for any $n$:
\begin{align*}
\mathbb{E}[X_{n+1}\mid\mathcal{F}_n]&=\mathbb{E}[\mathbb{E}[\zeta\mid\mathcal{F}_{n+1}]\mid\mathcal{F}_n]\\
&=\mathbb{E}[\zeta\mid\mathcal{F}_n]=X_n.
\end{align*}
\end{itemize}
\end{example}

\begin{example}[Martingale Transform]\label{Exp:4:2}
Let $C_n$ be the stake to be bet on game $n$, and $X_n-X_{n-1}$ be the net with per stake in game $n$, with $n\ge1$.
Suppose that the process $\{C_n\}_{n\ge1}$ is predictable, and the total win up to time $n$ is 
$Y_n = \sum_{1\le k\le n}C_k(X_k - X_{k-1})$.
Define $Y_0:=0$.
If $\{X_n\}$ is a martingale w.r.t. $\{\mathcal{F}_n\}$ and $\{C_n\}$ is bounded a.s.\footnote{
Here the boundedness means that $|C_n(\omega)|\le M$ for some $M>0$ and almost all $\omega$
}, then we can show that $\{Y_n\}$ is also a martingale:

Firstly note that $Y_n$ is $\mathcal{F}_n$-measurable since $X_k,C_k$ are all $\mathcal{F}_n$-measurable for $1\le k\le n$.
Then we check $\{Y_n\}$ is a martingale w.r.t. $\{\mathcal{F}_n\}$:
% what kind of boundedness
% how to show that Yn is adapted to Fn, what is Fn
% independence of Cn
\begin{itemize}
\item
For any $n$, we have
\begin{align*}
\mathbb{E}[|Y_n|]&\le \sum_{1\le k\le n}\mathbb{E}[|C_k(X_k - X_{k-1})|]\\
&\le M\cdot \sum_{1\le k\le n}\mathbb{E}[|X_k - X_{k-1}|]\\
&\le M\cdot \sum_{1\le k\le n}\mathbb{E}[|X_k|] + \mathbb{E}[|X_{k-1}|]<\infty,
\end{align*}
where the second inequality is by the boundedness of $\{C_n\}$.
\item
Moreover, 
\begin{align*}
\mathbb{E}[Y_{n+1}\mid\mathcal{F}_n]&=\mathbb{E}\bigg[
\sum_{1\le k\le n+1}C_k(X_k - X_{k-1})\bigg|\mathcal{F}_n
\bigg]\\
&=\mathbb{E}[Y_n + C_{n+1}(X_{n+1} - X_n)\mid\mathcal{F}_n]\\
&=\mathbb{E}[Y_n\mid\mathcal{F}_n] + C_{n+1}\mathbb{E}[X_{n+1} - X_n\mid\mathcal{F}_n]\\
&=Y_n + C_{n+1}\mathbb{E}[X_{n+1} - X_n\mid\mathcal{F}_n]=Y_n,
\end{align*}
where the third equality is by the $\mathcal{F}_n$-measurability of $C_{n+1}$;
the fourth equality is by the $\mathcal{F}_n$-measurability of $Y_n$,
and the last equality is by $\mathbb{E}[X_{n+1}\mid\mathcal{F}_n]=X_n$.
\end{itemize}


\end{example}

\begin{theorem}
Suppose that $\{X_n\}$ is a martingale with respect to $\mathcal{F}_n$, and
$\phi:\mathbb{R}\to\mathbb{R}$ is a convex function such that $\phi(X_n)$ is integrable for all $n$,
then $\{\phi(X_n)\}$ is a sub-martingale with respect to $\mathcal{F}_n$.
\end{theorem}
\begin{proof}
By the Jensen's inequality and $\mathbb{E}[X_{n+1}\mid\mathcal{F}_n]=X_n$, we have
\[
\mathbb{E}[\phi(X_{n+1})\mid\mathcal{F}_n]\ge 
\phi(\mathbb{E}[X_{n+1}\mid\mathcal{F}_n])=\phi(X_n).
\] 
\end{proof}
By similar proof, we can show the following theorem:
\begin{theorem}
Suppose that $\{X_n\}$ is a sub-martingale with respect to $\mathcal{F}_n$,
and $\phi$ is an increasing convex function such that $\phi(X_n)$ is integrable for all $n$,
% miss the integrability condition 
then $\{\phi(X_n)\}$ is a sub-martingale with respect to $\mathcal{F}_n$.
\end{theorem}
A direct example is the following:
\begin{example}
Suppose that $\{X_n\}$ is a sub-martingale, define the convex function $X^+:=\max(X,0)$, then
$\{X_n^+\}$ is also a sub-martingale.
\end{example}

\begin{theorem}\label{The:4:3}
Let $\{X_n\}$ be a martingale and $T$ be a stopping time.
Define the stopped process $\{X_{n\land T}\}$ as
\[
X_{n\land T}(\omega)\triangleq X_{n\land T(\omega)}(\omega),\qquad\forall\omega\in\Omega,\forall n.
\]
Then $\{X_{n\land T}\}$ is a martingale. In particular, 
\[
\mathbb{E}[X_{n\land T}]=\mathbb{E}[X_0],\quad\forall n.
\]
\end{theorem}
\begin{proof}
We will show this result by applying the Martingale transform technique mentioned in Example~\ref{Exp:4:2}.
Define the stake process $\{C_n^T\}_{n\ge0}$ as
\[
C_n^T(\omega) = 1\{n\le T(\omega)\}(\omega),\qquad\forall\omega\in\Omega,\forall n.
\]
Note that $\{C_n^T\}_{n\ge0}$ is predictable since $\{C_n^T=0\}=\{T(\omega)\le n-1\}\in\mathcal{F}_{n-1}$.
Now we begin to simplify $\sum_{1\le k\le n}C_k^T(X_k - X_{k-1})$:
\begin{align*}
\sum_{1\le k\le n}C_k^T(X_k - X_{k-1})
&=1\{T\ge1\}(X_1-X_0)+1\{T\ge2\}(X_2-X_1)+\cdots+1\{T\ge n\}(X_n - X_{n-1})\\
&=-1\{T\ge1\}X_0 + (1\{T\ge1\} - 1\{T\ge2\})X_1 + (1\{T\ge2\} - 1\{T\ge3\})X_2\\
&+\cdots+
(1\{T\ge n-1\} - 1\{T\ge n\})X_{n-1}+1\{T\ge n\}X_n\\
&=1\{T\ge n\}X_n-1\{T\ge1\}X_0 + \sum_{i=1}^{n-1}1\{T=i\}X_i\\
&=\bigg(
1\{T\ge n\}X_n+\sum_{i=0}^{n-1}1\{T=i\}X_i
\bigg)-\bigg(1\{T\ge1\}X_0-1\{T=0\}X_0\bigg)\\
&=X_{n\land T} - X_0.
\end{align*}
By the boundedness of $\{C_n^T\}$, and the result in Example~\ref{Exp:4:2},
we can show that $\{X_{n\land T} - X_0\}$ is a martingale, i.e., $\{X_{n\land T}\}$ is a martingale.
Therefore,
\begin{align*}
\mathbb{E}[X_{n\land T}]&=\mathbb{E}[\mathbb{E}[X_{n\land T}\mid\mathcal{F}_{n-1}]]
=\mathbb{E}[X_{(n-1)\land T}]=\cdots=\mathbb{E}[X_{0\land T}]=\mathbb{E}[X_0].
\end{align*}
\end{proof}

Note that $\mathbb{E}[X_T]$ does not equal to $\lim_{n\to\infty}\mathbb{E}[X_{n\land T}]=\mathbb{E}[X_0]$. The following provides a counter-example:
\begin{example}
Let $\{X_n\}$ be a simple symmetric random walk on integers and $X_0=0$.
Then $\{X_n\}$ is a martingale.Define the stopping time
\[
T\triangleq\inf\{n\ge0~X_n=1\}.
\]
Then $\mathbb{E}[X_{n\land T}]=\mathbb{E}[X_0]=0,\forall n$.
Since the random walk is recurrent, $\mathbb{P}(T<\infty)$, and $X_T=1$ a.s., which implies that
\[
1 = \mathbb{E}[X_T]\ne \lim_{n\to\infty}\mathbb{E}[X_{n\land T}]=\mathbb{E}[X_0]=0.
\]
\end{example}
The Doob's optional stopping theorem provides sufficient conditions for 
$\mathbb{E}[X_T]=\mathbb{E}[X_0]$:

\begin{theorem}[Doob's Optional Stopping Theorem]
Let $\{X_n\}$ be a martingale and $T$ be a stopping time.
Then $X_T$ is integrable and $\mathbb{E}[X_T]=\mathbb{E}[X_0]$ if any of the following conditions hold:
\begin{enumerate}
\item
$T$ is bounded a.s.; % what do you mean by a.s.
\item
$\{X_n\}$ is bounded and $T$ is finite $(\mathbb{P}(T<\infty)=1)$\footnote{The finiteness is a weaker condition, which does not imply boundedness}, a.s.;
\item
$\mathbb{E}[T]<\infty$\footnote{The $\mathbb{E}[T]<\infty$ is a little bit stronger condition than finiteness, but still does not imply boundedness.} and $\{|X_n - X_{n-1}|\}$ is bounded.
\end{enumerate}
% difference between finite and bounded?
\end{theorem}
\begin{proof}
\begin{enumerate}
\item
Suppose that $T$ is bounded a.s., which means that there exists $K$ such that
$\mathbb{P}\{\omega:~T(\omega)\le K\}=1$.
Therefore, $X_{K\land T}=X_T$ a.s. By Theorem~\ref{The:4:3},
$X_T$ is integrable with $\mathbb{E}[X_T] = \mathbb{E}[X_{K\land T}]=\mathbb{E}[X_0]$.
\item
Suppose that $T$ is finite a.s., then we can show that $X_{n\land T}\to X_T$ a.s.:
Note that $\mathbb{P}\{T<\infty\}=1$, and
\[
\omega\in\{T<\infty\}\implies \lim_{n\to\infty}X_{n\land T}(\omega)= X_T(\omega).
\]
%which implies that $X_{n\land T}\to X_T$ a.s. as $n\to\infty$.
Since $\{X_n\}$ is bounded a.s., $\{X_{n\land T}\}$ is bounded a.s. as well.
By the Bounded Convergence Theorem, $X_T$ is integrable % how to show?
and $X_{n\land T}\to X_T$ in $L^1$, which implies that
\[
\mathbb{E}[X_T] = \lim_{n\to\infty}\mathbb{E}[X_{n\land T}]=\mathbb{E}[X_0].
\]
\item
We first show that $\{X_{T\land n}-X_0\}$ is dominated by an integrable random variable:
\begin{align*}
|X_{T\land n}-X_0|&=\left|
\sum_{k=1}^{T\land n}(X_k - X_{k-1})
\right|\le \sum_{k=1}^{T\land n}|X_k - X_{k-1}|\le M\cdot(T\land n)\le MT,
\end{align*}
where the second inequality is by the boundedness of $\{|X_n - X_{n-1}|\}$, i.e., for any $n$ and $\omega\in\Omega$, $|X_n(\omega) - X_{n-1}(\omega)|\le M$.
Considering that $\mathbb{E}[T]<\infty$, $T$ is finite a.s., % how to show?
which implies that $X_{n\land T}\to X_T$.
Applying the dominated convergence theorem, $X_{T\land n}\to X_T$ in $L^1$, and
\[
\mathbb{E}[X_T] = \lim_{n\to\infty}\mathbb{E}[X_{n\land T}]=\mathbb{E}[X_0].
\]
\end{enumerate}
\end{proof}

\subsection{Doob's Inequalities}

\begin{theorem}[Doob's Optional Sampling Theorem]\label{The:sampling}
Let $\{X_n\}$ be a martingale and $S,T$ be two bounded stopping times, with $S\le T$ a.s.,
then $\mathbb{E}[X_T\mid\mathcal{F}_S] = X_S$ a.s. 

Moreover, if instead $\{X_n\}$ is assumed to be a sub-martingale or super-martingale, then the equality in the result is replaced by $\ge$ or $\le$, respectively.
\end{theorem}

\begin{proof}
We only show the result based on the assumption that $\{X_n\}$ is a sub-martingale, since the remaining part follows the similar logic.
Since $S,T$ are bounded a.s., random variables $X_T$ and $X_S$ are integrable.
In order to simplify $\mathbb{E}[X_T\mid\mathcal{F}_S]$, we need to study the structure of $\mathcal{F}_S$:
For any $A\in\mathcal{F}_S$, by the definition of stopping-time $\sigma$-algebra, $A\cap\{S\le j\}\in\mathcal{F}_j$.
Considering that $\{S>j-1\}=\{S\le j-1\}^c\in\mathcal{F}_{j-1}\subseteq\mathcal{F}_j$ and $\{T>j\}=\{T\le j\}^c\in\mathcal{F}_j$,
\[
A\cap\{S\le j\}\cap\{S>j-1\}\cap\{T>j\}=A\cap\{S=j\}\cap\{T>j\}\in\mathcal{F}_j,\quad\forall j.
\]
\begin{itemize}
\item
Assume that $0\le T-S\le 1$, a.s., it follows that 
\begin{subequations}
\begin{align}
\int_A(X_T-X_S)\diff\mathbb{P}&=\sum_{j=0}^N\int_{A\cap\{S=j\}}(X_T-X_S)\diff\mathbb{P}\label{Eq:4:1:a}\\
&=\sum_{j=0}^N\int_{A\cap\{S=j\}\cap\{T>j\}}(X_T-X_S)\diff\mathbb{P}
+
\sum_{j=0}^N\int_{A\cap\{S=j\}\cap\{T=j\}}(X_T-X_S)\diff\mathbb{P}\label{Eq:4:1:b}\\
&=\sum_{j=0}^N\int_{A\cap\{S=j\}\cap\{T>j\}}(X_T-X_S)\diff\mathbb{P}\label{Eq:4:1:c}
\end{align}
where (\ref{Eq:4:1:a}) is by the assumption that $S$ is bounded a.s., i.e., $|S|\le N$ a.s.;
(\ref{Eq:4:1:b}) is by the assumption that $T\ge S$;
(\ref{Eq:4:1:c}) is because $1\{S=j, T=j\}\cdot(X_T-X_S)=0$.
\end{subequations}
Since $\mathbb{E}[X_{j+1}\mid\mathcal{F}_j]\le X_j$ a.s. for any $j$,
\begin{equation}\label{Eq:4:2}
\int_{A\cap\{S=j\}\cap\{T>j\}}(X_{j}-X_{j+1})\diff\mathbb{P}\ge0\implies
\int_A(X_S-X_T)\diff\mathbb{P}\ge0,\forall A\in\mathcal{F}_S.
\end{equation}
For two $\mathcal{F}$-measurable random variables, if $\int_A(X-Y)\diff\mathbb{P}\le0, \forall A\in\mathcal{F}$, then one can assert that $X\le Y$ a.s. Therefore, (\ref{Eq:4:2}) implies 
$\mathbb{E}[X_T\mid\mathcal{F}_S]\le X_S$ a.s.
\item
Now suppose that $T-S\ge0$ a.s., and construct intermediate variables $R_j=T\land(S+j), j=1,2,\ldots,N$.
It follows that $R_j$ is a stopping time and $S\le R_1\le R_2\le\cdots\le R_N\le T$ a.s., with
\[
0\le R_1-S\le 1,\quad 0\le R_j-R_{j-1}\le 1,\forall j\quad 0\le T-R_N\le 1.
\]
Consider any $A\in\mathcal{F}_S$, and since $0\le R_1-S\le 1$ a.s.,
\[
\int_A(X_{S} - X_{R_1})\diff\mathbb{P}\ge0.
\]
By definition of stopping time $\sigma$-algebra, $A\cap\{S\le j\}\in\mathcal{F}_j$, which implies that
\[
A\cap\{S\le j\}\cap\{R_1\le j\}=A\cap\{R_1\le j\}\in\mathcal{F}_j\implies A\in\mathcal{F}_{R_1}.
\]
Considering that $0\le R_2-R_1\le 1$ a.s., 
\[
\int_A(X_{R_1} - X_{R_2})\diff\mathbb{P}\ge0.
\]
Similarly,
\[
\int_A(X_{R_{j-1}} - X_{R_j})\diff\mathbb{P}\ge0, j=2,\ldots,N,\quad
\int_A(X_{R_{N}} - X_{T})\diff\mathbb{P}\ge0.
\]
Adding those integrals above, $\int_A(X_S-X_T)\diff\mathbb{P}\ge0,\forall A\in\mathcal{F}_S$, i.e., 
\[
\mathbb{E}[X_T\mid\mathcal{F}_S]\le X_S,\quad\text{a.s.}
\]
\end{itemize}
\end{proof}


\begin{remark}
We can assume the uniform integrability of $\{X_n\}$ and the conclusion still holds, without assuming that $T,S$ are bounded.
% how to prove?
\end{remark}

\section{Thursday}
\subsection{Doob's Maximal Inequality}

\begin{theorem}[Doob's Maximal Inequality]\label{The:max:in}
Let $\{X_n\}$ be a super-martingale. Choose some $N>0$ % any N?
, then for any $\lambda>0$,
\begin{enumerate}
\item
$\lambda\cdot\mathbb{P}\bigg(
\sup\limits_{k\le N}X_k\ge\lambda
\bigg)\le \mathbb{E}[X_0] - \mathbb{E}\left[X_N\cdot 1\left\{\sup\limits_{k\le N}X_k<\lambda\right\}\right]$;
\item
$\lambda\cdot\mathbb{P}\bigg(
\inf\limits_{k\le N}X_k\le-\lambda
\bigg)\le \mathbb{E}\left[(-X_N)\cdot 1\left\{\inf\limits_{k\le N}X_k\le-\lambda\right\}\right]$.
\end{enumerate}
\end{theorem}

\begin{proof}
\begin{enumerate}
\item
Define a stopping time $R$:
\[
R(\omega) = \inf\{k\ge0:~X_k(\omega)\ge\lambda\},\quad\forall\omega\in\Omega.
\]
Take $T=R\land N$, which is a bounded stopping time. 
Apply the Optional Sampling Theorem~\ref{The:sampling},
\begin{align*}
\mathbb{E}[X_0]&\ge \mathbb{E}\bigg[
\mathbb{E}[X_T\mid\mathcal{F}_0]
\bigg]=\mathbb{E}[X_T]\\
&=\int 1\bigg\{
\sup\limits_{k\le N}X_k\ge\lambda
\bigg\}X_T\diff\mathbb{P}+\int 1\bigg\{
\sup\limits_{k\le N}X_k<\lambda
\bigg\}X_T\diff\mathbb{P}\\
&\ge
\lambda\cdot\mathbb{P}\bigg(
\sup\limits_{k\le N}X_k\ge\lambda
\bigg)+\int 1\bigg\{
\sup\limits_{k\le N}X_k<\lambda
\bigg\}X_N\diff\mathbb{P}
\end{align*}
where the first inequality is because that $X_0\ge\mathbb{E}[X_T\mid\mathcal{F}_0]$ a.s.;
and the last inequality is because that 
conditioned on the event $\bigg\{
\sup\limits_{k\le N}X_k<\lambda
\bigg\}$, 
$X_T\equiv X_N$.
%
%$X_T\ge\mathbb{E}[X_N\mid\mathcal{F}_T]$.
%(How to show the measurability of $\bigg\{
%\sup\limits_{k\le N}X_k<\lambda
%\bigg\}$?)
Thus the desired result holds.
\item
Let $Y_n=-X_n$, and $\{Y_n\}$ is a sub-martingale.
Define the stopping time
\[
R(\omega) = \inf\{k\ge0:~Y_k(\omega)\ge\lambda\},\quad\forall\omega\in\Omega.
\]
Take $T=R\land N$, which is a bounded stopping time. 
Apply the Optional Sampling Theorem~\ref{The:sampling},
\begin{align*}
\mathbb{E}[Y_N]\ge\mathbb{E}[Y_T]\ge \lambda\mathbb{P}\bigg(
\sup\limits_{k\le N}Y_k\ge\lambda
\bigg)+\mathbb{E}\bigg[
Y_N1\bigg(
\sup\limits_{k\le N}Y_k<\lambda
\bigg)
\bigg]
\end{align*}
It follows that
\begin{align*}
\lambda\cdot\mathbb{P}\bigg(
\inf\limits_{k\le N}X_k\le-\lambda
\bigg)
&=
\lambda\cdot\mathbb{P}\bigg(
\sup\limits_{k\le N}Y_k\ge\lambda
\bigg)\le \mathbb{E}[Y_N] - \mathbb{E}\bigg[
Y_N1\bigg(
\sup\limits_{k\le N}Y_k<\lambda
\bigg)
\bigg]\\
&=\mathbb{E}\bigg[
(-X_N)1\bigg(
\inf\limits_{k\le N}X_k\le-\lambda
\bigg)
\bigg].
\end{align*}

\end{enumerate}
\end{proof}

\begin{remark}
Summing up these two results in Theorem~\ref{The:max:in}, we imply
\begin{align*}\lambda\cdot\mathbb{P}
\bigg(
\sup\limits_{k\le N}|X_k|\ge\lambda
\bigg)
&\le
 \mathbb{E}[X_0] - \mathbb{E}\left[X_N\cdot 1\left\{\sup\limits_{k\le N}X_k<\lambda\right\}\right]
 -\mathbb{E}\left[X_N\cdot 1\left\{\inf\limits_{k\le N}X_k\le-\lambda\right\}\right]\\
 &\le \mathbb{E}[X_0] + 2\mathbb{E}[X_N^-],
\end{align*}
where $X^-\triangleq\max(-X,0)$.
\end{remark}

\begin{theorem}
Let $\{X_n\}$ be a martingale. Choose some $N>0$ and let $X_N\in\mathcal{L}^2$, i.e., $\mathbb{E}[X_N^2]<\infty$. Then for any $\lambda>0$,
\[
\mathbb{P}\bigg(
\sup\limits_{k\le N}|X_k|>\lambda
\bigg)\le\frac{1}{\lambda^2}\mathbb{E}[X_N^2].
\]
\end{theorem}

\begin{proof}
We can show that $\{X_n^2\}$ is a sub-martingale by applying Jensen's inequality:
\[
\mathbb{E}[X_{n+1}^2\mid\mathcal{F}_n]\ge \left(
\mathbb{E}[X_{n+1}\mid\mathcal{F}_n]
\right)^2=X_n^2.
\]
As a result, $\mathbb{E}[X_k^2]\le\mathbb{E}[X_N^2]<\infty, \forall k\le N$, i.e., $\{-X_k^2\}_{k\le N}$ is a super-martingale.
Apply the second part in Theorem~\ref{The:max:in} completes the proof:
\begin{align*}
\lambda^2\cdot\mathbb{P}\bigg(
\inf\limits_{k\le N}(-X_k^2)\le-\lambda^2
\bigg)
&=\lambda^2\cdot\mathbb{P}\bigg(
\sup\limits_{k\le N}|X_k|\ge\lambda
\bigg)\\
&\le 
\mathbb{E}\left[X_N^2\cdot 1\left\{\inf\limits_{k\le N}(-X_k^2)\le-\lambda^2\right\}\right]\\
&\le \mathbb{E}[X_N^2].
\end{align*}


\end{proof}

\begin{theorem}[Doob's $L^p$-inequality]
\begin{enumerate}
\item
Suppose that $\{X_n\}$ is a sub-martingale, then for any $p>1$,
\[
\mathbb{E}\bigg[
\left(
\sup\limits_{k\le n}X_k^+
\right)^p
\bigg]\le \left(
\frac{p}{p-1}
\right)^p\mathbb{E}[(X_n^+)^p].
\]
\item
Suppose that $\{X_n\}$ is a martingale, then for any $p>1$,
\[
\mathbb{E}\bigg[
\sup\limits_{k\le n}|X_k|^p
\bigg]\le \left(
\frac{p}{p-1}
\right)^p\mathbb{E}[|X_n|^p].
\]
\end{enumerate}
\end{theorem}


\begin{proof}
\begin{enumerate}
\item
W.l.o.g., assume that $\{X_n\}$ is non-negative, and we may replace $X_n^+$ by $X_n$.
Consider a continuous increasing function $\phi:\mathbb{R}_+\to[0,+\infty)$ with $\phi(0)=0$, and we evaluate the expectation for $\phi(Z)$, where $Z$ is a given random variable:
\begin{subequations}
\begin{align}
\mathbb{E}[\phi(Z)]
&=\int_{\Omega}\phi(Z(\omega))\diff\mathbb{P}(\omega)\nonumber\\
&=\int_{\Omega}\int_0^{Z(\omega)}\diff\phi(y)\diff\mathbb{P}(\omega)\nonumber\\
&=\int_{\Omega}\int_{[0,\infty)}1\{y\le Z(\omega)\}\diff\phi(y)\diff\mathbb{P}(\omega)\nonumber\\
&=\int_{[0,\infty)}\int_{\Omega}1\{y\le Z(\omega)\}\diff\mathbb{P}(\omega)\diff\phi(y)\label{Eq:4:3:a}\\
&=\int_{[0,\infty)}\mathbb{P}(Z\ge y)\diff\phi(y)\nonumber
\end{align}
where (\ref{Eq:4:3:a}) is by the Fubini's theorem.

Take $\phi(y)\equiv y^p$ and define $X_n^*=\sup_{k\le n}X_n$\footnote{called the running maximal of $\{X_n\}$} for notation simplifcation.
As a result,
using a little bit calculus gives
\begin{align}
\mathbb{E}[(X_n^*)^p]&=\int_{[0,\infty)}\mathbb{P}(X_n^*\ge \lambda)\diff\lambda^p\nonumber\\
&\le 
\int_{[0,\infty)}\frac{1}{\lambda}\mathbb{E}\left[
|X_n|1\left\{
\sup\limits_{k\le n}|X_k|\ge\lambda
\right\}
\right]\diff\lambda^p\label{Eq:4:3:b}\\
&=\int_0^\infty\frac{1}{\lambda}\int_{\Omega}
X_n(\omega)1\left\{
X_n^*(\omega)\ge\lambda
\right\}\diff\mathbb{P}(\omega)\diff\lambda^p\label{Eq:4:3:c}\\
&=\int_{\Omega}X_n(\omega)\int_0^\infty\frac{1}{\lambda}1\left\{
X_n^*(\omega)\ge\lambda
\right\}\diff\lambda^p\diff\mathbb{P}(\omega)\label{Eq:4:3:d}\\
&=\int_{\Omega}X_n(\omega)\int_0^{X_n^*(\omega)}p\lambda^{p-2}\diff\lambda\diff\mathbb{P}(\omega)\nonumber\\
&=\int_{\Omega}X_n(\omega)\frac{p}{p-1}\left[X_n^*(\omega)\right]^{p-1}\diff\mathbb{P}(\omega)\nonumber\\
&\le\frac{p}{p-1}\left(
\mathbb{E}[(X_n)^p]
\right)^{1/p}\left(
\mathbb{E}[(X_n^*)^{(p-1)q}]
\right)^{1/q},\quad\text{with $1/q = 1-1/p$}\label{Eq:4:3:e}\\
&=\frac{p}{p-1}\left(
\mathbb{E}[(X_n)^p]
\right)^{1/p}\left(
\mathbb{E}[(X_n^*)^{p}]
\right)^{(p-1)/p}\label{Eq:4:3:f}
\end{align}
where (\ref{Eq:4:3:b}) is by the Doob's maximal inequality;
(\ref{Eq:4:3:c}) is by the assumption that $X_n\ge0$;
(\ref{Eq:4:3:d}) is by Fubini's theorem;
(\ref{Eq:4:3:e}) is by Holder's inequality.
If dividing both sides in (\ref{Eq:4:3:f}) by $\left(
\mathbb{E}[(X_n^*)^{p}]
\right)^{(p-1)/p}$, we get the desired result.
\end{subequations}
\item
The second inequality follows by applying the first and replacing $\{X_n\}$ with $\{|X_n|\}$.
\end{enumerate}
\end{proof}


\begin{example}
Let $\{X_n\}$ be a non-negative sub-martingale. Then we can apply the similar proceed to show the following upper bound holds:
\[
\mathbb{E}\bigg[
\sup_{k\le n}X_k
\bigg]
\le
\frac{e}{e-1}\bigg(
1+\sup\limits_{k\le n}\mathbb{E}[X_k\log^+X_k]
\bigg),
\]
where $\log^+X\triangleq(\log X)1\{X\ge1\}$.

Tale $\phi(y)=(y-1)^+$, then
\begin{subequations}
\begin{align}
\mathbb{E}[(X_n^*-1)^+]
&=\int_0^\infty\mathbb{P}(X_n^*\ge\lambda)\diff\phi(\lambda)\nonumber\\
&\le \int_0^\infty\frac{1}{\lambda} \mathbb{E}[X_n1\{X_n^*\ge\lambda\}]\diff\phi(\lambda)\label{Eq:4:4:a}\\
&=\int_{\Omega}X_n\int_0^{X_n^*}\frac{1}{\lambda}\diff\phi(\lambda)\diff\mathbb{P}\label{Eq:4:4:b}\\
&=\int_{\Omega}X_n\int_0^{X_n^*}\frac{1}{\lambda}1\{\lambda\ge1\}\diff\lambda\diff\mathbb{P}\nonumber\\
&=\int_{\Omega}X_n1\{X_n^*\ge1\}\log X_n^*\diff\mathbb{P}\nonumber\\
&=\mathbb{E}[X_n\log^+X_n^*].\nonumber
\end{align}
where (\ref{Eq:4:4:a}) is by the Doob's maximal inequality, and (\ref{Eq:4:4:b}) is by Fubini's theorem.
As a result,
\[
\mathbb{E}[X_n^*]-1\le \mathbb{E}[(X_n^*-1)^+]\le \mathbb{E}[X_n\log^+X_n^*].
\]
We can use a bit calculus to show that 
\begin{equation}
a\log^+b\le a\log^+a+\frac{b}{e},
\end{equation}
which implies that
\[
\mathbb{E}[X_n^*]-1\le \mathbb{E}[X_n\log^+X_n]+\frac{1}{e}\mathbb{E}[X_n^*].
\]
The proof is complete.
\end{subequations}
\end{example}























