
\chapter{Week3}

\section{Tuesday}\index{week3_Tuesday_lecture}
\begin{theorem}[Optimality Condition]
\begin{itemize}
\item
primal feasible: $\bm{Ax}=\bm b,\bm x\ge0$
\item
Dual feasible: $\bm{A}\trans\bm y\le\bm c$
\item
Complementarity: $\bm x\circ \bm s=\bm0$, i.e., $x_i\cdot (c_i-\bm A_i\trans\bm y)=\bm0$ for each $i$.
\end{itemize}
\end{theorem}
\begin{remark}
(Primal) Simplex method:
\begin{enumerate}
\item
Always keep primal feasibility:
\item
Always keep complementarity:

Define $\bm y=(\bm A_B^{-1})\trans\bm c_B$ as the dual solution. The reduced costs vector is $\bm c\trans-\bm c_B\trans\bm A_B^{-1}\bm A=\bm c-\bm y\trans\bm A$
\item
Not necessarily keep dual feasible until get the optimal solution, i.e., it will seeks solution that is dual feasible.
\end{enumerate}
\end{remark}
\paragraph{Dual Simplex method}
Dual Simplex method remains both dual feasiblity and complementarity conditions in each iteration but seeks primal feasibility.

Cases for applying dual simplex method:
\begin{itemize}
\item
There is a dual BFS available but no primal BFS available.
\item
$\bm b$ is changed by a large amount or a constraint isadded, i.e., lose the primal feasible solution.
\end{itemize}

\paragraph{Interior Point Method}
Consider the relaxed version of optimality condition:
\begin{align*}
\bm{Ax}&=\bm b,\bm x\ge0\\
\bm{A}\trans\bm y+\bm s&=\bm c,\bm s\ge0\\
x_i\cdot s_i&=\mu,\quad\forall i,\mbox{small }\mu_i>0
\end{align*}
Keep decreasing $\mu$ and finally get the solution to LP.
\begin{remark}
\begin{itemize}
\item
The optimal solution output from interior point method may not necessarily BFS. If the optimal solution is unique, it is BFS.
\item
Initial solution for the interior point method can be found by solving the auxiliary problem.
\item
The complexity for interior point method is $O(n^{3.5})$
\item
The interior point method gives stable running time compared with simplex method.
\item
Interior point method always find the optimal solution with maximum possible number of \emph{non-zeros}.
\item
Interior point method finds high-rank solution (the center of all optimal solutions); but the simplex method finds the low-rank solution.
\end{itemize}
\end{remark}

\subsection{Reviewing}
\paragraph{Linear optimization formulation}
Standard Form LP Transformation
\[
\begin{array}{ll}
\min&\bm c\trans\bm x\\
\mbox{such that}&\bm{Ax}=\bm b\\
&\bm x\ge\bm0
\end{array}
\]

Maximin / minimax objective

Absolute values in objective function or constraints.
\begin{theorem}
The BFS for standard LP is equivalent to extreme point.
\end{theorem}
\begin{theorem}
If there is a feasible solution, then there is a basic feasible solution; If there is a optimal solution, then there is a basic feasible optimal solution.
\end{theorem}
Care about corollary

\paragraph{Simplex method}
\begin{enumerate}
\item
Understand how simplex method works, and cases for unbounded, infeasible
\item
Apply simplex method to solve small LPs
\item
Read and interpret simplex tableau (make use of it to avoid inverse calculation)
\item
Apply two-phase method
\end{enumerate}




\paragraph{Duality Theory}
\begin{enumerate}
\item
Be able to constrauct the dual for any LP.
\item
Know the (strong/weak) duality theorems and apply them in different situations.
\item
Be able to write down the complentarity conditions and apply them



\end{enumerate}



\paragraph{Sensitivity Analysis}
Related to duality theory; 




\paragraph{Complexity Theory and interior method}
Complexity of LP:
\begin{enumerate}
\item
No guarntee of simplex method to achieve polynomial time
\item
Interior point can achieve polynomial time
\end{enumerate}

Properties of simplex method



























