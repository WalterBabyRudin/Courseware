%\chapter{Introduction to Linear Programming}
\chapter{Graph Coloring}

\paragraph{Problem Setting}

\begin{definition}[Colourable]
Consider an undirected graph $G$ without loops.
$G$ is $c$-colourable if 
each vertex can be assigned one of $c$ colours such that
the adjacent vertices have different colours.
\end{definition}
\begin{definition}[Chromatic]
If $G$ is $k$-colourable but not $(k-1)$-colourable, then $G$ is called $k$-chromatic, and $\mathcal{X}(G)=k$.
\end{definition}
w.l.o.g., assume that $G$ is simple, loopless, and connected.

What is the chromatic number of:
\begin{itemize}
\item
Complete graph $K_n$?
\item
The null graph (has no edges)?
\item
A bipartite graph?
\item
A path? A tree?
\item
A cycle? even or odd cycle?
\end{itemize}


\paragraph{Applications of Graph colouring}
\begin{enumerate}
\item
Exam period arranging:
The vertices are classes. The class $v$ and $w$ are adjacent if $v,w$ belong to the same student.
\item
What is the minimum phases needed for the traffic light so that all cars in all lanes can go through the
intersection?
The vertices are traffice lanes. The lanes $i$ and $j$ are adjacent if there is a collision between them.
\end{enumerate}
\section{Facts about Colourability}
\begin{theorem}
Let $G$ be a simple graph with maximum vertex degree $\Delta$. Then $G$ is $(\Delta+1)$-colourable.
\end{theorem}
\begin{proof}
By induction on the number of vertices.
For the null graph with one vertex, it is 1-colourable.
Consider simple graph $G$ with $n$ vertices. Delete any vertex $v$ of $G$ and its incident edges, the resultant graph $G'$ has $(n-1)$ vertices and degree at most $\Delta$, which is $(\Delta+1)$ colourable. Therefore, $G$ is $(\Delta+1)$-colourable by assigning a colour to $v$ that is different to all of its adjacent vertices.
\end{proof}

\begin{theorem}[Stronger Result for Colourability]
Let $G$ be a simple connected graph, and $\Delta(G)$ denote the maximum vertex degree of $G$.
If $G$ is neither a complete graph nor an odd cycle, then $G$ is $\Delta$-colourable, i.e., $\mathcal{X}(G)\le\Delta(G)$.
\end{theorem}
\begin{remark}
Brooks’ theorem give a tight bound for the chromatic number if a graph is regular, or “close” to regular, but it is not very helpful for graphs where a few vertices have very large degree
\end{remark}
\begin{theorem}[Colourability for Planar Graphs]
Every simple planar graph is $6$-colourable.
\end{theorem}
\begin{proof}
Clearly, the result holds for all graphs with 6 or fewer vertices. Now we consider a planar graph $G$ with $n$ vertices, and assume that all simple planar graphs with $n - 1$ vertices are 6-colourable.

Recall that a simple planar graph must have a vertex $v$ with degree at most $5$.
Removing this vertex and its incident edges, we obtain a graph with $n-1$ vertices that is $6$-colourable. Assign a colour to $v$ that is different to its 5 neighbours.
\end{proof}

\begin{theorem}[Colourability for Planar Graphs]
Every simple planar graph is $5$-colourable.
\end{theorem}
\begin{proof}
Clearly, the result holds for all graphs with 5 or fewer vertices. Now we consider a planar graph $G$ with $n$ vertices, and assume that all simple planar graphs with $n - 1$ vertices are 5-colourable.

Recall that a simple planar graph must have a vertex $v$ with degree at most $5$.
Removing this vertex and its incident edges, we obtain a graph with $n-1$ vertices that is $5$-colourable.
\begin{itemize}
\item
If this $v$ has degree no more than $4$, we are done.
\item
Otherwise $v$ has degree $5$ with neighbours $v_1,\dot,v_5$. If all $v_i$'s are pairwise adjacent, $K_5\subseteq G$, i.e., $G$ cannot be planar. Suppose $v_1$ and $v_2$ are not adjacent. The graph contracting $(v,v_1)$ and $(v,v_2)$ are $5$-colurable. We colour $v_1$ and $v_2$ with the colour assigned to the combined vertex in $G'$, and colour $v$ with a colour different from $v_1,v_2$, and the colours of $v_3,v_4,v_5$.
\end{itemize}
\end{proof}

\begin{theorem}
Every simple planar graph is $4$-colourable.
\end{theorem}

\begin{remark}
a graph is obtained if a point is assigned in each county, and a line segment joins the points of two counties that share a boundary.
Therefore, the $4$-colour problem for maps is equivalent to the $4$-colour vertex colouring problem for planar graphs.
\end{remark}


\section{Colouring Edges}
\begin{definition}
Consider an undirected graph $G$ without loops.
$G$ is $h$-edge-colourable if each edge can be assigned one of $h$ colours such that the adjacent edges have different colours.

If $G$ is $k$-edge-colourable but not $(k-1)$-colourable, then $k$ is called the \emph{chromatic index} of $G$, denoted as $\mathcal{X}'(G)$.
 \end{definition}
We assume that $G$ is loopless and connected, but multiple edges are allowed.

\begin{theorem}[Edge Colouring for Complete Graphs]
For $n\ge2$, $\mathcal{X}'(K_n)=n$ for odd $n$,
and $\mathcal{X}'(n)=n-1$ for even $n$.
\end{theorem}
\begin{proof}
Assume $n\ge3$.
\begin{itemize}
\item
For odd $n$, consider an $n$-cycle in $K_n$. Colour the edges of the cycle with a different colour for each edge. Colour each remaining edge automatically. We have a $n$-colurable scheme.
The maximum number of edges with the same colour~(non-adjacent) is $(n-1)/2$, so the maximum number of coloured edges is $\mathcal{X}'(K_n)\cdot (n-1)/2$, i.e., $\mathcal{X}'(K_n)=n$
\item
For even $n$, consider $K_{n-1}$ obtained by removing one vertex $v$ and its incident edges, which is $n-1$ coloruable. Colour that edge that joints each vertex to $v$ with the missing colour.
\end{itemize}
\end{proof}

\begin{theorem}
For a bipartite graph $G$ with maximum vertex degree $\Delta$, $\mathcal{X}'(G)=\Delta$.
\end{theorem}
\begin{proof}
The proof is by induction on the number of edges; we show that if all but one of the edges has been coloured using $\Delta$ colours, then the remaining edge can also be coloured to obtain a $\Delta$-colouring of $G$. We call this an augmentation.
\end{proof}

\begin{theorem}
If $G$ is a simple grpah, then $\Delta(G)\le\mathcal{X}'(G)\le\Delta(G)+1$
\end{theorem}
However, determining whether a graph belongs to Class I or Class II is not easy.














