
\chapter{Week 10}

\section{Monday}\index{Monday_lecture}
\subsection{Stationary Meansure and Distribution}
\begin{definition}[stationary measure]
\begin{itemize}
\item
We call $\{\pi(i):~i\in S\}$ a measure on $S$ if $\pi(i)\ge0$ for any $i\in S$.
\item
Furthermore, it is called a stationary measure if 
\[
\pi(j) = \sum_{i\in S}\pi(i)p_{i,j},\quad j\in S,
\]
and the equation above is called a stationary equation.
\item
In particular, $\{\pi(i):~i\in S\}$ is called a stationary equation if 
\[
\sum_{i\in S}\pi(i)=1.
\]
\end{itemize}
\end{definition}

\begin{example}
Transition matrix $P$ is said to be doubly stochastic if $p_{i,j}=p_{j,i}$.
A Markov chain with doubly stochastic transitions always has the stationary measure
:
$\pi(i)\equiv C$.
\end{example}

\begin{remark}
Inductively, we can show that $\pi(k)=\sum_{i\in S}\pi(i)p_{i,k}^{(n)}$:
\begin{align*}
\pi(k)&=\sum_{i\in S}\pi(i)p_{i,k}=\sum_{i\in S}\sum_{j\in S}\pi(j)p_{j,i}p_{i,k}\\
&=\sum_{j\in S}\pi(j)\left[
\sum_{i\in S}p_{j,i}p_{i,k}
\right]=\sum_{j\in S}\pi(j)\pi_{j,k}^{(2)}.
\end{align*}
In other words, if the initial measure is $\pi$, then this measure is invariant of time.
That's why $\pi$ is called a stationary measure.
\end{remark}

\begin{theorem}
For a Markov chain with state space $S$ and transition probabilities $\{p_{i,j}\}$,
if it is irreducible and aperiodic, then the following two conditions are equivalent:
\begin{enumerate}
\item
It has a stationary equation;
\item
It is positive recurrent.
\end{enumerate}
If either one of the conditions holds, then the stationary distribution can be uniquely obtained as
\[
\lim_{n\to\infty}p_{i,j}^{(n)}=\pi(j)=\frac{1}{\mu_{j,j}}.
\]
\end{theorem}
The aperiodic condition is not essential for the equivalence, but for the uniqueness of stationary equation.
\begin{proof}
\begin{itemize}
\item
Assume 1) holds but 2) does not.
Then the Markov chain is transient or null recurrent by the irreducibility.
By Theorem~\ref{The:9:3} and Theorem~\ref{The:9:4}, 
$p_{i,j}^{(n)}\to0$ as $n\to\infty$.
Then by the dominated convergence theorem,
\[
\pi(j)=\lim_{n\to\infty}\sum_{i\in S}\pi(i)p_{i,j}^{(n)}=\sum_{i\in S}\pi(i)\lim_{n\to\infty}p_{i,j}^{(n)}=0.
\]
This contradicts to 1).
\item
Assume 2) holds, then by Theorem~\ref{The:9:4}, $p_{i,j}^{(n)}\to\frac{1}{\mu_{j,j}}$.
We construct the stationary measure as $\pi(j)=\frac{1}{\mu_{j,j}}$.
Then by Fatou's lemma,
\[
\sum_{j\in S}\frac{1}{\mu_{j,j}}=\sum_{j\in S}\lim_{n\to\infty}p_{i,j}^{(n)}\le
\liminf_{n\to\infty}\sum_{j\in S}p_{i,j}^{(n)}=1.
\]
Moreover, by $p_{i,k}^{(n+1)}=\sum_{j\in S}p_{i,j}^{(n)}p_{j,k}$ we have
\[
\pi(k)=\liminf_{n\to\infty}p_{i,k}^{(n+1)}=\liminf_{n\to\infty}\sum_{j\in S}p_{i,j}^{(n)}p_{j,k}
\ge
\sum_{j\in S}\liminf_{n\to\infty}p_{i,j}^{(n)}p_{j,k}
=
\sum_{j\in S}\pi(j)p_{j,k}.
\]
Summing up both sides for $k\in S$,
\[
\sum_{k\in S}\pi(k)\ge \sum_{k\in S}\sum_{j\in S}\pi(j)p_{j,k}=\sum_{j\in S}\pi(j).
\]
Hence the inequaality must be equality, i.e., the statarion equation holds.
\item
Finally, we show that the stationary measure $\pi$ is unique.
Let $\nu$ be an arbitrary stationary distribution, then $\nu(j)=\sum_{i\in S}\nu(i)p_{i,j}^{(n)}$.
Taking $n\to\infty$ gives
\[
\nu(j)=\sum_{i\in S}\nu(i)\lim_{n\to\infty}p_{i,j}^{(n)}=\sum_{i\in S}\nu(i)\frac{1}{\mu_{j,j}}=\frac{1}{\mu_{j,j}}.
\]
The proof is completed.
\end{itemize}
\end{proof}
\begin{example}
The stationary equation for birth and death process is
\[
\left\{
\begin{aligned}
\pi(0)&=q\pi(0)+q\pi(1)\\
\pi(i)&=p\pi(i-1)+q\pi(i+1),\quad i\ge1.
\end{aligned}
\right.
\]
From the second equation, $q\pi(i)-p\pi(i-1)=q\pi(i+1)-p\pi(i)$, which implies
\[
q\pi(i)-p\pi(i-1)=q\pi(1)-p\pi(0)=0.
\]
Take $\rho=p/q$, then
\[
\pi(j)=\rho \pi(j-1)=\cdots=\rho^j\pi(0).
\]
Thus $\pi$ is a stationary distribution iff $\rho<1$.
In this case, $\pi(j)=(1-\rho)\rho^j$.
\end{example}

Then we study the stationary distribution for null recurrent case:
\begin{theorem}
Assume that $\{X_n\}$ is irreducible and recurrent.
\begin{enumerate}
\item
For each fixed $k\in S$, define
\[
\nu_k(j)=\mathbb{E}\left(
\sum_{\ell=1}^\infty1(X_{\ell}=j, \ell\le \tau_k)\middle| X_0=k
\right),\quad j\in S,
\]
then $\nu_k$ is a stationary measure.
\item
A stationary measure is unique up to constant multiple of $\nu_k$
\item
If it is positive recurrent, then the stationary distribution is defined as
\[
\pi(j)=\frac{1}{\mu_{k,k}}\nu_k(j).
\]
\end{enumerate}
\end{theorem}
\begin{remark}
$\sum_{\ell=1}^\infty1(X_{\ell}=j, \ell\le \tau_k)$ denotes the number of visits to state $j$ before returning to $k$.
The collection of its expectations is called an \emph{occupation measure} before returning to $k$.
\end{remark}

\begin{proof}
\begin{enumerate}
\item
Assume that $\nu_k(j)<\infty$, then we start to show the stationary equation:
\begin{align*}
\nu_k(j)&=\mathbb{E}\left(
\sum_{\ell=1}^\infty1(X_{\ell}=j, \ell\le \tau_k)\middle| X_0=k
\right)\\
&=p_{k,j}+\sum_{\ell=2}^\infty\sum_{i\in S\setminus k}
\mathbb{P}
\left(
X_{\ell}=j, \ell\le \tau_k, X_{\ell-1}=i\middle| X_0=k
\right)\\
&=p_{k,j}+\sum_{\ell=2}^\infty\sum_{i\in S\setminus k}
\mathbb{P}
\left(
X_{\ell-1}=i, \ell\le \tau_k\middle| X_0=k
\right)\\
&\qquad\qquad\times\mathbb{P}
\left(
X_{\ell}=j\middle| X_0=k,X_{\ell-1}=i, \ell\le \tau_k
\right)\\
&=p_{k,j}+\sum_{\ell=1}^\infty\sum_{i\in S\setminus k}\mathbb{P}
\left(
X_{\ell}=i, \ell\le \tau_k-1\middle| X_0=k
\right)p_{i,j}
\end{align*}
Since $\sum_{i\in S\setminus k}\mathbb{P}
\left(
X_{\ell}=i, \ell= \tau_k\middle| X_0=k
\right)=0$, we have 
\[
\nu_k(j)=\sum_{\ell=1}^\infty\sum_{i\in S}\mathbb{P}
\left(
X_{\ell}=i, \ell\le \tau_k\middle| X_0=k
\right)p_{i,j}=\sum_{i\in S}\nu_k(i)p_{i,j}.
\]
Next, we show that $\nu_k(j)<\infty,\forall j\in S$.
We can see that $\nu_k(k)=1$.
By the stationary equation and substitute $j$ with $k$, 
we can see $\nu_k(i)<\infty$ when $p_{i,j}>0$.
Furthermoe, by irreducibility result, there exists $i_0=j, i_1,\ldots,i_{\ell}=k$ so that $p_{i_m, i_{m+1}}>0$, and thus $\nu_k(i_m)>0$ for $m=\ell-1,\ldots,0$.
\item
Let $\eta$ be an arbitrary measure, then
\begin{align*}
\eta(j)&=\sum_{i\in S}\eta(i)p_{i,j}=\eta(k)p_{k,j}+\sum_{i\in S\setminus k}\eta(i)p_{i,j}\\
&=\eta(k)p_{k,j}+\sum_{i\in S\setminus k}\eta(k)p_{k,i}p_{i,j}+\sum_{i_1\in S\setminus k, i_2\in S\setminus k}\eta(i_2)p_{i_2,i_1}p_{i_1,j}\\
&=\eta(k)\sum_{\ell=1}^n\mathbb{P}(\ell<\tau_k, X_{\ell}=j\mid X_0=k)
+\sum_{i_n\in S\setminus k}\mathbb{P}(n<\tau_k, X_{n+1}=j\mid X_0=i_n)\eta(i_n)
\end{align*}
Taking $n\to\infty$ gives $\eta(j)\ge \eta(k)\nu_k(j)$.
Hence
\[
\eta(k)=
\sum_{j\in S}\eta(j)p_{j,k}\ge \sum_{j\in S}\eta(k)\nu_k(j)p_{j,k}=\eta(k)\nu_k(k)=\eta(k).
\]
The inequality must be inequality.
\item
3) is obvious since $\sum_{j\in S}\nu_k(j)=\mathbb{E}[\tau_k\mid X_0=k]=\mu_{k,k}$.
\end{enumerate}
\end{proof}


\begin{theorem}
Let $X_{\cdot}$ be an irreducible and positive recurrent Markov chain, and $\pi$ be the stationary distribution.
Then for function $f:~S\to\mathbb{R}_+$, 
\[
\lim_{n\to\infty}\frac{1}{n}\sum_{\ell=1}^nf(X_{\ell})=\sum_{j\in S}f(j)\pi(j)
\]
The RHS can be denoted as $\mathbb{E}_{\pi}[f(X)]$, which can be infinite.
\end{theorem}
This formula means that the time average equals the state space average, i.e., \emph{ergodic}.

\begin{proof}
We show this result under the conditional distribution given $X_0=i$.
Take $\tau_i(0)=0$ and $\tau_i(n)=\inf\{\ell>\tau_i(n-1):~X_{\ell}=i\}$.
By strong Markov property, $T_i(n)=\tau_i(n)-\tau_{i-1}(n)$ are i.i.d. random variables.
By positive recurrence, $\mu_{ii}=\mathbb{E}[T_i(n)\mid X_0=i]<\infty$.
Take $Y_n=\sum_{\ell=\tau_i(n-1)+1}^{\tau_i(n)}f(X_{\ell})$.
\begin{itemize}
\item
We first consider $\mathbb{E}[Y_1]<\infty$, then $\frac{1}{n}\sum_{\ell=1}^mY_{\ell}\to\mathbb{E}[Y_1]$.
There exists $N_n$ so that $\tau_i(N_n)\le n<\tau_i(N_{n+1})$. By LLN,
\[
\lim_{n\to\infty}\frac{N_n}{n}=\lim_{n\to\infty}\frac{N_n}{\tau_i(N_n)}=\frac{1}{\mu_{ii}}
\]
On the other hand, 
\[
\frac{N_n}{n}\frac{1}{N_n}\sum_{m=1}^{N_n}Y_{m}\le 
\frac{1}{n}\sum_{\ell=1}^nf(X_{\ell})
\le \frac{N_{n+1}}{n}\frac{1}{N_{n+1}}\sum_{m=1}^{N_{n+1}}Y_{m}
\]
Taking $n\to\infty$ gives $\frac{1}{n}\sum_{\ell=1}^nf(X_{\ell}\to\frac{1}{\mu_{ii}}\mathbb{E}[Y_1]$.
It suffices to show that 
\[
\frac{1}{\mu_{ii}}\mathbb{E}[Y_1]=\sum_{j\in S}f(j)\pi(j).
\]
The detailed computation is the following:
\begin{align*}
\frac{1}{\mu_{ii}}\mathbb{E}[Y_1]&=\frac{1}{\mu_{ii}}\mathbb{E}\left[
\sum_{\ell=1}^{\tau_i}f(X_{\ell})\middle|X_0=i
\right]\\
&=\frac{1}{\mu_{ii}}\mathbb{E}\left[
\sum_{\ell=1}^{\infty}f(X_{\ell})1(\ell\le\tau_i)\middle|X_0=i
\right]\\
&=\frac{1}{\mu_{ii}}\mathbb{E}\left[
\sum_{\ell=1}^{\infty}\sum_{j\in S}f(X_{\ell})1(X_{\ell}=j, \ell\le\tau_i)\middle|X_0=i
\right]\\
&=\frac{1}{\mu_{ii}}\sum_{j\in S}f(X_{j})\mathbb{E}\left[
\sum_{\ell=1}^{\infty}1(X_{\ell}=j, \ell\le\tau_i)\middle|X_0=i
\right]\\
&=\sum_{j\in S}f(X_{j})\pi(j)
\end{align*}
\item
Now we consider $\mathbb{E}[Y_1]=\infty$.
Define the truncated function $f^{(k)}(i)=\min(k, f(i))$.
Thus $\lim_{n\to\infty}\frac{1}{n}\sum_{\ell=1}^nf^{(k)}(X_{\ell})=\sum_{j\in S}f^{(k)}(X_{j})\pi(j)$.
Moreover,
\[
\liminf_{n\to\infty}\frac{1}{n}\sum_{\ell=1}^nf(X_{\ell})\ge
\liminf_{n\to\infty}\frac{1}{n}\sum_{\ell=1}^nf^{(k)}(X_{\ell})=\sum_{j\in S}f^{(k)}(X_{j})\pi(j).
\]
By monotone convergence theorem, $\sum_{j\in S}f^{(k)}(X_{j})\pi(j)\to\mathbb{E}[Y_1]=\infty$.
The proof is completed.
\end{itemize}

\end{proof}

\subsection{Stability Conditions}
It is important to determine whether an irreducible Markov chain is positive recurrent or not,  since $p_{i,j}^{(n)}$ vanishes as $n\to\infty$ otherwise.
We call this problem as stability one since hte positive recurrence implies that the limiting state distribution is stationary.
We consider the problem in two cases:
\begin{itemize}
\item
$|S|<\infty$: all states are positive recurrent;
\item
$|S|=\infty$: if we can solve the stationary equation, then we can answer.
However, it is difficult to do so for general case.
\end{itemize}
Here we use a test function $f:~S\to\mathbb{R}$.
By the irreducibility, it suffices to find $\exists i\in S$ so that $\mu_{i,i}<\infty$ to prove the positive recurrence.
This means that the Markov chain returns to some finite set of states sufficiently fast,
which can be realized by a drift condition towards the set:

\begin{theorem}
Let $X_{\cdot}$ be a Markov chain that is irreducible and aperiodic.
If there is a test function $f:~S\to\mathbb{R}_+$ and a finite set $S_0\subseteq S$ so that
\begin{itemize}
\item
$\mathbb{E}[f(X_n)]$ is finite for all $n$;
\item
there exists $\varepsilon>0$ and $\forall n\ge0$,
\[
\mathbb{E}[f(X_{n+1})\mid X_n]-f(X_n)\le-\varepsilon,\quad X_n\notin S_0,
\]
then $X_{\cdot}$ is positive recurrent.
\end{itemize}
\end{theorem}

\begin{example}
Let $X_{\cdot}$ be a simple random walk on $\mathbb{Z}^2$.
We modify the random walk so that it is restricted in $\mathbb{Z}_+^2$.
Define the Lyapunov function
\[
f(\bm x)=\sqrt{ax_1^2+bx_2^2+cx_1x_2},\quad 4ab>c^2.
\]
Then by calculation,
\begin{align*}
f(\bm x+U(\bm x))&=
f(\bm x)\left(
1+\frac{f^2(U(\bm x)) + (2ax_1+cx_2)U_1(\bm x) + (2bx_2+cx_1)U_2(\bm x)}{f^2(\bm x)}
\right)^{1/2}\\
&=f(\bm x)\left(
1+\frac{f^2(U(\bm x)) + (2ax_1+cx_2)U_1(\bm x) + (2bx_2+cx_1)U_2(\bm x)}{2f^2(\bm x)}
\right)\\
&=f(\bm x) + \frac{f^2(U(\bm x)) + (2ax_1+cx_2)U_1(\bm x) + (2bx_2+cx_1)U_2(\bm x)}{2f(\bm x)}
\end{align*}
It follows that
\[
\mathbb{E}[f(\bm X_{n+1})\mid \bm X_n=\bm x]-f(\bm x)=
\frac{(2a\mu_1(\bm x) + c\mu_2(\bm x))x_1 + (2b\mu_2(\bm x) + c\mu_1(\bm x))x_2}{2f(\bm x)}
\]
\end{example}

\subsection{Markov Modulated Process in Discrete Time}
A random walk on a line has the following features:
\begin{itemize}
\item
Skip free in one-step movements;
\item
Independent increments.
\end{itemize}
A discrete time Markov chain relaxes these conditions, but too general as an extension.
Now we use the following ideas:
\begin{itemize}
\item
Keep the state space and the skip free movements of the original process, denoted as $X_{\cdot}$;
\item
Add a background state, described by a finite state Markov Chain, denoted as $J_{\cdot}$.
\end{itemize}

\begin{definition}[Markov modulated random walk]
We say $Y_{\cdot}$ is a \emph{Markov modulated random walk} with $Y_n=(X_n,J_n)\in S\equiv S_L\times S_B$ if the following conditions hold:
\begin{enumerate}
\item
$J_{\cdot}$ is a Markov chain with transition matrix $A\equiv\{a_{i,j}:~i,j\in S_B\}$, where $a_{i,j}$ can be decomposed as $a_{i,j}=a_{i,j}^- + a_{i,j}^0 + a_{i,j}^+$ for $a_{i,j}^- ,a_{i,j}^0,a_{i,j}^+\ge0$;
\item
Markov chain $Y_{\cdot}$ has the following transition probabilities:
\[
p_{(k,i), (\ell,j)}=\left\{
\begin{aligned}
a_{i,j}^-&\quad \ell=k-1\\
a_{i,j}^0&\quad \ell=k\\
a_{i,j}^+&\quad \ell=k+1\\
0&\quad\text{otherwise}
\end{aligned}
\right.
\]
\end{enumerate}
Here $X_{\cdot}$ is called a level process, and $J_{\cdot}$ is called a phase process.
The transition of phase $J_n$ to $J_{n+1}$ controls the level increment $\Delta X_n=X_{n+1}-X_n$, i.e., $J_{\cdot}$ modulates the skip free process $X_{\cdot}$.
\end{definition}
The most simple case is $S_B=\{1\}$, and $a^+_{1,1}=p, a^0_{1,1}=q, a^-_{1,1}=r$ with $p+q+r=1$.
Consider the occupation measure $\nu^+$ on $S_L^+=\{\ell\in S_L:~\ell\ge0\}$ before returning to $\{k\in S_L:~k\le0\}$.
Now we have
\[
\nu^+(\ell)=\left\{
\begin{aligned}
(q+r)\cdot\nu^+(0) + r\cdot \nu^+(1),&\quad \ell=0\\
p\cdot\nu^+(\ell-1) + q\cdot\nu^+(\ell) + r\cdot\nu^+(\ell+1),&\quad \ell\ge1.
\end{aligned}
\right.
\]
Then we have $\nu^+(\ell)=(p/r)^{\ell}\nu^+(0)$ for $\ell\ge1$.
We are interested in the following questions:
\begin{enumerate}
\item
The occupation measure of a Markov modulated random walk has a similar geometric form?
\item
How such a result can be used in application?
\end{enumerate}

\begin{definition}
Let $\tau_{0-}=\inf\{n\ge1:~X_n-X_0\le0\}$ and $D_b(S)$ be the set of all bounded functions from $S$ to $\mathbb{R}$.
Define $S_B\times S_B$ matrices $R^{(\ell)}$ for $\ell\ge1$ as
\[
R_{i,j}^{(\ell)}=\mathbb{E}\left(
\sum_{n=1}^{\tau_{0-}-1}1(X_n = \ell, J_n=j)\bigg| Y_0=(0,i)
\right),\quad i,j\in S_B
\]
Here $\{R_{i,j}^{(\ell)}:~j\in S_B\}$ is the occupation measure at $\ell\ge1$
before returning to the set $A=\{(n,k)\in S:~n\le0\}$.
\end{definition}

\begin{proposition}
The matrix $R^{(\ell)}$ admits the following product form, while $R$ is not a probability matrix in general:
\[
R^{(\ell)} = R^{\ell},\qquad \ell\ge1.
\]
\end{proposition}
\begin{proof}
\begin{align*}
R^{(\ell+1)}_{i,j}&=\sum_{n=1}^\infty 
\mathbb{P}\left(
Y_n=(\ell+1,j), n\le \tau_{0-}-1\bigg| Y_0=(0,i)
\right)\\
&=\sum_{n=1}^\infty\sum_{m=1}^{n-1}\sum_{i'\in S_B}
\mathbb{P}\left(
Y_m=(\ell,i'), X_k>\ell, k=m+1,\ldots,n,
Y_n=(\ell+1,j), n\le \tau_{0-}-1\bigg| Y_0=(0,i)
\right)\\
&=\sum_{i'\in S_B}\sum_{m=1}^{\infty}
\mathbb{P}\left(
Y_m=(\ell,i'), m\le \tau_{0-}-1\bigg| Y_0=(0,i)
\right)\\&\qquad\qquad\qquad\times \sum_{n=m+1}^\infty \mathbb{P}\left(Y_n=(\ell+1,j), n\le \tau_{0-}-1\bigg| Y_m=(\ell,i')\right)\\
&=\sum_{i'\in S_B}R_{i,i'}^{(\ell)}R_{i',j}.
\end{align*}
\end{proof}

\begin{proposition}
The rate matrix $R$ is a minimal non-negative matrix solution to the equation
\[
T=A^+ + TA^0 + T^2A^-.
\]
\end{proposition}
\begin{proof}
We first show that $R^{(1)}$ satisfies the stationary equation:
\begin{align*}
R_{i,j}&=\sum_{n=1}^\infty\mathbb{P}\left(
Y_n=(1,j), n<\tau_{0-}\middle| Y_0=(0,i)
\right)\\
&=\mathbb{P}\left(
Y_1=(1,j)\middle| Y_0=(0,i)
\right)+\sum_{n=2}^\infty\mathbb{P}\left(
Y_n=(1,j), n<\tau_{0-}\middle| Y_0=(0,i)
\right)\\
&=a_{i,j}^++\sum_{i'\in S_B}\sum_{n=2}^\infty
\mathbb{P}\bigg(
Y_{n-1}=(1,i'),
Y_n=(1,j), n<\tau_{0-}\bigg| Y_0=(0,i)
\bigg)\\
&+\sum_{i'\in S_B}\sum_{n=2}^\infty
\mathbb{P}\bigg(
Y_{n-1}=(2,i'),
Y_n=(1,j), n<\tau_{0-}\bigg| Y_0=(0,i)
\bigg)
\end{align*}
Then we compute the last two double summations:
\begin{align*}
&\mathbb{P}\bigg(
Y_{n-1}=(1,i'),
Y_n=(1,j), n<\tau_{0-}\bigg| Y_0=(0,i)
\bigg)\\
=&
\mathbb{P}\bigg(
Y_{n-1}=(1,i'),
n-1<\tau_{0-}\bigg| Y_0=(0,i)
\bigg)\\&\qquad\qquad
\mathbb{P}\bigg(
Y_n=(1,j), n<\tau_{0-}\bigg| Y_0=(0,i), Y_{n-1}=(1,i'), n-1<\tau_{0-}
\bigg)\\
&=\mathbb{P}\bigg(
Y_{n-1}=(1,i'),
n-1<\tau_{0-}\bigg| Y_0=(0,i)
\bigg)a_{i',j}^0.
\end{align*}
Similarly, 
\begin{align*}
&\mathbb{P}\bigg(
Y_{n-1}=(2,i'),
Y_n=(1,j), n<\tau_{0-}\bigg| Y_0=(0,i)
\bigg)\\
&=
\mathbb{P}\bigg(
Y_{n-1}=(2,i'),
n-1<\tau_{0-}\bigg| Y_0=(0,i)
\bigg)a_{i',j}^-.
\end{align*}
Therefore, $R=A^+ + RA^0 + R^2A^-$.

Next, we show that $R$ is a minimal non-negative solution.
Let $T^{(0)}=O$ and inductively define $T^{(n)}$ by
\[
T^{(n)}=A^+ + T^{(n-1)}A^0 + (T^{(n-1)})^2A^-
\]
Note that $T^{(n)}$ satisfies 
\[
(T^{(n)} - T^{(n-1)}) = (T^{(n-1)} - T^{(n-2)}) A^0 + ((T^{(n-1)})^2 - (T^{(n-2)})^2)A^-.
\]
Hence, $T^{(\infty)}$ exists and it must be the minimal solution since $T^{(0)}=O$,
In particular, $T^{(\infty)}\le R$. It suffices to show that $R\le T^{(\infty)}$.
Define $R(n)$ by
\[
[R(n)]_{i,j}=\sum_{m=1}^\infty\mathbb{P}\bigg(
Y_m=(1,j),~m<\tau_{0-}\land n\bigg| Y_0=(0,i)
\bigg).
\]
We can show that $T^{(n-1)}=R(n)$ since $T^{(0)}=O=R(1)$ and $R(n)$ satisfies the inductive equation
\[
R(n) =A^+ + R(n-1)A^0 + (R(n-1))^2A^-
\]
\end{proof}

\begin{definition}[Markov modulated reflected simple random walk]
We put a reflecting boundary at state $0\in S_L$ in such a way that
\[
\tilde{P}=\begin{pmatrix}
B^{00}&B^{01}&0&0&0&\cdots\\
B^{10}&A^0&A^+&0&0&\cdots\\
0&A^-&A^0&A^+&0&\vdots\\
0&0&A^-&A^0&A^+&\vdots\\
0&0&0&A^-&A^0&\ddots\\
\vdots&\vdots&\vdots&\vdots&\ddots&\ddots
\end{pmatrix}
\]
The modified process $\tilde{Y}_{\cdot}=(\tilde{X}_{\cdot}, \tilde{J}_{\cdot})$ is called a Markov modulated reflected simple random walk.
\end{definition}

Let $\pi$ be the unique stationary distribution for transition matrix $A$, and denote the mean drift of the random walk $Y_{\cdot}$ as
\[
\gamma = -\pi A^- + \pi A^+.
\]
It is easy to see $\tilde{Y}_n$ has a stationary distribution iff $\gamma<0$.
Denote the stationary distribution as $\tilde{\nu}$, characterized by $\tilde{\nu}=\tilde{\nu}\tilde{P}$.
Partition $\tilde{\nu}$ by level and $\tilde{\nu}_n=\mathbb{P}(\tilde{X}_\infty=n)$.
Then we have the stationary equation
\begin{align*}
\tilde{\nu}_0&=\tilde{\nu}_0B^{00} + \tilde{\nu}_1B^{10}\\
\tilde{\nu}_1&=\tilde{\nu}_0B^{01} + \tilde{\nu}_1A^0 + \tilde{\nu}_2A^-\\
\tilde{\nu}_n&=\tilde{\nu}_{n-1}A^{-} + \tilde{\nu}_nA^0 + \tilde{\nu}_{n+1}A^-,\quad n\ge2
\end{align*}
From the third equation we have $\tilde{\nu}_n = \tilde{\nu}_1R^{n-1}$. From the first two equation we have
\[
\left\{
\begin{aligned}
\tilde{\nu}_0&=\tilde{\nu}_1B^{10}(I - B^{00})^{-1}\\
\tilde{\nu}_1&=\tilde{\nu}_0B^{01}(I - A^0 - RA^{-})^{-1}
\end{aligned}
\right.
\]

Here we ask the following questions:
\begin{itemize}
\item
$\tilde{\nu}_n$ has a nice expression. Is it possible to use it for asymptotic analysis?
\item
Can we apply this method to a queueing network?
\end{itemize}
The first question is yes.
For the second question, the corresponding random walk is multi-dimensional, though we only discuss for the one-dimensional case. 
When $S_L, S_B$ are uncountable, the matrices need to be in the function space, which is a big paradigm beyond the present approach.









