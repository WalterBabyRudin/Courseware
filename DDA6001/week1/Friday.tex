% !TEX encoding = UTF-8 Unicode

\section{Thursday}\index{Thursday_lecture}
\paragraph{Reviewing}
Probability space $(\Omega,\mathcal{F},\mathbb{P})$, where
\begin{itemize}
\item
$\Omega$: collection of data;
\item
$\mathcal{F}$: collection of events;
\item
$\mathbb{P}$: probability measure $\mathcal{F}\to[0,1]$ satisfying:
\begin{itemize}
\item
$\mathbb{P}(\Omega)=1$;
\item
$A_i\in\mathcal{F}, A_i\cap A_j=\emptyset$, then
\[
\mathbb{P}\bigg(
\bigcup_{i=1}^\infty A_i
\bigg)
=
\sum_{i=1}^\infty\mathbb{P}(A_i)
\]
\end{itemize}
\end{itemize}

From observation to number: $\Omega\ni\omega\to X(\omega)\in\mathbb{R}$.
Therefore, $X$ should be measurable, i.e.,
\[
\{\omega\in\Omega; X(\omega)\in B\}:=\{X\in B\}:=X^{-1}(B)\in\mathcal{F}, ~~\forall B\in\mathcal{B}(\mathbb{R}).
\]
If $X$ is measurable, we call $X$ as a random variable.
We can also show that
\[
\{\omega\in\Omega; X(\omega)\in B\}\in\mathcal{F}\Longleftrightarrow
\{X\le a\}\in\mathcal{F}
\]


\subsection{Representative value of random variable $X$}
We define the random variable by the following three steps:
\begin{itemize}
\item
Consider first that $X$ is simple, i.e., taking finite many values.
Assume that $X$ only takes $x_1,x_2,\ldots,x_n\in\mathbb{R}$.
\[
\mathbb{E}[X]:=\sum_{i=1}^m x_i\mathbb{P}(X=x_i),
\]
with $\mathbb{P}(X = x_i) = \mathbb{P}(\{X=x_i\})=\mathbb{P}\{\omega\in\Omega; X(\omega)=x_i\}$.
\item
Then consider the case where $X$ is non-negative, which is approximated by simple random variables.
For each $n\ge1$, 
\[
X_n(\omega) = \left\{
\begin{aligned}
(i-1)2^{-n},&\quad\text{if $(i-1)2^{-n}\le X(\omega)<i2^{-n}, i=1,2,\dots,n2^n+1$}\\
n,&\quad\text{if $X(\omega)\ge n$}
\end{aligned}
\right.
\]
Taking $n\to\infty$, $X_n(\omega)\uparrow X(\omega)$.
Define $\mathbb{E}[X] = \lim_{n\to\infty}\mathbb{E}[X_n]$.
\item
Consider finally $X$ is in the whole real line.
\begin{align*}
X_+(\omega) = \max\{0, X(\omega)\}\ge0
\end{align*}
\end{itemize}


\begin{example}
Consider the news vender problem:
\end{example}



The random variable $X$ may take other types of values than $(-\infty,\infty)$, such as vector values.
\[
X:~~\omega\in\Omega\to X(\omega)\in S,
\]
with $S$ being a topological space.
A speical case is the metric space.
The advantage of topological space is that we can use it to define borel set conveniently:
\[
\mathcal{B}(S):=\sigma(\{\mbox{topology of $S$}\})
\]
$X$ is $S$-valued random variable if $X:\Omega\to S$ and $\{X\in B\}\in\mathcal{F}, \forall B\in\mathcal{B}(S)$.

It may be difficult to evaluate the expectation of a random variable, and therefore we pick the test function
\[
f: S\to\mathbb{R}
\]
satisfying measurability:
\[
f^{-1}(B')\in\mathcal{B}(S), \forall B'\in\mathcal{B}(\mathbb{R})
\]
Define $f(X)(\omega) = f(X(\omega)), \forall \omega\in\Omega$:
\[
\Omega\xrightarrow{X} S\xrightarrow{f}\mathbb{R}.
\]

We use a family of test functions, denoted by $\mathcal{C}$.
If we take a sufficiently large $\mathcal{C}$, we can determine the distribution of $X$.
\begin{itemize}
\item
When $S=\mathbb{R}$,
\[
\mathcal{C} = \{f_{\theta}:~~f_{\theta}(x) = e^{\theta x}, \theta\in\mathbb{R}\}
\]
and $\mathbb{E}[f_{\theta}(x)]$ is the moment generating function.

\end{itemize}







