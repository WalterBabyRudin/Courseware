
\chapter{Week 7}

\section{Monday}\index{Monday_lecture}
\subsection{Stopping Time and Martingale}


\begin{definition}[Discrete Time Stochastic Basis]
For probability space $(\Omega,\mathcal{F},\mathbb{P})$ and filtration $\mathbb{F}$ on $(\Omega,\mathcal{F})$, the set $(\Omega,\mathcal{F},\mathbb{F},\mathbb{P})$ is called a \emph{stochastic basis} (or a discrete-time filtered probability space).
\end{definition}

Consider to bet money by fair coin tossing. 
We are interested when to stop this coin tossing to maximize the assert,
based on all outcomes up to the time.
We define this time as a random variable as the following.

\begin{definition}[Stopping Time]
Let $\mathbb{F}\triangleq \{\mathcal{F}_n:~n\ge0\}$ be a filtration.
A random variable $\tau$ valued in $\overline{\mathbb{Z}}_+\equiv\mathbb{Z}_+\cup\{\infty\}$
is called a stopping time if $\{\tau\le n\}\in\mathcal{F}_n, \forall n\ge0$.
\end{definition}
\begin{remark}
$\{\tau\le n\}\in\mathcal{F}_n, \forall n\ge0$ if and only if $\{\tau= n\}\in\mathcal{F}_n, \forall n\ge0$
\end{remark}

\begin{definition}[Stopped $\sigma$-field]
For filtration $\mathbb{F}\triangleq \{\mathcal{F}_n:~n\ge0\}$ and stopping time $\tau$,
\[
\mathcal{F}_\tau \triangleq\{A\in\mathcal{F}:~A\cap\{\tau\le n\}\in\mathcal{F}_n,\forall n\ge0\}
\]
is called a $\sigma$-field stopped by $\tau$.
\end{definition}


\begin{example}
Let $X_{\cdot}\triangleq \{X_n:~n\ge0\}$ with $X_0=0$ be a real valued process.
Let $\mathcal{F}_n=\sigma(X_0,X_1,\ldots,X_n)$ and $\mathbb{F}\triangleq \{\mathcal{F}_n:~n\ge0\}$.
\begin{itemize}
\item
Define $\tau_a$ as 
\[
\tau_a = \inf\{\ell\ge0:~X_\ell\ge a\},\quad a>0
\]
and $\inf\emptyset=+\infty$.
Then $\tau_a$ is a stopping time since 
\[
\tau_a=n\Longleftrightarrow
\max_{0\le \ell\le n-1}X_\ell<a\le X_n.
\]
\item
Define $\tau_{\max}$ as 
\[
\tau_{\max}=\inf\{\ell\ge0:~\max_{k\ge0}X_k=X_\ell\}.
\]
Then $\tau_{\max}$ is not a stopping time since 
\[
\tau_{\max}=n
\Longleftrightarrow
X_\ell<X_n, \ell\le n-1,
X_\ell\le X_n, \ell\ge n+1.
\]
\end{itemize}
\end{example}
\begin{remark}
If $\tau,\eta$ are stopping times, then
\[
\tau+\eta,\quad \tau\land \eta,\quad \tau\lor\eta
\]
are stopping times, but $0\lor(\tau-\eta)$ may not be.
\end{remark}

\begin{proposition}\label{Pro:7:1}
Let $X_{\cdot}\triangleq \{X_n:~n\ge0\}$ be a martingale (sub or super-martingale).
If $\tau$ is a stopping time, then $\{X_{\tau\land n}:~n\ge0\}$ is an $\mathbb{F}$-martingale (sub or super-martingale).
\end{proposition}
\begin{proof}
We prove the case where $X_{\cdot}$ is a sub-martingale.
\begin{itemize}
\item
Write $X_{\tau\land n}$ as
\[
X_{\tau\land n}=X_0 + \sum_{\ell=1}^{\tau\land n}(X_{\ell} - X_{\ell-1})
=X_0 + \sum_{\ell=1}^{n}1(\ell\le \tau)(X_{\ell} - X_{\ell-1})
\]
Since $1(\ell\le \tau)$ is $\mathcal{F}_n$-measurable for $\ell=1,\ldots,n$, we imply that 
$X_{\tau\land n}$ is also $\mathcal{F}_n$-measurable.
\item
Moreover, 
\[
X_{\tau\land(n+1)}-X_{\tau\land n}=1(n+1\le \tau)(X_{n+1} - X_{n}).
\]
which implies that
\[
\mathbb{E}[X_{\tau\land(n+1)}\mid\mathcal{F}_n] - X_{\tau\land n}
=1(n+1\le \tau)(\mathbb{E}[X_{n+1}\mid\mathcal{F}_n] - X_{n})\ge0.
\]
%which implies that 
%\[
%X_{\tau\land(n+1)}-X_{\tau\land n}=1(n+1\le \tau)(X_{n+1} - X_{n}).
%\]
\end{itemize}
\end{proof}

We can define a stopped process by a stopping time:
\[
X_{\cdot}^\tau\triangleq \{X_n^\tau:~n\ge0\},\quad
\text{where }X_n^\tau=X_{\tau\land n}.
\]
Then $X_{\cdot}^\tau$ is called a $\tau$-stopped process.
It is adapted to $\mathbb{F}$ as well as to $\mathbb{F}^\tau\triangleq\{\mathcal{F}_{\tau\land n}:~n\ge0\}$, if $X$ is adapted to $\mathbb{F}$.

For a sequence of finite stopping times $\{\tau_n:~n\ge1\}$ such that $\tau_n\to\infty$ as $n\to\infty$,
a sequnece of stopped processes $X_{\cdot}^{\tau_n}\triangleq\{X_\ell^{\tau_n}:~\ell\ge0\}$ for $n=1,2,\ldots$ is called a \emph{localization} of $X_{\cdot}$.
Since $X_\ell^{\tau_n}\to X_{\ell}$ a.s. as $n\to\infty$, it may be possible to consider $X_\ell^{\tau_n}$ instead of $X_\ell$.
This localization does not much earn for a discrete-time process, but it does for a continuous-time process as we will see.

\subsection{Optional Sampling Theorem}

\begin{theorem}[Optional Inequality 1]\label{The:7:1}
Suppose that $X_{\cdot}$ is a submartingale and $\tau$ is a stopping time taking values in 
$\{0,1,\ldots,m\}$ for a positive integer $m$, then we have
\[
\mathbb{E}[X_0]\le \mathbb{E}[X_\tau]\le \mathbb{E}[X_m]
\]
\end{theorem}

\begin{proof}
By proposition~\ref{Pro:7:1}, $\{X_{\tau\land n}:~n\ge0\}$ is a submartingale, which implies 
\[
X_{\tau\land 0}\le \mathbb{E}[X_{\tau\land m}\mid\mathcal{F}_0]\implies
\mathbb{E}[X_0]=\mathbb{E}[X_{\tau\land 0}]\le \mathbb{E}[X_{\tau\land m}]= \mathbb{E}[X_{\tau}].
\]
Moreover, $\{X_n - X_{\tau\land n}:~n\ge0\}$ is a submartingale:
\begin{align*}
\mathbb{E}[X_n - X_{\tau\land n}\mid\mathcal{F}_{n-1}]
&=\mathbb{E}[X_n - (X_\tau1(\tau\le n-1) + X_n1(\tau\ge n))\mid\mathcal{F}_{n-1}]\\
&=\mathbb{E}[X_n1(\tau\le n-1) - X_\tau1(\tau\le n-1)\mid\mathcal{F}_{n-1}]\\
&=\mathbb{E}[X_n\mid\mathcal{F}_{n-1}]1(\tau\le n-1) -  X_\tau1(\tau\le n-1)\\
&\ge(X_{n-1}-X_\tau)1(\tau\le n-1)\\
&=X_{n-1} - X_{\tau\land (n-1)}
\end{align*}
Therefore,
\[
\mathbb{E}[X_m - X_\tau] = \mathbb{E}[X_m - X_{\tau\land m}]\ge \mathbb{E}[X_0 - X_{\tau\land 0}]=0.
\]

\end{proof}

\begin{corollary}\label{Cor:7:1}
Suppose that  $X_{\cdot}$ is a submartingale and $\tau,\eta$ are stopping times such that $\tau\le\eta\le m$ for a constant time $m$, then we have
\[
\mathbb{E}[X_0]\le \mathbb{E}[X_\tau]\le \mathbb{E}[X_\eta]\le\mathbb{E}[X_m]
\]
\end{corollary}

\begin{proof}
Take $Y_n = X_{\eta\land n}$, then $Y_n$ is a sub-martingale.
Applying Theorem~\ref{The:7:1} into the stopping time $\tau$ and $Y_{\cdot}$ gives 
\[
\mathbb{E}[Y_{\tau}]\le \mathbb{E}[Y_{m}]=\mathbb{E}[X_\eta]
\]
In other words, $\mathbb{E}[X_{\tau}]\le \mathbb{E}[X_\eta]$.
\end{proof}


\begin{remark}
In Theorem~\ref{The:7:1} and Corollary~\ref{Cor:7:1},
``$\le$'' is replaced by ``$=$'' or ``$\ge$''
if
$X_{\cdot}$ is a martingale or super-martingale.
\end{remark}

\begin{theorem}[Optional Sampling Theorem]
If $X_{\cdot}$ is a sub-martingale and $\tau,\eta$ are stopping times such that $\tau\le\eta\le m$ for a constant time $m$, then
\[
X_\tau\le \mathbb{E}[X_\eta\mid\mathcal{F}_\tau]
\]
almost surely.
Here $\ge$ is replaced by $=$ if $X_{\cdot}$ is a martingale.
\end{theorem}
\begin{proof}
For any $A\in\mathcal{F}_\tau$, define
\[
\zeta = \tau1(A) + \eta1(A^c).
\]
Then $\zeta$ is a stopping time with $\tau\le\zeta\le\eta\le m$.
Applying Corollary~\ref{Cor:7:1} gives 
\[
\mathbb{E}[X_\tau1_A]+\mathbb{E}[X_\eta1_{A^c}] = \mathbb{E}[\zeta]\le  \mathbb{E}[X_\eta].
\]
As a result,
\[
\mathbb{E}[X_\tau1_A]\le \mathbb{E}[X_\eta1_A]=\mathbb{E}[\mathbb{E}[X_\eta\mid\mathcal{F}_\tau]1_A],\quad A\in\mathcal{F}_\tau.
\]
This means that $\mathbb{P}(X_\tau\le \mathbb{E}[X_\eta\mid\mathcal{F}_\tau])=1$.
\end{proof}

\begin{example}[Bet by Coin Tossing with initial asset $a$]
Let $U_n$'s be i.i.d. with $\mathbb{P}(U_n=1)=\mathbb{P}(U_n=-1)=\frac{1}{2}$.
Define $X_0=a$ and $X_n=X_0+U_1+\cdots+U_n$ for $n\ge1$.

Let $M_n = X_n-X_0$, then $M_{\cdot}=\{M_n:~n\ge0\}$ is a martingale for the filtration $\mathbb{F}^X$ since $\mathbb{E}[U_n]=0$.
We are interetsed in the probability that $X_n$ hits $b$ before to hit $0$.
Define stopping times
\[
\tau_0=\inf\{n\ge1:~X_n=0\},\quad
\tau_b=\inf\{n\ge1:~X_n=b\}.
\]
Then we aim to compute $\mathbb{P}(\tau_0<\tau_b)$.
Apply Theorem~\ref{The:7:1} for $\tau_n:=\tau_0\land\tau_b\land n$ gives
\[
\mathbb{E}[M_{\tau_n}] = \mathbb{E}[M_{0}]=0.
\]
Or equivalently,
\[
(b-a)\mathbb{P}(\tau_b<\tau_0\land n) - a\mathbb{P}(\tau_0<\tau_b\land n)
+\mathbb{E}[M_n1(n\le\tau_b\land \tau_0)]=0
\]
Considering that $\liminf M_n=-\infty,\limsup M_n=\infty$, by the law of iterated logarithms, we can assert that $\mathbb{P}(n\le\tau_b\land \tau_0)\to0$. Together with the fact that $|M_n|\le b$ under the event $\{n\le\tau_b\land \tau_0\}$, we imply
\[
(b-a)\mathbb{P}(\tau_b<\tau_0\land n) - a\mathbb{P}(\tau_0<\tau_b\land n)=0.
\]
On the other hand, $\mathbb{P}(\tau_b\lor \tau_0<\infty)=1$ gives
\[
\mathbb{P}(\tau_b<\tau_0)+\mathbb{P}(\tau_0<\tau_b)=1.
\]
Thus $\mathbb{P}(\tau_b<\tau_0)=b/a$.
\end{example}
This deviation has the following problems:
\begin{enumerate}
\item
It requires the extra result: the law of iterated logarithms;
\item
It cannot be applied for an unfair coin such that $\mathbb{P}(U_n=1)\ne\frac{1}{2}$.
\end{enumerate}
We will overcome these problems by considering another type of martingale, in the next week.

\subsection{Strong and Stopped Markov Processes}
Recall the definition of discrete Markov process:
\begin{enumerate}
\item
$X_{\cdot}$ is $\mathbb{F}$-adapted;
\item
$\mathbb{E}[f(X_{n+1})\mid\mathcal{F}_n] = P[f(X_n)]$
\end{enumerate}
For a stopping time $\tau$, we want to ask
\begin{enumerate}
\item[(a)]
Does the second condition still holds for $n=\tau$?
\item[(b)]
Is the stopped prcess $X_{\cdot}^{\tau}$ again Markov?
\end{enumerate}
We first introduce a notion for (a):
\begin{definition}[Strong Markov Property]
A discrete-time Markov process $X_{\cdot}$ is called \emph{strong Markov} if 
\[
\mathbb{E}[f(X_{\tau+1})\mid\mathcal{F}_\tau] = P[f(X_\tau)],\quad f\in D_b(S)
\]
for all finite stopping time $\tau$ (i.e., $\mathbb{P}(\tau<\infty)=1$).
\end{definition}

\begin{theorem}
A discrete-time Markov process $X_{\cdot}$ is strong Markov.
\end{theorem}
\begin{proof}
Let $\tau$ be an arbitrary finite stopping time.
Then the strong Markov property holds iff 
\[
\mathbb{E}[f(X_{\tau+1})1_A] = \mathbb{E}[P[f(X_\tau)]1_A],\quad A\in\mathcal{F}_\tau.
\]
Or equivalently, for any $n\ge1$,
\[
\mathbb{E}[f(X_{\tau+1})1_{A\cap\{\tau=n\}}] = \mathbb{E}[P[f(X_\tau)]1_{A\cap\{\tau=n\}}],\quad A\in\mathcal{F}_\tau.
\]
The RHS is equivalent to
\[
\mathbb{E}[\mathbb{E}[f(X_{n+1}\mid X_n)]1_{A\cap\{\tau=n\}}]
=
\mathbb{E}[\mathbb{E}[f(X_{n+1}\mid \mathcal{F}_n)]1_{A\cap\{\tau=n\}}]
=\mathbb{E}[f(X_{n+1})1_{A\cap\{\tau=n\}}].
\]
The proof is completed.
\end{proof}

In some applications, we start to observe a discrete-time Markov process $X_{\cdot}$ from time $\tau$, which is a finite stopping time.
In this case, is $X_{\tau+\cdot}=\{X_{\tau+n}:~n\ge0\}$ again Markov?
It must be. To see this, we need to show that 
\[
\mathbb{E}[f(X_{\tau+n+1})\mid\mathcal{F}_{\tau+n}] = \mathbb{E}[f(X_{\tau+n+1})\mid X_{\tau+n}]
\]
This is true since $\tau+n$ is a stopping time.

\begin{proposition}
For a discrete-time Markov process $X_{\cdot}$ and stopping time $\tau$, the stopped process 
$X_{\cdot}^{\tau}$ is $\mathbb{F}^{\tau}$-Markov, where
$\mathbb{F}^{\tau}=\{\mathcal{F}_{\tau\land n}:~n\ge0\}$.
\end{proposition}
\begin{proof}
It suffices to show that 
\[
\mathbb{E}[f(X_{\tau\land(n+1)})\mid\mathcal{F}_{\tau\land n}] = \mathbb{E}[f(X_{\tau\land(n+1)})\mid X_{\tau\land n}]
\]
By definition of conditional expectation, it suffices to show that
\begin{equation}\label{Eq:7:1}
\mathbb{E}[f(X_{\tau\land(n+1)})1_A] = \mathbb{E}[\mathbb{E}[f(X_{\tau\land(n+1)})\mid X_{\tau\land n}]1_A],~\forall A\in\mathcal{F}_{\tau\land n}.
\end{equation}
Note that 
\[
\sigma(X_{\tau\land n}) = \sigma\bigg(
\cup_{\ell=0}^n(\{\tau=\ell\}\cap\sigma(X_{\ell}))\cup(\{\tau>n\}\cap\sigma(X_n))
\bigg)
\]
Thus $\{\tau\le n\}$ and $\{\tau>n\}$ are in $\sigma(X_{\tau\land n})$. Then we can simplify the RHS of (\ref{Eq:7:1}):
\begin{align*}
\mathbb{E}[\mathbb{E}[f(X_{\tau\land(n+1)})\mid X_{\tau\land n}]1_A]
&=
\mathbb{E}[\mathbb{E}[f(X_{\tau\land(n+1)})(1\{\tau\le n\} + 1\{\tau>n\})\mid X_{\tau\land n}]1_A]\\
&=
\mathbb{E}[f(X_{\tau\land(n)})1\{\tau\le n\}1_A]
+
\mathbb{E}[\mathbb{E}[f(X_{\tau\land(n+1)})1\{\tau>n\}\mid X_{\tau\land n}]1_A]\\
&=
\mathbb{E}[f(X_{\tau\land(n)})1\{\tau\le n\}1_A]
+
\mathbb{E}[\mathbb{E}[f(X_{\tau\land(n+1)})\mid \mathcal{F}_{\tau\land n}]1\{A\cap\{\tau>n\}\}]\\
&=\mathbb{E}[f(X_{\tau\land(n)})1\{\tau\le n\}1_A]
+
\mathbb{E}[f(X_{\tau\land(n+1)})1\{A\cap\{\tau>n\}\}]\\
&=\mathbb{E}[f(X_{\tau\land(n+1)})1_A]
\end{align*}


\end{proof}

\begin{remark}
Note that the stopped process $X_{\cdot}^{\tau}$ does not have time-homogeneous transition in general. 
For instance, let $\tau\equiv k$, then $X_n^\tau=X_{k\land n}$, i.e., the transition may change before or after time $k$.
\end{remark}






















