
\chapter{Week 3}

\section{Monday}\index{Monday_lecture}

\subsection{Weak law of large numbers}
Firstly, review two modes of convergence for random variables:
\begin{definition}[Convergence of random variables]
For a sequence of $X_n, n\ge1$ and $X$,
\begin{enumerate}
\item
we say $X_n\to X$ a.s. if $\mathbb{P}(\Omega_0)=1$ for 
\[
\Omega_0 = \{\omega: \lim_{n\to\infty}X_n(\omega)=X(\omega)\}.
\]
\item
we say $X_n\to X$ in probability if
\[
\forall\varepsilon>0,~\lim_{n\to\infty}\mathbb{P}\bigg(
|X_n-X|>\varepsilon
\bigg)=0.
\]
\end{enumerate}
\end{definition}
Almost sure convergence is defined by the sample-path based condition,
while the convergence in probability is defined based on probability.
To answer the relation between these two convergence modes,
we reformulate the condition for almost sure convergence into a probability-based one.

\begin{proposition}
The sequence of random variables $X_n\to X$ a.s. if and only if 
\[
\forall k\ge1,\quad
\lim_{\ell\to\infty}\mathbb{P}\bigg\{
\cup_{n=\ell}^\infty\{|X_n-X|>1/k\}
\bigg\}=0.
\]
\end{proposition}
\begin{proof}
Starting from the a.s. convergence definition, we find an equivanent definition for $\omega\in\Omega_0$:
\[
\forall \varepsilon>0,~\exists n_0\ge1\text{ s.t. }\forall n\ge n_0,~|X_n(\omega) - X(\omega)|<\varepsilon.
\]
Or equivalently,
\[
\forall k\ge1,~\exists n_0\ge1\text{ s.t. }\forall n\ge n_0,~|X_n(\omega) - X(\omega)|\le\frac{1}{k}.
\]
Therefore, we have an equivalent definition for $\Omega_0$:
\[
\Omega_0 = 
\bigcap_{k=1}^\infty
\bigcup_{n_0=1}^\infty
\bigcap_{n=n_0}^\infty
\bigg\{
\omega:~|X_n(\omega) - X(\omega)|\le\frac{1}{k}
\bigg\}
\]
Taking the complement of $\Omega_0$, we say $X_n\to X$ a.s. if and only if
\[
\mathbb{P}\bigg(
\bigcup_{k=1}^\infty
\bigcap_{n_0=1}^\infty
\bigcup_{n=n_0}^\infty
\bigg\{
\omega:~|X_n(\omega) - X(\omega)|>\frac{1}{k}
\bigg\}
\bigg)=0.
\]
Since $\mathbb{P}(\cup_{k=1}^\infty A_k)=0$ is equivalent to $\mathbb{P}(A_k)=0,\forall k\ge1$, we imply 
\[
\mathbb{P}\bigg(
\lim_{n_0\to\infty}
\bigcup_{n=n_0}^\infty
\bigg\{
\omega:~|X_n(\omega) - X(\omega)|>\frac{1}{k}
\bigg\}
\bigg)=0,\forall k\ge1.
\]
By the continuity of probability, 
\[
\forall k\ge1, \lim_{n_0\to\infty}\mathbb{P}\bigg(
\bigcup_{n=n_0}^\infty
\bigg\{
\omega:~|X_n(\omega) - X(\omega)|>\frac{1}{k}
\bigg\}
\bigg)=0.
\]
The proof is complete.
\end{proof}
Based on this proposition, we imply the relation between convergence a.s. and convergence in probability:
\begin{corollary}
If $X_n\to X$ a.s., then $X_n\to X$ in probability.
\end{corollary}
\begin{proof}
This is because for any $\varepsilon>0$,
\[0\le
\mathbb{P}(|X_n-X|<\varepsilon)\le 
\mathbb{P}(\cup_{n=\ell}^\infty \{|X_n-X|<\varepsilon\}).
\]
\end{proof}

These facts above shows that we need to show the law of large numbers (LLN) based on different modes of convergence.
We will first show the weak LLN, assuming the finiteness of moments of $X_n$.

\begin{definition}[Moment of a random variable]
Consider a random variable $X$ and $a\ge0, c\in\mathbb{R}$.
\begin{itemize}
\item
$\mathbb{E}(X^a)$ denotes the $a$-th moment of $X$;
\item
$\mathbb{E}(|X|^a)$ denotes the $a$-th absolute moment of $X$;
\item
$\mathbb{E}(|X-c|^a)$ denotes the $a$-th absolute moment of $X$, centered at $c$.
\end{itemize}
\end{definition}

\begin{proposition}
Consider a random variable $X$,
\begin{itemize}
\item
For $a,b$ satisfying $0\le a<b$, 
\[
\mathbb{E}(|X|^b)<\infty\implies \mathbb{E}(|X|^a)<\infty.
\]
\item
For $a>0$ and $c\in\mathbb{R}$,
\[
\mathbb{E}(|X|^a)<\infty\Longleftrightarrow
\mathbb{E}(|X-c|^a)<\infty.
\]
\end{itemize}
\end{proposition}
\begin{proof}
The proof of the first and the second part is based on the following inequalities:
\begin{align*}
|x|^a&\le 1+|x|^b,\\
|x-c|&\le |x|+|c|\le2\max\{|x|,|c|\}.
\end{align*}
\end{proof}

An important tool for showing the weak LLN is Markov inequality:
\begin{proposition}[Markov Inequality]
For a random variable $X$ and $\mathcal{B}(\mathbb{R})$-measurable function $h:\mathbb{R}\to\mathbb{R}_+$,
\[
\mathbb{P}\bigg(
h(X)>t
\bigg)\le \frac{1}{t}\mathbb{E}[h(X)],\qquad t>0.
\]
\end{proposition}
\begin{proof}
Obseve the inequality holds: $t1(h(X)>t)\le h(X)$. Taking the expectation both sides leads to the desired result.
\end{proof}

\begin{corollary}[Chebyshev's inequality]
For a random variable $X$, we have the following concentration-type inequality:
\[
\mathbb{P}\bigg(
|X-\mathbb{E}[X]|>\varepsilon
\bigg)
\le\frac{1}{\varepsilon^2}\mathbb{E}[|X-\mathbb{E}[X]|^2].
\]
\end{corollary}
\begin{proof}
Apply Markov inequality with $h(x) = |x - \mathbb{E}[X]|^2$ and $t=\varepsilon^2$.
\end{proof}

Let $\{X_n\}_{n\ge0}$ be a sequence of random variables, and define the finite sum and sample mean by
\[
S_n = X_1+X_2+\cdots+X_n,\qquad
\overline{X}_n=S_n/n,~n\ge1.
\]
\begin{itemize}
\item
If  
$\overline{X}_n\to m$ in probability
and
$\mathbb{E}[\overline{X}_n]\to m$,
the sequence $\{X_n\}_{n\ge0}$ is said to {obey} the \emph{weak law of large numbers}.
\item
If
there are constants $a_n,b_n$ such that $\frac{S_n-b_n}{a_n}\to 0$ in probability
and
$\mathbb{E}[\overline{X}_n]\to m$,
the sequence $\{X_n\}_{n\ge0}$ is said to {obey} the \emph{extended weak law of large numbers}.
\item
If
$\overline{X}_n\to m$ a.s.
and
$\mathbb{E}[\overline{X}_n]\to m$,
the sequence $\{X_n\}_{n\ge0}$ is said to {obey} the \emph{strong law of large numbers}.
\end{itemize}

Here we give a general sufficient condition for constructing the weak LLN:
\begin{proposition}
If 
\begin{equation}\label{Eq:3:1}
\lim_{n\to\infty}\mathbb{E}[(\overline{X}_n - m)^2]=0,
\end{equation}
then $\overline{X}_n\to m$ in probability.
\end{proposition}
\begin{proof}
Apply the Chebyshev's inequality while replacing $X$ with $\overline{X}_n$:
\[
\mathbb{P}\left(
|\overline{X}_n-m|>\varepsilon
\right)
\le
\frac{1}{\varepsilon^2}\mathbb{E}[(\overline{X}_n - m)^2]\to0.
\]
The proof is complete.
\end{proof}

Note that once (\ref{Eq:3:1}) is satisfied, the weak LLN holds, no matter whether $\{X_n\}$ is i.i.d. or not.
Now we provide the sufficient condition of $\{X_n\}$ for satisfying (\ref{Eq:3:1}).

\begin{theorem}[A baby version of weak LLN]
Assume that 
\begin{enumerate}
\item
If $\{X_n\}$ is i.i.d.
\item
$\mathbb{E}[X_n^2]<\infty$, or equivalently, $\sigma_X^2<\infty$,
\end{enumerate}
then condition (\ref{Eq:3:1}) for the weak LLN is satisfied, with $m=\mathbb{E}[X_n]$.
\end{theorem}
\begin{proof}
Observe the following equality holds:
\begin{align*}
\mathbb{E}[(\overline{X}_n-m)^2]&=\frac{1}{n^2}\mathbb{E}\left[
\bigg(\sum_{\ell=1}^n(X_{\ell}-m)\bigg)^2
\right]\\
&=
\frac{1}{n^2}\mathbb{E}\left[
\sum_{\ell=1}^n(X_{\ell}-m)^2
+
\sum_{\ell\ne k: \ell,k\in[1:n]}(X_{\ell}-m)(X_k-m)
\right]\\
&=\frac{1}{n^2}\sum_{\ell=1}^n\mathbb{E}\left[
(X_{\ell}-m)^2\right]\triangleq \frac{1}{n^2}\sum_{\ell=1}^n\sigma_X^2\\
&=\frac{1}{n}\sigma_X^2\to0.
\end{align*}
\end{proof}

Let's see an interesting application of weak LLN in polynomial fitting:
\begin{proposition}
For a continuous function $f:[0,1]\to\mathbb{R}$, let
\[
f_n(x)\triangleq \sum_{k=0}^n\binom{n}{k}x^k(1-x)^{n-k}f(k/n),
\]
then $\sup_{x\in[0,1]}|f_n(x) - f(x)|\to0$ as $n\to\infty$.
\end{proposition}
\begin{proof}
The idea is to construct $\{X_n\}$ such that $f_n(x)$ relates to the expectation of some function of $\{X_n\}$.
Construct $S_n = X_1+X_2+\cdots+X_n$ with i.i.d. $\{X_n\}$ and $\mathbb{P}(X_n=1)=p, \mathbb{P}(X_n=0)=1-p$.
By weak LLN, $S_n/n\to p$ in probability.

Moreover, observe that 
\[
\mathbb{E}[f(S_n/n)] = f_n(p)
\]
Apply the uniform continuity of $f$, i.e., for $\forall\varepsilon>0$, there exists $\delta>0$ such that $|x-y|<\delta$ implies $|f(x)-f(y)|<\varepsilon$. Also, $f$ is bounded, i.e., $\sup_{x\in[0,1]}|f(x)|<M$ for some $M\ge0$.
Now we begin to upper bound $|f_n(p) - f(p)|$:
\begin{align*}
|f_n(p) - f(p)| &= \left|
\mathbb{E}[f(S_n/n) - f(p)]
\right|\\
&\le
\mathbb{E}[
|f(S_n/n) - f(p)|]\\
&=
\mathbb{E}[|f(S_n/n) - f(p)| 1(|S_n/n-p|\le\delta)] + \mathbb{E}[|f(S_n/n) - f(p)| 1(|S_n/n-p|>\delta)]
]\\
&\le\varepsilon + 2M\mathbb{P}(|S_n/n-p|>\delta)
\end{align*}
Since $S_n/n\to p$ in prob., $\lim_{n\to\infty}\mathbb{P}(|S_n/n-p|>\delta)=0$ holds uniformly for $p$.
Therefore, taking $n\to\infty$ both sides, $|f_n(p) - f(p)|\le (2M+1)\varepsilon$.


\end{proof}

Now we study a stronger version of weak LLN:
\begin{theorem}[Stronger version of weak LLN]
Assume that 
\begin{enumerate}
\item
If $\{X_n\}$ is i.i.d.
\item
$\mathbb{E}[|X_n|]<\infty$,
\end{enumerate}
then $\overline{X}_n\to m$ in probability, with $m=\mathbb{E}[X_n]$.
\end{theorem}
This theorem can be proved by the truncation of $X_n$:
\begin{enumerate}
\item
First define $X_{\ell}^{(n)} = X_{\ell}1(|X_{\ell}\le n|)$ and show $\overline{X}_n\to \overline{m}_{n}^{(n)}$ with $\overline{m}_{n}^{(n)}=\mathbb{E}(\overline{X}_{n}^{(n)})$;
\item
Then show that $\overline{m}_{n}^{(n)}\to m$.
\end{enumerate}
\begin{proof}[Proof for the first part]
Define $X_{\ell}^{(n)} = X_{\ell}1(|X_{\ell}\le n|)$,
$S_n^{(n)} = \sum_{\ell=1}^nX_{\ell}^{(n)}$,
$\overline{X}_n^{(n)} = S_n^{(n)}/n$,
and $\overline{m}_{n}^{(n)}=\mathbb{E}[\overline{X}_n^{(n)}]$.
Firstly show that $n\cdot\mathbb{P}(X>n)\to0$:
\begin{align*}
0&\le n\cdot\mathbb{P}(X>n)\le n\cdot\mathbb{P}(|X|>n)\\
&=\mathbb{E}[n\cdot 1(|X|>n)]\le \mathbb{E}[|X|\cdot 1(|X|>n)]\\
&=\mathbb{E}[|X|] - \mathbb{E}[|X|\cdot 1(|X|\le n)]\to0.
\end{align*}

For fixed $\varepsilon>0$, we need to upper bound $\mathbb{P}(|\overline{X}_n- \overline{m}_{n}^{(n)}|>\varepsilon)$ as the following:
\begin{align*}
\mathbb{P}(|\overline{X}_n- \overline{m}_{n}^{(n)}|>\varepsilon)
&=
\mathbb{P}(|\overline{X}_n- \overline{m}_{n}^{(n)}|>\varepsilon, S_n^{(n)}\ne S_n)
+
\mathbb{P}(|\overline{X}_n- \overline{m}_{n}^{(n)}|>\varepsilon, S_n^{(n)}= S_n)\\
&\le \mathbb{P}(S_n^{(n)}\ne S_n) + \mathbb{P}(|\overline{X}_n^{(n)}- \overline{m}_{n}^{(n)}|>\varepsilon)
\end{align*}
where
\[
\mathbb{P}(S_n^{(n)}\ne S_n)\le \sum_{\ell=1}^n\mathbb{P}(X_{\ell}^{(n)}\ne X_{\ell})=n\cdot \mathbb{P}(|X_1|>n)\to0,
\]
and
\[
\mathbb{P}(|\overline{X}_n^{(n)}- \overline{m}_{n}^{(n)}|>\varepsilon)
\le\frac{1}{\varepsilon^2}\mathbb{E}
[|\overline{X}_n^{(n)}- \overline{m}_{n}^{(n)}|^2]\to0.
\]
\end{proof}
\begin{proof}[Proof for the second part]
Note that
\[
\overline{m}^{(n)}_n-m = \frac{1}{n}\sum_{\ell=1}^nm_{\ell}^{(n)}-m
=
m_1^{(n)} - m = -\mathbb{E}[X_1 1(|X_1|>n)]\to0.
\]
\end{proof}









%
%We show the LLN for different version of convergence. 
%
%We first show the weak LLB, by assuming the finiteness of moments of $X_n$.
%
%Definition of moment:
%
%How they are related
%\begin{proposition}
%For $0\le a<b$,
%\[
%\mathbb{E}[|X|^b]<\infty\implies
%\mathbb{E}[|X|^a]<\infty
%\]
%\[
%\mathbb{E}[|X|^a]<\infty\Longleftrightarrow
%\mathbb{E}[|X-c|^a]
%\]
%\end{proposition}
%\begin{proof}
%For any $x\in\mathbb{R}$, $|x|^a\le 1+|x|^b$. Thus
%\[
%\mathbb{E}[|X|^a]\le \mathbb{E}[|X|^b]+1<\infty.
%\]
%
%\end{proof}
%
%\begin{proposition}
%For a random variable $X$ and $\mathcal{B}(\mathbb{R})$-measurable function 
%$h: \mathbb{R}\to\mathbb{R}_+$,
%\[
%\mathbb{P}(h(X)>t) \le \frac{1}{t}\mathbb{E}[h(X)]
%\]
%\end{proposition}

\subsection{Strong Law of Large Numbers}

The strong law of large numbers says that the sample average $\overline{X}_n$ of $X_1,X_2,\ldots,X_n$ almost surely converges to their mean $m=\mathbb{E}[X_1]$.
The a.s. convergence result allows us to estimate $m$ by the sample average.
Since the strong LLN requires $\mathbb{P}(|\overline{X}_n-m|>\varepsilon)$ converges to zero \emph{faster} than the weak LLN, we may need a higher moment of $X_n$ to be finite.

In order to show the strong LLN, first prepare for two lemmas:
\begin{proposition}
For $A_{\ell}\in\mathcal{F}, \ell=1,2,\ldots$, if $\sum_{\ell=1}^\infty\mathbb{P}(A_{\ell})<\infty$,
then
\[
\lim_{n\to\infty}\mathbb{P}\bigg(
\bigcup_{\ell=n}^\infty A_{\ell}
\bigg)=0.
\]
\end{proposition}
\begin{proof}
This is because
\[
\mathbb{P}\bigg(
\bigcup_{\ell=n}^\infty A_{\ell}
\bigg)
\le \sum_{\ell=n}^\infty\mathbb{P}(A_{\ell})\to0.
\]
\end{proof}

\begin{proposition}\label{pro:3:7}
The sequence of random variables $X_n\to X$ a.s. if 
\[\forall\varepsilon>0,\quad
\sum_{n=1}^\infty \mathbb{P}(|X_n - X|>\varepsilon)<\infty.
\]
\end{proposition}

The baby version of strong LLN is stated in the following:
\begin{theorem}[Baby version of strong LLN]
Assume that 
\begin{itemize}
\item
$\{X_n\}$ is i.i.d.;
\item
$\mathbb{E}[X_1^4]<\infty$;
\end{itemize}
then $\overline{X}_n\to m$ a.s. with $m=\mathbb{E}[X_1]$.
\end{theorem}
By proposition~\ref{pro:3:7}, it suffices to show $\sum_{n=1}^\infty \mathbb{P}(|\overline{X}_n- m|>\varepsilon)<\infty,\forall\varepsilon>0$.
Actually, we need to show that for $\varepsilon>0$,
\[
\mathbb{P}(|\overline{X}_n- m|>\varepsilon)<\text{constant}\cdot1/n^2.
\]
By Markov inequality,
\[
\mathbb{P}(|\overline{X}_n- m|>\varepsilon)\le\frac{\mathbb{E}[(\overline{X}_n- m)^4]}{\varepsilon^4}
\]
Therefore, we need to show the following result:
\[
\mathbb{E}[(\overline{X}_n- m)^4]
=
\frac{1}{n^4}\bigg(
n\cdot\beta + 3n(n-1)\sigma^4
\bigg),\qquad\beta = \mathbb{E}[(X_1-m)^4]<\infty.
\]

The $4$-th moment condition can be weakened as follows:
\begin{theorem}[Strong LLN]
The strong LLN holds if $\{X_n\}_{n\ge1}$ is i.i.d. and $\mathbb{E}[|X_1|]<\infty$.
\end{theorem}
\begin{remark}
The i.i.d. assumption in the strong LLN can also be weaked into:
\begin{itemize}
\item
$\{X_n\}$ is \emph{stationary}, and \emph{ergodic}.
\end{itemize}
\end{remark}

























