
\chapter{Week 9}

\section{Monday}\index{Monday_lecture}
\subsection{Classification of States}

\paragraph{Motivation}
Let $X_{\cdot}$ denote a Markov chain.
We are interested in the transition probability $p_{i,j}^{(n)}$ for $i,j\in S$.
We want to ask the following questions:
\begin{enumerate}
\item
Is the limit $p_{i,j}^{(n)}$ exists as $n\to\infty$?
\item
If so, how to compute it?
\end{enumerate}
Before answering this question, we classify states in $S$ according to:
\begin{itemize}
\item
Go to state $j$ from state $i$ with positive probability?
\item
Is it any period to return to the starting state?
\item
With probability 1 to return to the same state?
The expectation time for return is finite?
\end{itemize}

\begin{definition}
\begin{itemize}
\item
We say $i,j$ commute to each other if $i\Leftrightarrow j$;
\item
We say $C$ is irreducible if $i\Leftrightarrow j$ for any $i,j\in C$;
\item
We say $C$ is closed if $p_{i,j}=0$ for $\forall i\in C, j\in S\setminus C$.
\end{itemize}
\end{definition}
The notion $\Rightarrow$ is an equivalence relation on $C$: 
when $i\Rightarrow j$ and $j\Rightarrow k$, i.e., $p_{i,j}^{(\ell)}>0, p_{j,k}^{(m)}>0$, then
\[
p_{i,k}^{(\ell+m)}=\sum_{j'\in S}p_{i,j'}^{(\ell)}p_{j',k}^{(m)}\ge p_{i,j}^{(\ell)}p_{j,k}^{(m)}>0,
\]
which means that $i\Rightarrow k$.

\begin{proposition}
The state space $S$ of a Markov chain can be partitioned into mutually exclusive irreducible closed sets and the other set which does not contain any irreducible closed set.
\end{proposition}
\begin{proof}
Construction of such set: 
Let $C(i)=\{j\in S:~i\Leftrightarrow j\}$ if it is closed, otherwise $C(i)=\emptyset$.
Finally, construct $T=S\setminus(\cup_{i\in S}C(i))$.
\end{proof}

This lemma suggests that we only need to consider an irreducible closed set to ask the question at the beginning.

\begin{definition}[Stopping Time]
\begin{itemize}
\item
Define the first time to get $j$ from $i$ as $\tau_j=\inf\{n\ge1:~X_n=j\}$, where $\tau_j=\infty$ if $X_n\ne j,\forall n$;
\item
Define the hitting probability
$f_{i,j}^{(n)}=\mathbb{P}(\tau_j=n\mid X_0=i).
$
\end{itemize}
\end{definition}

\begin{proposition}[First Time Decomposition]
When $n=0$, $p_{i,j}^{(0)}=1(i=j)$.
When $n\ge1$,
\[
p_{i,j}^{(n)}=\sum_{\ell=1}^nf_{i,j}^{(\ell)}p_{j,j}^{(n-\ell)}.
\]
\end{proposition}
\begin{proof}
\begin{align*}
\mathbb{P}(X_n=j\mid X_0=i)&=\sum_{\ell=1}^n\mathbb{P}(X_n=j, \tau_j=\ell\mid X_0=i)\\
&=\sum_{\ell=1}^n\mathbb{P}(\tau_j=\ell\mid X_0=i)\mathbb{P}(X_n=j\mid \tau_j=\ell, X_0=i)\\
&=\sum_{\ell=1}^n\mathbb{P}(\tau_j=\ell\mid X_0=i)\mathbb{P}(X_n=j\mid X_\ell=j)=\sum_{\ell=1}^nf_{i,j}^{(\ell)}p_{j,j}^{(n-\ell)}.
\end{align*}
\end{proof}

\begin{definition}[Period]
Let $\mathbb{N}=\{k\in\mathbb{Z}, k\ge1\}$,
For $i\in S$, define the period of $i$ as
\[
d(i)=\text{GCD}\{n:~p_{i,i}^{(n)}>0\}
\]
We denote $d(i)=\infty$ if $p_{i,i}^{(n)}=0,\forall n$.
\end{definition}

\begin{proposition}
For $i,j\in S$ with $i\ne j$, $i\Leftrightarrow j$ implies $d(i)=d(j)<\infty$.
\end{proposition}
\begin{proof}
It suffices to show that $d(j)$ is a multiple of $d(i)$.
Since $i\Leftrightarrow j$, there exists $\ell, m$ such that 
$p_{i,j}^{(\ell)}>0, p_{j,i}^{(m)}>0$.
For $n$ such that $p_{j,j}^{(n)}>0$,
\[
p_{i,i}^{(\ell+n+m)}\ge p_{i,j}^{(\ell)}p_{j,j}^{(n)}p_{j,i}^{(m)}>0,
\]
which means that $\ell+n+m$ is a multiple of $d(i)$.
Since $\ell+m$ does so, $n$ is a multiple of $d(i)$.
\end{proof}

\begin{definition}[Aperiodic]
An irreducible Markov chain is said to be aperiodic if some state has period $1$.
\end{definition}
The following gives an alternative for finding the period of a state.
\begin{proposition}
For $j\in S$, let $f(j)=\text{GCD}\{n:~f_{j,j}^{(n)}>0\}$.
Then $f(j)=d(j)$.
\end{proposition}
\begin{proof}
Note that $f(j)$ is a multiple of $d(j)$ since $p_{j,j}^{(n)}>0$.
Conversely, pick any $n$ so that $p_{j,j}^{(n)}>0$.
Then there exists $\ell_1$ so that $f_{j,j}^{(\ell_1)}p_{j,j}^{(n-\ell_1)}>0$.
Inductively, $n=\ell_1+\cdots+\ell_k$, i.e., $n$ is a multiple of $f(j)$.
Hence, $d(j)$ is a multiple of $f(j)$.
\end{proof}

\begin{definition}[Return-Time Classes]
Define the following notation for $i,j\in S$:
\[
f_{i,j}^{(*)}=\mathbb{P}(\tau_j<\infty\mid X_0=i),\quad
\mu_{i,j}=\left\{
\begin{aligned}
\mathbb{E}[\tau_j\mid X_0=i],&\quad\text{if $f_{i,j}^{(*)}=1$}\\
\infty,&\quad\text{if $f_{i,j}^{(*)}<1$}
\end{aligned}
\right.
\]
\begin{enumerate}
\item
The state $i$ is transient if $f_{i,j}^{(*)}<1$;
\item
The state $i$ is recurrent if $f_{i,j}^{(*)}=1$;
\item
The state is positive recurrent if $f_{i,j}^{(*)}=1$ and $\mu_{i,i}<\infty$;
\item
The state is null recurrent if $f_{i,j}^{(*)}=1$ and $\mu_{i,i}=\infty$;
\end{enumerate}
\end{definition}

Define the following series to characterize the relation between $\{f_{i,j}^{(n)}\}$ and $\{p_{i,j}^{(n)}\}$:
\[
F_{i,j}(z)=\sum_{n=1}^\infty f_{i,j}^{(n)}z^n,\qquad
U_{i,j}(z)=\sum_{n=0}^\infty p_{i,j}^{(n)}z^n.
\]
\begin{proposition}
For $i,j\in S$ and $0<z<1$,
\[
U_{i,j}(z)-1(i=j)=F_{i,j}(z)U_{j,j}(z).
\]
Moreover, $\lim_{z\to1-}F_{i,i}(z)=f_{i,i}^{(*)}$ and 
$\lim_{z\to1-}\frac{1-F_{i,i}(z)}{1-z}=\mu_{i,i}$ if $f_{i,i}^{(*)}=1$.
\end{proposition}

\begin{proof}
\begin{align*}
U_{i,j}(z)-1(i=j)&=\sum_{n=1}^\infty z^n\sum_{\ell=1}^nf_{i,j}^{(\ell)}p_{j,j}^{(n-\ell)}=F_{i,j}(z)U_{j,j}(z).
\end{align*}
\end{proof}

\begin{theorem}
If $i\Leftrightarrow j$, then $i$ and $j$ are in the same return-time class.
\end{theorem}
\begin{proof}
The key is to show the identity
\[
(1-F_{j,j}(z))\ge c_{i,j}^{(\ell,m)}(z)(1-F_{i,i}(z))\ge0.
\]
\end{proof}
\begin{proposition}
\[
f_{i,i}^{(*)}=p_{i,i}+\sum_{j\in S\setminus i}p_{i,j}f_{j,i}^{(*)}.
\]
\end{proposition}
\begin{proof}
\[
\mathbb{P}(\tau_i<\infty\mid X_0=i)=p_{i,i}+\sum_{j\in S\setminus i}\mathbb{P}(\tau_i<\infty\mid X_1=j)\mathbb{P}(X_1=j\mid X_0=i).
\]
\end{proof}
If there is some $j$ so that $p_{i,j}>0$ but $f_{j,i}^{(*)}=0$, then immediately $f_{i,i}^{(*)}<1$.


\subsection{Renewal Equation}
For a probability distribution $\{f_n:~n\ge1\}$, define the renewal equation by $u_0=1$,
\[
u_n=\sum_{\ell=1}^nf_{\ell}u_{n-\ell}.
\]
Specifically, we take $f_{\ell}=f_{j,j}^{(\ell)}$. Then $\{f_{\ell}\}$ is a distribution, and $p_{j,j}^{(n)}$ satisfies the renewal equation.
The existence of the limit $u_n$ is answered by the following theorem:

\begin{theorem}
For the renewal equation, if $\text{GCD}\{n\ge1:~f_n>0\}=1$, then the solution $\{u_n~:n\ge1\}$ admits the limit
\[
\lim_{n\to\infty}u_n=\frac{1}{\mu}
\]
with $\mu=\sum_{\ell=1}^\infty\ell f_{\ell}$.
\end{theorem}

\subsection{Limit of State Distribution}

\begin{theorem}\label{The:9:3}
If state $j$ is transient, then
\[
\lim_{n\to\infty}p_{i,j}^{(n)}=0,\quad \forall i\in S.
\]
\end{theorem}
\begin{proof}
First consider $i=j$. 
Since $j$ is transient, $F_{j,j}(1)<1$, and 
\[
\sum_{n=0}^\infty p_{j,j}^{(n)}=\lim_{z\to1-}U_{j,j}(z)=\frac{1}{1-F_{j,j}(1)}<\infty.
\]
Thus $p_{j,j}^{(n)}\to0$.

For $i\ne j$, 
\[
p_{i,j}^{(n)}=\sum_{\ell=1}^\infty f_{i,j}^{(\ell)}1(\ell\le n)p_{j,j}^{(n-\ell)}
\]
Taking $n\to\infty$ gives the desired result.
\end{proof}

\begin{theorem}\label{The:9:4}
If $j$ is aperiodic and recurrent, then
\[
\lim_{n\to\infty}p_{j,j}^{(n)}=\frac{1}{\mu_{j,j}},\quad
\lim_{n\to\infty}p_{i,j}^{(n)}=f_{i,j}^{(*)}\frac{1}{\mu_{j,j}},
\]
where $\mu_{j,j}=\mathbb{E}[\tau_j\mid X_0=j]$.
\end{theorem}

\begin{proof}
The first limit is by renewal theorem: 
\[
\mu_{j,j}=\sum_{j=1}^\infty \ell f_{j,j}^{(\ell)}=\mathbb{E}[\tau_j\mid X_0=j].
\]
The second limit is based on the first-time decomposition:
\[
p_{i,j}^{(n)}=\sum_{\ell=1}^n f_{i,j}^{(\ell)}p_{j,j}^{(n-\ell)}
\]

\end{proof}

\begin{theorem}
If $j$ is aperiodic and has period $d$, then
\[
\lim_{n\to\infty}p_{i,j}^{(nd)}=f_{i,j}^{(*)}\frac{d}{\mu_{j,j}}
\]
\end{theorem}
\begin{proof}
Construct the 1-step transition Markov process $Y_n=X_{nd}$, then
\[
\lim_{n\to\infty}p_{j,j}^{(nd)}=\frac{1}{\mathbb{E}[\tau_j^{(d)}\mid X_0=j]}
\]
\end{proof}

\begin{theorem}
If the Markov chain is aperiodic and irreducible, then
\[
\lim_{n\to\infty}p_{i,j}^{(n)}=\frac{1}{\mu_{j,j}}.
\]
\end{theorem}
\begin{proof}
It suffices to show $f_{i,j}^{(*)}=1$.
\end{proof}

\textbf{Remaining problems}
\begin{itemize}
\item
What is the role of $p_{i,j}^{(n)}$ for large $n$ in application?
\item
In the positive recurrent case, the distribution $X_n$ weakly converges to a distribution?
\item
Is there any sufficient condition for $\mu_j<\infty$?
\item
Is there any simpler way to compute $\mu_j$?
\end{itemize}















