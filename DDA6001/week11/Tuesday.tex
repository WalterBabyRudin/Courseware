
\chapter{Week 11}

\section{Monday}\index{Monday_lecture}

\subsection{Continuous Time Stochastic Processes}

The big diference between discrete and continuous time stochastic processes is that the time index is uncountable.
\begin{itemize}
\item
Thus it is natural to consider a continuous time stochastic process $X_{\cdot}$ as a function from $\mathbb{R}_+\times\Omega\to S$;
\item
Then we need to define the $\sigma$-algebra on $\mathbb{R}_+\times\Omega$;
We also need the notion of null set and completeness.
\end{itemize}

\begin{definition}[Null Set and Complete Set]
For a $\sigma$-algebra $\mathcal{F}$, we say $N\subseteq\Omega$ is a null set if there exists $A\in\mathcal{F}$ so that $N\subseteq A$ and $\mathbb{P}(A)=0$.
We say $\mathcal{F}$ and all its sub $\sigma$-fields are complete, if they contains all null sets.
\end{definition}

\begin{definition}[Product $\sigma$-algebra]
The product $\sigma$-algebra, $\mathcal{B}(\mathbb{R}_+)\otimes\mathcal{F}$ defined on $\mathbb{R}_+\times\Omega$, is defined as the minimal $\sigma$-field on $\mathbb{R}_+\times\Omega$ containing
\[
\{B\times A:~B\in\mathcal{B}(\mathbb{R}_+), A\in\mathcal{F}\}.
\]
Then we have the measurable space $(\mathbb{R}_+\times\Omega, \mathcal{B}(\mathbb{R}_+)\otimes\mathcal{F})$.
We can define a measure on this space as the following.
For a measure $\mu$ on $(\mathbb{R}_+, \mathcal{B}(\mathbb{R}_+))$ and probability measure $\mathbb{P}$ on $(\Omega,\mathcal{F})$, the product measure $\mu\otimes\mathbb{P}$ is defined as
\[
(\mu\otimes\mathbb{P})(B\times A)=\mu(B)\mathbb{P}(A).
\]
\end{definition}

\begin{definition}[Measurable Process]
A continuous time stochastic process $X(\cdot)$ is a mapping
\[
X(\cdot):~(t,\omega)\in\mathbb{R}_+\times\Omega\to X(t,\omega)\equiv X(t)(\omega)\in S.
\]
We say $X(\cdot)$ is a measurable process, or $\mathcal{B}(\mathbb{R}_+)\otimes\mathcal{F}$-measurable, if
\[
\{(t,\omega)\in\mathbb{R}_+\times\Omega:~X(t,\omega)\in B\}\in \mathcal{B}(\mathbb{R}_+)\otimes\mathcal{F},\quad\forall t\ge0, B\in\mathcal{B}(S).
\]
\end{definition}

\begin{definition}[Filtration]
Let $\mathcal{F}_t$ be a sub $\sigma$-algebra of $\mathcal{F}$ so that $\mathcal{F}_s\subseteq\mathcal{F}_t$ for $0\le s<t$. Then $\mathbb{F}\equiv\{\mathcal{F}_t:~t\ge0\}$ is called a filtration.
For $t\ge0$, define
\[
\mathcal{F}_{t-}=\sigma\bigg(
\cup_{0\le s<t}\mathcal{F}_s
\bigg),\quad
\mathcal{F}_{t+}=\cap_{t<s}\mathcal{F}_s,\quad
\mathcal{F}_\infty=\sigma\bigg(
\cup_{s\ge0}\mathcal{F}_s
\bigg),
\]
with $\mathcal{F}_{0-}=\{\emptyset,\Omega\}$.
Assume that the filtration $\mathbb{F}$ is right-continuous, i.e., $\mathcal{F}_{t}=\mathcal{F}_{t+}$ for all $t\ge0$.
Then $(\Omega,\mathcal{F},\mathbb{F},\mathbb{P})$ is called a stochastic basis.
\end{definition}

\begin{definition}[Stopping Time]
A random variable $\tau$ valued in $[0,\infty]$ is called a stopping time if $\{\tau\le t\}\in\mathcal{F}_t$ for any $t\ge0$. Define the $\tau$-stopped $\sigma$-field as
\[
\mathcal{F}_{\tau}\equiv\{A\in\mathcal{F}:~A\cap\{\tau\le t\}\in\mathcal{F}_t, \forall t\in[0,\infty]\}.
\]
\end{definition}

We say $X(\cdot)$ on $(\Omega,\mathcal{F},\mathbb{F},\mathbb{P})$ is adapted if $X(t)$ is $\mathcal{F}_t$-measurable. 
We extend this notion on the product measurable space $(\mathbb{R}_+\times\Omega, \mathcal{B}(\mathbb{R}_+)\otimes\mathcal{F})$:
\begin{definition}[Progressively Measurable]
We say $X(\cdot)$ is progressively measurable if
\[
\forall B\in\mathcal{B}(S),\forall t\ge0,\quad
\{(u,\omega)\in[0,t]\times\Omega:~X(u,\omega)\in B\}\in\mathcal{B}_t\otimes\mathcal{F}_t
\]
is progressively measurable.
\end{definition}

\begin{proposition}
Let $\tau$ be a stopping time, and $\mathbb{F}$ be right-continuous.
Suppose that $X(\cdot)$ is adapted to $\mathbb{F}$ and right-continuous in time, then
\begin{enumerate}
\item
$X(\tau)$ is $\mathcal{F}_{\tau}$-measurable;
\item
$X(\cdot)$ is progressively measurable;
\item
$\tau$-stopped process $X^{\tau}(\cdot)$ is progressively measurable.
\end{enumerate}
\end{proposition}
\begin{proof}
Here we only present the proof for b) implies c), i.e., suffices to show that for any $B\in\mathcal{B}(S)$ and $t\ge0$, $A(\tau,t,B)\in\mathcal{B}_t\otimes\mathcal{F}_t$, where
\[
A(\tau,t,B)=\{(u,\omega)\in[0,t]\times\Omega:~X(u\land\tau,\omega)\in B\}.
\]
Construct the stochastic interval
\[
[\sigma,\tau]=\{(u,\omega)\in[0,t]\times\Omega:~\sigma(\omega)\le u\le \tau(\omega)\},
\]
then we have
\[
[0,\tau]\cap A(\infty,t,B)\in \mathcal{B}_t\otimes\mathcal{F}_t,\quad
(\tau,t]\cap A(\tau, t, B)\in \mathcal{B}_t\otimes\mathcal{F}_t.
\]
\end{proof}


The progressive measurability is wearker than the right-continuity of a sample path.
But the  latter is easy to check. Therefore, we make the following assumptions:
\begin{itemize}
\item
$X_{\cdot}$ is adapted;
\item
For each $\omega$, the sample path $X(t)(\omega)$ is right-continuous, and has limit from the left at each $t\ge0$:
\begin{align*}
X(t)(\omega)&=X(t+)(\omega),\qquad
X(t-)(\omega)\text{ exists}.
\end{align*}
\item
Filtration $\mathbb{F}$ is right-continuous, i.e., $\mathcal{F}=\mathcal{F}_{t+}=\sigma(\cap_{t<u}\mathcal{F}_u)$.
\item
Filtration $\mathbb{F}$ is complete.
\end{itemize}

\begin{definition}[Predictable]
Let $\mathcal{U}_{-}(S)$ be the set of all left-continuous and adapted measurable process $\mathbb{R}_+\times\Omega$ to $S$.
Define
\[
\mathcal{P}=\sigma\left(
\bigcup_{f\in \mathcal{U}_{-}(S)}\bigcup_{B\in\mathcal{B}(S)}
\{(t,\omega)\in\mathbb{R}_+\times\Omega:~f(t,\omega)\in B\}
\right),
\]
which is called a predictable $\sigma$-field.
A stochastic process $X(\cdot)$ is said to be predictable if it is $\mathcal{P}$-measurable.
A random variable $\tau\ge0$ is said to be a predictable time if $[0,\tau)$ is $\mathcal{P}$-measurable.
\end{definition}

\begin{proposition}
The predictable $\sigma$-algebra $\mathcal{P}$ is generated by either one of the sets:
\begin{itemize}
\item
$\{0\}\times A$ for $A\in\mathcal{F}_0$, and $[t,\tau]$ for all stopping time $\tau$;
\item
$\{0\}\times A$ for $A\in\mathcal{F}_0$, and $(s,t]\times A$ for all $s<t, A\in\mathcal{F}_s$.
\end{itemize}
\end{proposition}

\begin{proof}
Let $\mathcal{P}_1,\mathcal{P}_2$ be the $\sigma$-algebra generated by i), ii).
Define $f(t,\omega)=1(0\le t\le \tau(\omega))$, which is predictable, i.e., $\mathcal{P}_1\subseteq\mathcal{P}$.
For $A\in\mathcal{F}_s$, define the notion $u_A=u1_A+\infty1_{A^c}$ for $u\ge s$.
For the set of rational numbers $Q$,
\[
(s,t]\times A=(s_A,t_A]=[0,t_A]\setminus[0,s_A]\in\mathcal{P}_1.
\]
Thus $\mathcal{P}_2\subseteq\mathcal{P}_1$.
Note that $X(t)$ is the limit of the discrete process $X^n(t)\equiv\sum_{i\ge0}X(i2^{-n})1(i2^{-n}<t\le (i+1)2^{-n})$, which is $\mathcal{P}_2$-measurable. So $\mathcal{P}\subseteq\mathcal{P}_2$.


\end{proof}

\begin{remark}
If $X(\cdot)$ is predictable, then $X(t)$ is $\mathcal{F}_{t-}$-measurable. But the converse is not true.
\end{remark}

\subsection{Continuous time martingale}
For discrete case, we find any real-valued process can be expressed as a semi-martingale.
We wish to study the similar topic for continuous case.
In general this result does not hold, but for most processes we encounter in applications, the answer is yes.
We will focus on a semi-martingale.

\begin{definition}[Martingale]
The process $M(\cdot)$ is called a martingale if
\[
\mathbb{E}[M(t)\mid\mathcal{F}_s]=M(s),\quad 0\le s<t.
\]
Now we define the semi-martingale as
\[
X(t)=A(t)+M(t),
\]
where $A(\cdot)$ is an adapted process of finite variations:
\[
|A|(t)=\sup_{\delta>0, t_0=0}\sup_{\Delta t\ge\delta}\sum_{n=1}^\infty1(t_n\le t)|A(t_n) - A(t_{n-1})|M\infty,\quad\forall t\ge0,
\]
and $M(\cdot)$ is a martingale.
The process $A(\cdot)$ can be expressed as the difference of non-decreasing functions.
If $A(\cdot)$ is predictable, then $X(\cdot)$ is called a special semi-martingale.
\end{definition}

\begin{definition}[Uniform Integrable]
For a set $K$, a collection of real-valued random variable $\{X_u:~u\in K\}$ is said to be uniformly integrable if
\[
\lim_{a\to\infty}\sup_{u\in K}\mathbb{E}[|X_u|1(|X_u|>a)]=0.
\]
A martingale $X(\cdot)$ is said to be uniform integrable if $\{X(t):~t\ge0\}$ is uniform integrable.
\end{definition}

\begin{proposition}[UI for Conditional Expectations]
Let $Y$ be a random variable with $\mathbb{E}[|Y|]<\infty$, and $\mathcal{H}$ be a collection of all $\sigma$-fields on $\Omega$, then $\{\mathbb{E}[Y\mid\mathcal{G}]:~\mathcal{G}\in\mathcal{H}\}$ is uniform integrable.
\end{proposition}
\begin{proof}
Define $Y_{\mathcal{G}}= \mathbb{E}[Y\mid\mathcal{G}]$, then
\begin{align*}
\mathbb{E}[|Y_{\mathcal{G}}|1(|Y_{\mathcal{G}}|>a)]
&=
\mathbb{E}[|\mathbb{E}[Y\mid\mathcal{G}]|1(|Y_{\mathcal{G}}|>a)]\\
&\le 
\mathbb{E}[\mathbb{E}[|Y|\mid\mathcal{G}]1(|Y_{\mathcal{G}}|>a)]\\
&=\mathbb{E}[|Y|1(|Y_{\mathcal{G}}|>a)],
\end{align*}
where 
\[
\mathbb{E}[|Y|1(|Y_{\mathcal{G}}|>a)]\le 
\mathbb{E}[|Y|1(|Y|>b)] + b\mathbb{E}[1(|Y_{\mathcal{G}}|>a)]
\]

\end{proof}

Then we define $X(t)=\mathbb{E}[Y\mid\mathcal{F}_t]$ for $t\ge0$.
By the proposition above, $X(\cdot)$ is UI. Moreover, we can show that $X(\cdot)$ is a martingale.

\begin{definition}[Closed Martingale]
A martingale $X_{u}(\cdot)=\{X_u(t):~0\le t<u\}$ for a constaint $u\in[0,\infty]$ is said to be closed if $X_u(u)$ exists and $\{X_u(t):~0\le t\le u\}$ is a martingale.
\end{definition}

If we define $X(\infty)=\mathbb{E}[Y\mid\mathcal{F}_\infty]$, then $X(\cdot)$ is closed.
Moreover, it is UI. This suggests the following lemma: 
\begin{proposition}\label{pro:11:4}
Let $X_u(\cdot)$ be a martingae for $u\in[0,\infty]$.
Then $X_u(\cdot)$ is UI iff it is closed.
\end{proposition}
\begin{proof}
It suffices to show the closeness result.
Construct $u_n\uparrow u$ and $Y_n=X_u(u_n)$, then $Y_{\cdot}$ is UI.
There exists $Y_{\infty}$ so that $Y_n\to Y_\infty$ almost surely.
Therefore, $X_u(\cdot)$ can be expressed as the conditional expectation form, which is closed:
\[
X_u(t)=\mathbb{E}[X_u(u_n)\mid\mathcal{F}_t]
=\mathbb{E}[\mathbb{E}[Y_\infty\mid\mathcal{F}_{u_n}]\mid\mathcal{F}_t]=\mathbb{E}[Y_\infty\mid\mathcal{F}_t].
\]
\end{proof}

\begin{definition}[Local Martingale]
A process $X(\cdot)$ is called a \emph{local martingale} if
there is a sequence of finite stopping times $\tau_n\uparrow \infty$ so that 
$X^{\tau_n}\equiv\{X(\tau_n\land t):~t\ge0\}$ is a martingale for each $n\ge1$.
\end{definition}

\begin{definition}[Class D]
A process $X(\cdot)$ is called of class $D$ if $\{X(\tau):~\tau\in\mathcal{P}\}$
is uniformly integrable for $\mathcal{P}$ be the set of all finite stopping times.
\end{definition}

\begin{proposition}
\begin{enumerate}
\item
A martingale $X(\cdot)$ is a local martingale;
\item
If $X(\cdot)$ is an uniformly integrable martingale, then it is of class $D$;
\item
A local martingale $X(\cdot)$ is uniformly integrable iff it is of class $D$.
\end{enumerate}
\end{proposition}
\begin{proof}
\begin{enumerate}
\item
It is clear if we take $\tau_n\equiv n$.
\item
By proposition~\ref{pro:11:4}, $X(\cdot)$ is closed, and therefore $X(t)=\mathbb{E}[Y\mid\mathcal{F}_t]$ for a random variable $Y$ with finite $\mathbb{E}[Y]$. 
Therefore, $X(\tau)=\mathbb{E}[Y\mid\mathcal{F}_{\tau}]$ for any stopping time $\tau$.
Hence, $\{X(\tau):~\tau\in\mathcal{P}\}$ is uniformly integrable.
\item
Let $\{\tau_n:~n\ge1\}$ be localizing sequence of finite stopping times for $X(\cdot)$.
Assume that it is of class $D$, then $\{X(\tau_n\land t):~t\ge0, n\ge1\}$ are uniformly integrable.
The uniform integrability of $X(\cdot)$ is because
\[
\sup_{n\ge1,t\ge0}\mathbb{E}[|X^{\tau_n}(t)|1(|X^{\tau_n}(t)|>a)]\ge
\sup_{t\ge0}\mathbb{E}[|X(t)|1(|X(t)|>a)].
\]
Conversely, assume that $\{X(\tau_n\land t):~t\ge0, n\ge1\}$ are uniformly integrable,
then $\{X(\tau_n\land u):~t\ge0, n\ge1\}$ are uniformly integrable and $X^{\tau_n}(u)\to X(u)$ almost surely. Hence, $X^{\tau_n}(u)\xrightarrow{L^1}X(u)$ and $X(s)=\mathbb{E}[X(t)\mid\mathcal{F}_s]$.
Thus it is of class $D$.
\end{enumerate}
\end{proof}

A non-decreasing process $X(\cdot)$ is a sub-martingale since
\[
\mathbb{E}[X(t)\mid\mathcal{F}_s] = X(s) + \mathbb{E}[X(t)-X(s)\mid\mathcal{F}_s]\ge X(s),\quad 0\le s<t.
\]
The following tells what a sub-martingale is.

\begin{theorem}[Doob Decomposition]
If $X(\cdot)$ is a sub-martingale of class $D$, then there uniquely exists a predictable non-decreasing process $A(\cdot)$ with $A(0)=0$ and a uniformly integrable martingale $M(\cdot)$ so that
\[
X(t) = A(t) + M(t),\quad t\ge0.
\]
\end{theorem}
\begin{proposition}
If a martingale $X(\cdot)$ is predictable and has finite variations, then $X(t)=0$ for all $t\ge0$ except for all the null sets.
\end{proposition}
\begin{proof}
It suffices to consider the case where $X(\cdot)$ is non-decreasing.
Then we have $X(\cdot)=A(\cdot)$ and $X(\cdot)=M(\cdot)$.
By the  uniqueness of decomposition, $X(\cdot)=0$.
\end{proof}

\begin{theorem}
If $X(\cdot)$ is a special martingale, i.e., $A(\cdot)$ is predictable, then the semi-martingale decomposition is unique.
\end{theorem}
\begin{proof}
Suppose that $X(t)=A'(t)+M'(t)$ is another semi-martingale form, then $A(t)-A'(t)=M'(t)-M(t)$ is a predictable martingale of finite variations, so it must vanish.
\end{proof}

\subsection{From discrete to continuous time}
Assume that $X(\cdot)$ is a cadlag process. We transfer results from its discrete time counter-part in the following way:
\begin{enumerate}
\item
For $n,i\ge0$, define the mesh $t_{n,i}=i2^{-n}$ and $Y_i^{(n)}=X(t_{n,i}), \mathcal{F}_i^{(n)}=\mathcal{F}_{n,i}$. Consider the discrete time process $Y_{\cdot}^{(n)}=\{Y_i^{(n)}:~i\ge0\}$ under the filtration 
$\mathbb{F}^{(n)}\equiv\{\mathcal{F}_i^{(n)}:~i\ge0\}$.
\item
For $t>0, n\ge0$, let $Y_{\cdot}^{(n)}(t)=\{Y_i^{(n)}:~0\le i\le n_t-1\}\cup\{X(t)\}$, and define 
$\mathbb{F}^{(n)}(t)\equiv\{\mathcal{F}_i^{(n)}:~0\le i\le n_t-1\}\cup\{\mathcal{F}_t\}$,
where 
\[
n_t=\min\{\ell\in\mathbb{Z}_+:~l\ge t2^n\}.
\]
Approximate $\{X(u):~0\le u\le t\}$ by $Y_{\cdot}^{(n)}(t)$;
\item
Derive continuous time results from those for $Y_{\cdot}^{(n)}$ by taking $n\to\infty$.
Here we need a topology on state space $S$ of $X(\cdot)$.
Usually we take $S=\mathbb{R}$ and the topology by Euclid norm.
\end{enumerate}
This procedure involves at most countable events, so the limiting result holds with probability 1 if 
each result holds with probability 1.

\begin{proposition}[Observation at Stopping Time]
Let $\tau$ be a stopping time, then
\begin{enumerate}
\item
$X(\tau)1(\tau<\infty)$ is $\mathcal{F}_{\tau}$-measurable;
\item
If $X(\cdot)$ is predictable, then $X(\tau)1(\tau<\infty)$ is $\mathcal{F}_{\tau-}$-measurable,
with
\[
\mathcal{F}_{\tau-}=\sigma(\mathcal{F}_0, \{A\in\mathcal{F}:~
A\cap\{t<\tau\}\in\mathcal{F}_t, t\ge0\}).
\]
\end{enumerate}
\end{proposition}
\begin{proof}
It suffices to show that $\{X(\tau)\in B\}\cap\{\tau\le t\}\in\mathcal{F}_t$ for any $t$.
On the event $\{\tau<\infty\}$, we approximate $\tau$ by:
\[
\tau_n^+=\sum_{i=0}^\infty t_{n,i}1(t_{n,i-1}\le \tau<t_{n,i}),\quad n\ge0,
\]
with $t_{n,i}=i2^{-n}$ and $\mathbb{F}^{(n)}\equiv \{\mathcal{F}_{t_{n,i}}:~i\ge0\}$.
It's clear that $\tau_n^+$ is $\mathbb{F}^{(n)}$-stopping time and $\tau_n^+\downarrow \tau$ as $n\to\infty$.
Note that $X(t_{n,i})1(\tau_n^+=t_{n,i})$ is $\mathcal{F}_{t_{n,i}}$-measurable, which implies
\[
X(\tau_n^+)1(\tau_n^+\le t_{n,j})=\sum_{i=0}^{j}X(t_{n,i})1(\tau_n^+=t_{n,i})\text{ is $\mathcal{F}_{t_{n,j}}$-measurable}.
\]
Thus $X(\tau_n^+)$ is $\mathcal{F}_{\tau_n^+}$-measurable for each $n\ge0$.
Moreover, $X(\tau_n^+)\to X(\tau)$ on $\{\tau<\infty\}$ as $n\to\infty$. Therefore, $X(\tau)1\{\tau<\infty\}$ is $\mathcal{F}_{\tau}$-measurable.

For part 2), we construct
\[
\tau_n^-=\sum_{i=0}^\infty t_{n,i-1}1(t_{n,i-1}\le \tau<t_{n,i}),\quad n\ge0,
\]
then $\tau_n^-\uparrow \tau$. We can see that $X(\tau_n^-)$ is $\mathcal{F}_{\tau_n^-}$-measurable.
By the predictability of $X(\cdot)$, $X(\tau)1\{\tau<\infty\}$ is $\mathcal{F}_{\tau-}$-measurable since $t<\tau_n^-$ implies $t<\tau$.
\end{proof}

We introduce some new notions to study stopped martingales.
\begin{definition}[Reverse Filtration]
The filtration $\mathbb{F}\equiv\{\mathcal{F}_n:~0\le n\le\infty\}$ is called a reverse filtration if
$\mathcal{F}_n\downarrow F_\infty=\cap_{\ell=1}^\infty\mathcal{F}_{\ell}$.
An adapted process $X_{\cdot}$ is called a reverse martingale if $\mathbb{E}[X_m\mid\mathcal{F}_n]=X_n$ for $m<n$.
\end{definition}

\begin{proposition}
A reverse sub-martingale $X_{\cdot}$ is uniformly integrable if $\mathbb{E}[X_n]>b$ for any $n\ge0$ and some $b\in\mathbb{R}$.
\end{proposition}
\begin{proof}
By the reverse sub-martingale assumption, $\mathbb{E}[X_n]$ is decreasing in $n$ and converges to a finite $\alpha$. Moreover, $X_n\le \mathbb{E}[X_0\mid\mathcal{F}_n]$.
Hence, $X_{\cdot}^+$ is uniformly integrable by conditional expectation.
It remains to check whether $X_{\cdot}^-$ is uniformly integrable.
For $a>0$,
\begin{align*}
\mathbb{E}[X_n^{-}1(X_n^->a)]&=\mathbb{E}[-X_n1(-X_n>a)]=\mathbb{E}[-X_n(1 - 1(-X_n\le a))]\\
&=-\mathbb{E}[X_n] + \mathbb{E}[X_n1(-X_n\le a)]
\end{align*}
By the convergence of $X_n$, there exists an $m\ge0$ so that $|\mathbb{E}[X_n]-\mathbb{E}[X_m]|<\varepsilon$ as long as $n\ge m$.
As a result,
\begin{align*}
\mathbb{E}[X_n^{-}1(X_n^->a)]&\le -\mathbb{E}[X_m] + \mathbb{E}[X_n1(-X_n\le a)]+\varepsilon\\
&\le -\mathbb{E}[X_m] + \mathbb{E}[X_m1(-X_n\le a)]+\varepsilon\\
&=-\mathbb{E}[X_m1(-X_n>a)]+\varepsilon\le \mathbb{E}[|X_m|1(-X_n>a)]+\varepsilon.
\end{align*}
It remains to show that 
\[
\lim_{a\to\infty}\sup_{n\ge m}\mathbb{E}[|X_m|1(-X_n>a)]=0.
\]
Observe that $\mathbb{E}[|X_n|]\le -\alpha+\varepsilon+2\mathbb{E}[X_n^+]\le -\alpha+\varepsilon+2\mathbb{E}[X_0^+]\equiv\beta$.
It follows that 
\[
\mathbb{E}[|X_m|1(-X_n>a)]\le \mathbb{E}[|X_m|1(|X_n|>a)]\to0
\]
\end{proof}

Recall that $X^{\tau}(t)=X(\tau\land t)$ for $t\ge0$, and $X^{\tau}(\cdot)$ is a $\tau$-stopped process.
\begin{proposition}
If $X(\cdot)$ is a martingale and $\tau$ is a stopping time, then $X^{\tau}(\cdot)$ is also a martingale.
\end{proposition}

In discrete time case, we obtain the optional sampling theorem under the uniform integrability condition in the following steps:
\begin{enumerate}
\item
Assume that the stopping time $\tau$ is bounded, and derive the theorem;
\item
Under the condition 1), show the almost sure convergence of a sub-martingale assuming $\sup_{n\ge0}\mathbb{E}[X_n^+]<\infty$. The key is applying Doob's upcrossing inequality;
\item
For a general stopping time $\tau$, show that a stopped process $X_{\cdot}^{\tau}$ is a uniformly integrable martingale if $X_{\cdot}$ is so. 
\end{enumerate}

All those steps can be carried for the continuous process $X(\cdot)$ in such a way that in step 2), the condition is replaced by $\sup_{n\ge0}\mathbb{E}[X^+(t)]<\infty$, and Doob's inequality is applied into the discrete time process $Y_{\cdot}^{(n)}(t)$.

Consider $Y_{\cdot}^{(n)}(t)$ defined as
\[
Y_{\cdot}^{(n)}(t)=\{X(t_{n,i}):~0\le i\le n_t-1\}\cup\{X(t)\},
\]
which is a sub-martingale when $X(\cdot)$ is so.
For $a<b$, define $N_0=-1$, and define the down and up crossings:
\[
N_{2k-1}^{(n)}=\inf\{i> N_{2k-2}^{(n)}:~Y_i^{(n)}(t)\le a\},\quad
N_{2k}^{(n)}=\inf\{i> N_{2k-1}^{(n)}:~Y_i^{(n)}(t)\ge b\}.
\]
Take $U^{(n)}(t)=\sup\{k\ge1:~N_{2k}^{(n)}\le n_t\}$, where $n_t=\min\{\ell\ge0:~\ell 2^{-n}\ge t\}$.
The following lemma holds immediately:

\begin{proposition}
If $X(\cdot)$ is a sub-martingale, then for any $a,b\in\mathbb{R}$ with $a<b$, and any $t>0,n\ge0$,
\[
(b-a)\mathbb{E}[U^{(n)}(t)]\le \mathbb{E}[(X(t) - a)^+] - \mathbb{E}[(X(0) - a)^+]
\]
\end{proposition}

\begin{theorem}
Suppose that $X(\cdot)$ is a sub-martingale and $\sup_{n\ge0}\mathbb{E}[X^+(t)]<\infty$, then
\[
X(t)\xrightarrow{a.s.}X(\infty)
\]
for some random variable $X(\infty)$ with $\mathbb{E}[|X(\infty)|]<\infty$.
\end{theorem}

\begin{theorem}
Suppose that $X(t)\xrightarrow{P}X$ as $t\to\infty$, then the followings are equivalent:
\begin{enumerate}
\item
$X(\cdot)$ is UI;
\item
$X(t)\xrightarrow{L^1}X$;
\item
$\mathbb{E}[|X(t)|]\to\mathbb{E}[|X|]<\infty$.
\end{enumerate}
\end{theorem}

\begin{theorem}[Uniformly Integrable Submartingale]
If $X(\cdot)$ is a uniformly integrable sub-martingale, then there exists $X(\infty)$ with $X(t)\to X(\infty)$ almost surely and $L^1$. Furthermore, $\mathbb{E}[|X(\infty)|]<\infty$, $X(t)\le\mathbb{E}[X(\infty)\mid\mathcal{F}_t]$.
\end{theorem}

\begin{theorem}[Optional Sampling Theorem]
If $X(\cdot)$ is a uniformly integrable sub-martingale, and $\tau,\eta$ are stopping times with $\tau\le\eta$, then
\[
\mathbb{E}[X(0)]\le \mathbb{E}[X(\tau)]\le \mathbb{E}[X(\eta)]\le \mathbb{E}[X(\infty)],
\]
and $X(\tau)\le \mathbb{E}[X(\eta)\mid\mathcal{F}_{\tau}]$.
\end{theorem}







