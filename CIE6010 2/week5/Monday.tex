
\chapter{Week5}

\section{Monday}\index{week5_Friday_lecture}
\subsection{Review}
\paragraph{Optimality Condition}
Given a general problem
\[
\begin{array}{ll}
\min&f(x)\\
\mbox{s.t.}&x\in X\subseteq\mathbb{R}^n\\
&f\mbox{ is }\mathcal{C}^1\mbox{ or }\mathcal{C}^2
\end{array}
\]
One of the most important thing is the optimality condition.
\begin{itemize}
\item
Unconstrainted: $X=\mathbb{R}^n$. (First order and second order)
\item
Constrainted:
\begin{itemize}
\item
1st order necessary condition: Let $\bm x^*$ be a local minimum, then
\[
\inp{\nabla f(\bm x^*)}{(\bm x-\bm x^*)}\ge0,\forall x\in X
\]
\item
For convex function $f$, the above is also the sufficient condition, since
\[
f(\bm x)\ge f(\bm x^*)+\inp{\nabla f(\bm x^*)}{(\bm x-\bm x^*)},\forall x,x^*\in X.
\]
\end{itemize}
\end{itemize}

The optimality condition for constrainted problem is difficult to check. We want a more efficient way, which will be discussed later.
\paragraph{Iterative descent methods}
For unconstraint problem, we consider the iterative descent methods:
\[
\begin{array}{ll}
\bm x \leftarrow \bm x-\alpha\cdot\bm D\cdot\nabla f(\bm x)
&
(\bm D\succ0)
\end{array}
\]
\begin{itemize}
\item
If $\bm D=\bm I$, it is the first order (gradient) method.
\item
If $\bm D=\left(\nabla^2f(\bm x)\right)^{-1}$, it is the second order (Newton's) method
\begin{remark}
Sometimes it is difficult to compute $\nabla^2f(\bm x)$. We can apply finite difference method to accurately approximate the Hessian matrix.
\end{remark}
\item
If $\bm D=(\bm J\trans\bm J)^{-1}$ with Jacobian matrix for nonlinear least squares problem, it is the Gauss-Newton method.
\item
Sometimes we apply rough method to approximate the Hessian matrix in-accurately, which is called the \emph{Quasi-Newton} method. The most famous one is BFGS (L-BFGS).
\end{itemize}
There are more generalized iterative descent methods, such as the accelerated descent method tried in Assignment 3.
\paragraph{Reading materials} CG-conjugate gradient methods; and Nestorov's method (\emph{optimal} accelerated method in \emph{worse} case).

There is a method which is much faster than Nestorov's method in most cases:
\begin{align*}
\bm D^r&=\frac{1}{L}\bm I+\frac{\bm S^r (\bm S^r)\trans}{(\bm y^r)\trans(\bm y^r)}\succ0\\
\alpha&=1\\
\bm S^r&=\bm x^{r+1} - \bm x^r\\
\bm y^r&=\nabla f(\bm x^{r+1}) - \nabla f(\bm x^r)
\end{align*}
\paragraph{Step-size}
\begin{itemize}
\item
Back-tracking with Amijo condition:
\[
\begin{array}{ll}
\mbox{Amijo condition}
&
f(\bm x+\alpha\bm d)\le f(\bm x)+C_1\alpha\inp{\nabla f(\bm x)}{\bm d},0<C_1<1
\end{array}
\]
\item
Wolfe condition for line search:
\[
\begin{array}{ll}
\mbox{Wolfe condition}
&
\left\{
\begin{aligned}
f(\bm x+\alpha\bm d)&\le f(\bm x)+C_1\alpha\inp{\nabla f(\bm x)}{\bm d},&0<C_1<1\\
\inp{\nabla f(\bm x+\alpha\bm d)}{\bm d}&\ge C_2\inp{\nabla f(\bm x)}{\bm d}, &0<C_2<1
\end{aligned}
\right.
\end{array}
\]
Define $h(\alpha)=h(\bm x+\alpha\bm d)$, then the Wolfe condition is essentially
\[
\left\{
\begin{aligned}
h(\alpha)&\le h(0)+C_1h'(0)\\
h'(\alpha)&\ge C_2h'(0)
\end{aligned}
\right.
\]
\item
Constant step-size: $\alpha^r\equiv\frac{1}{L}$ with $L$ be the Lipschitz constant of $\nabla f(\bm x)$.
\item
$\alpha^r\to0$ with $\sum\alpha^r=+\infty$.
\end{itemize}
\begin{remark}
Amijo condition guarantees that $f(\bm x^r) - f(\bm x^{r+1})\ge-C_1\alpha^r\inp{\nabla f(\bm x^r)}{\bm d^r}$. Assume $f(\bm x)>-\infty$, then
\[
\alpha^r\inp{\nabla f(\bm x^r)}{\bm d^r}\to0
\]
We want $\nabla f(\bm x^r)\to0$, which means your direction $\bm d^r$ should not be perpendicular to $\nabla f(\bm x^r)$ after some iterations. If choosing $\bm d^r=-\nabla f(\bm x^r)$, then $\alpha^r\|\nabla f(\bm x^r)\|^2\to0$, which implies $\nabla f(\bm x^r)\to0$.
\end{remark}
Under reasonable conditions, applying first order condition we expect $\nabla f(\bm x^r)\to0$. Is $\bm x^r$ always convergent? not necessarily.

\paragraph{Local convergence rate}
The first order method has linear or sub-linear convergence rate; while the second order method has quadratic convergence rate.

\paragraph{Finite difference Method}
Given $F(\bm x): \mathbb{R}^n\mapsto\mathbb{R}^n$, its Jacobian is given by:
\[
F'(\bm x)=\begin{bmatrix}
\nabla\trans F_1(\bm x)\\
\vdots\\
\nabla\trans F_n(\bm x)
\end{bmatrix}
\]
Its $j$th column is given by:
\begin{align*}
F'(\bm x)\bm e_j:&=\lim_{h\to0}\frac{F(\bm x+h\bm e_j) - F(\bm x)}{h}\\
&\approx \frac{F(\bm x+h\bm e_j) - F(\bm x)}{h}\mbox{ for small }h
\end{align*}
where for $\varepsilon=10^{-8}$,
\[
h=\varepsilon\max\{1,|x_j|\}\mbox{sign}(x_j),
\]
more-multiplying the term $\mbox{sign}(x_j)$ means we avoid subtract between $\bm x$ and $h\bm e_j$.

\subsection{Existence of solution to Quadratic Programming}
\begin{theorem}
Let $\{S^k\}$ be a sequence of non-empty closed nested sets. Suppose that all \emph{asymptotic} sequences corresponding to asymptotic directions of $\{S^k\}$ are \emph{retractive}, then $\bigcap_{k=0}^\infty S^k$ is \emph{non-emtpy}.
\end{theorem}
\begin{definition}
Let $\{S^k\}$ be a sequence of non-empty closed nested sets. We say that a vector $\bm d\ne\bm0$ is an \emph{asymptotic direction} of $\{S^k\}$ if there exists a sequence $\{\bm x^k\}$ such that
\[
\begin{array}{lll}
\bm x^k\in S^k,
&
\bm x^k\ne\bm 0,
&
k=0,1,2,\dots
\end{array}
\]
and
\[
\begin{array}{ll}
\|\bm x^k\|\to\infty,
&
\frac{\bm x^k}{\|\bm x^k\|}\to\frac{\bm d}{\|\bm d\|}
\end{array}
\]
\begin{itemize}
\item
$(\{\bm x^k\},\bm d)$ is said to be the asymptptoc pair of $\{S^k\}$
\item
$\{\bm x^k\}$ is said to be \emph{retractive} if there exists a bounded sequence of positive numbers $\{\alpha^k\}$ and $\bar k$ such that
\[
\bm x^k-\alpha^k\bm d\in S^k,\qquad
\forall k\ge\bar k
\]
\end{itemize}
\end{definition}

\begin{theorem}
Let $\bm Q$ be a positive semi-definite symmetric $n\times n$ matrix, let $\bm c$ and $\bm a_1,\dots,\bm a_r$ be vectors in $\mathbb{R}^n$, and let $b_1,\dots,b_r$ be scalars. Assume that the optimal value of the problem
\begin{equation}
\begin{array}{ll}
\min&\bm x\trans\bm Q\bm x+\bm c\trans\bm x\\
\mbox{such that}&\bm a_j\trans\bm x+b_j\le0,\qquad j=1,2,\dots,r,
\end{array}
\end{equation}
is finite. Then the problem has \emph{at least one optimal} solution.
\end{theorem}
\begin{proof}
Suppose $f^*$ is the optimal solution. The feasible region is denoted by:
\[
F=\{\bm x \mid a_j\trans\bm x+b_j\le0,j=1,\dots,r\}.
\] 

Set a decreasing sequence $\{\gamma^k\}$ with limit $f^*$, and set
\[
S^k:=\{\bm x\in F\mid \bm x\trans\bm Q\bm x+\bm c\trans\bm x\le\gamma^k\}
\]
Thus the set of optimal solutions is $\bigcap_{k=0}^\infty S^k$. It suffices to show that all asymptotic seqeuences corresponding to asymptotic directions are \emph{retractive}.

\paragraph{Asymptotic Directions are essentially Boundary Directions} For fiexed asymptotic pair $(\{\bm x^k\},\bm d)$, we claim that 
\begin{align}
\bm Q\bm d&=0,\inp{\bm c}{\bm d}\le0\\
\inp{\bm a_j}{\bm d}&\le0, j=1,2,\dots,r
\end{align}

For first equaility, define $\bm d^k=\frac{\bm x^k}{\|\bm x^k\|}$. Since $\bm x^k\in S^k$, we have
\begin{equation}
(\bm d^k)\trans\bm Q\bm d^k+\frac{\inp{\bm c}{\bm d^k}}{\|\bm x^k\|}\le\frac{\gamma^k}{\|\bm x^k\|^2}\label{Eq:5:4}
\end{equation}
Taking $k\to\infty$, we imply $(\bm d)\trans\bm Q\bm d\le0$, and therefore $\bm{Qd}=0$ as $\bm Q\succeq0$.

Due to (\ref{Eq:5:4}) and the semi-definiteness of $\bm Q$, we have
\[
\inp{\bm c}{\bm d^k}\le 
\|\bm x^k\|(\bm d^k)\trans\bm Q\bm d^k+\inp{\bm c}{\bm d^k}\le\frac{\gamma^k}{\|\bm x^k\|}
\]
Taking $k\to\infty$, we imply $\inp{\bm c}{\bm d^k}\le0$. Similarly, $\inp{\bm a_j}{\bm d}\le0$ since $\inp{\bm a_j}{\bm d^k}\le-\frac{b_j}{\|\bm x^k\|}$.
\paragraph{Finiteness of optimal value}Next we show $\inp{\bm c}{\bm d}=0$. For a feasible vector $\bar{\bm x}$, consider $\tilde{\bm x}:=\bar{\bm x}+m\bm d$ for any positive $m$, which is also feasible as $\inp{\bm a_j}{\bm d}\le0$. Then checking the function evaluated at $\tilde{\bm x}$:
\[
f^*\le (\tilde{\bm x})\trans\bm Q(\tilde{\bm x})+\inp{\bm c}{\tilde{\bm x}}
=
(\bar{\bm x})\trans\bm Q\bar{\bm x}+\inp{\bm c}{\bar{\bm x}}+m\inp{\bm c}{\bm d}
\]
As $f^*$ is finite, $\inp{\bm c}{\bm d}\ge0$, i.e., $\inp{\bm c}{\bm d}=0$.

As a result, for fixed $\bm x^k$, the function evaluated at $\bm x^k-\alpha\bm d$ satisfies
\[
(\bm x^k-\alpha\bm d)\trans\bm Q(\bm x^k-\alpha\bm d)
+
\inp{\bm c}{\bm x^k-\alpha\bm d}=(\bm x^k)\trans\bm Q\bm x^k+\inp{\bm c}{\bm x^k}\le\gamma^k,\forall\alpha,k,
\]
\paragraph{Feasibleness of $\bm x^k-\alpha\bm d$}It suffices to choose $\alpha>0$ to let $\bm x^k-\alpha\bm d\in F$ for sufficiently large $k$, i.e., 
\[
\begin{array}{ll}
\inp{\bm a_j}{\bm x^k-\alpha\bm d}+b_j\le0,
&
j=1,\dots,r
\end{array}
\]
\begin{itemize}
\item
This is true for $\forall\alpha>0$ if $\inp{\bm a_j}{\bm d}=0$
\item
Otherwise, suppose $\inp{\bm a_j}{\bm d}<-\varepsilon$. Thus $\inp{\bm a_j}{\bm d^k}<-\varepsilon$ for sufficiently large $k$, i.e., $\inp{\bm a_j}{\bm x^k}\le-\varepsilon\|\bm x^k\|$. Combining the unboundness of $\{\bm x^k\}$, we imply
\[
\inp{\bm a_j}{\bm x^k-\alpha\bm d}+b_j\le-\varepsilon\|\bm x^k\|-\alpha\inp{\bm a_j}{\bm d}+b_j<0
\]


\end{itemize}



\end{proof}


















