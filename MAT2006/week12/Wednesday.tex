
\chapter{Week12}
\section{Wednesday}
\subsection{Recap for Rank Theorem}
\paragraph{Inverse Function Theorem}
Given a $\mathcal{C}^p$ function $f:E(\subseteq\mathbb{R}^m)\subseteq\mathbb{R}^m$ and $Df(\bm x_0)$ is invertible. Then we imply that there is a neighborhood $U(\bm x_0)\times V(\bm y_0)$ of $(\bm x_0,f(\bm x_0))$ such that $f$ is a $\mathcal{C}^p$-diffeomorphism between $U(\bm x_0)$ and $V(\bm y_0)$; moreover,
\[
D(f^{-1})(\bm y_0)=(Df(\bm x_0))^{-1}
\]

\paragraph{Rank Theorem}
Given a $\mathcal{C}^p$ function $f:U(\bm x_0)\to\mathbb{R}^n$ of constant rank $k$ throughout $U(\bm x_0)$. Then there exists a neighborhood $N(\bm x_0)\times N(f(\bm x_0))$ and two $\mathcal{C}^p$-diffeomorphisms
\[
\begin{array}{ll}
\bm u=\phi(\bm x), \bm x\in N(\bm x_0)
&
\bm v=\psi(\bm y), \bm y\in N(\bm y_0), \bm y_0:=f(\bm x_0),
\end{array}
\]
such that the composition $\psi\circ f\circ\phi^{-1}$ takes the form
\[
(u_1,\dots,u_k,u_{k+1},\dots,u_m)\to
(u_1,\dots,u_k,0,0,\dots,0)
\]
\paragraph{Outline of proof}
\begin{enumerate}
\item[Step 1:] Identify that
\[
f=\begin{pmatrix}
f_1\\\vdots\\f_k\\\vdots\\f_n
\end{pmatrix},\qquad
Df=\frac{\partial(f_1,\dots,f_n)}{\partial(x_1,\dots,x_m)}
\]
w.l.o.g., assume the first $k\times k$ principal minors of $Df(\bm x_0)$ to be non-singular. Since $\det(\cdot)$ and $f$ are continuously differentiable, we imply that $Df(\bm x)$ is non-singular for some neighborhood of $\bm x_0$, say $N(\bm x_0)$.
\item[Step 2:] Then construct the map $\phi(\bm x)$
\[
\phi(\bm x)=\begin{pmatrix}
f_1(\bm x)\\\vdots\\f_k(\bm x)\\x_{k+1}\\\vdots\\x_m
\end{pmatrix}\implies
D\phi=\begin{pmatrix}
\frac{\partial(f_1,\dots,f_k)}{\partial(x_1,\dots,x_k)}&
\frac{\partial(f_1,\dots,f_k)}{\partial(x_{k+1},\dots,x_m)}
\\
\bm0&\bm I
\end{pmatrix},
\]
by computing the determinant we assert that $D\phi$ is invertible over $N(\bm x_0)$, which implies $\phi$ is a differeomorphism.
\item[Step 3:] Define $g:=f\circ \phi^{-1}: \phi(N(\bm x_0))\to\mathbb{R}^n$, then re-write $g$ as
\[
\begin{pmatrix}
y_1\\\vdots\\y_k\\y_{k+1}\\\vdots\\y_n
\end{pmatrix}
:=g(\bm u)=
\begin{pmatrix}
u_1\\\vdots\\u_k\\g_{k+1}(\bm u)\\\vdots\\g_n(\bm u)
\end{pmatrix}
\implies
Dg=\begin{pmatrix}
\bm I&\bm0\\
\frac{\partial(g_1,\dots,g_k)}{\partial(u_1,\dots,u_k)}&\frac{\partial (g_{k+1},\dots,g_n)}{\partial(u_{k+1},\dots,u_n)}
\end{pmatrix},
\]
which implies the lower right corner should be zero matrix since $\rank(Dg)=\rank(Df[\phi^{-1}]\cdot D\phi^{-1})=k$. 

In other words, $(g_{k+1},\dots,g_n)(\bm u)$ depends only on the first $k$ variables. Thus rewrite $g$ as:
\[
\begin{pmatrix}
y_1\\\vdots\\y_k\\y_{k+1}\\\vdots\\y_n
\end{pmatrix}
=
\begin{pmatrix}
u_1\\\vdots\\u_k\\g_{k+1}(u_1,\dots,u_k)\\\vdots\\g_n(u_1,\dots,u_k)
\end{pmatrix}
\]
\item[Step 4:] Define the map $\bm v=\psi(\bm y)$:
\[
\begin{pmatrix}
v_1\\\vdots\\v_k\\v_{k+1}\\\vdots\\v_n
\end{pmatrix}
=
\begin{pmatrix}
y_1\\\vdots\\y_k\\y_{k+1}-g_{k+1}(y_1,\dots,y_k)\\\vdots\\
y_{k+1} - g_n(y_1,\dots,y_k)
\end{pmatrix}
\]

With careful computation, we find
\[
\psi\circ f\circ\phi^{-1}(\bm u)=\psi\circ g(\bm u)=\psi\begin{pmatrix}
u_1\\\vdots\\u_k\\g_{k+1}(\bm u)\\\vdots\\g_n(\bm u)
\end{pmatrix}=\begin{pmatrix}
u_1\\\vdots\\u_k\\0\\\vdots\\0
\end{pmatrix}
\]
\end{enumerate}
\begin{remark}
The rank theorem answers the question that whether we can \emph{flatten out} the curve at given point by some proper way of change of variables. The detailed way for the change of variables follows the insights in the proof.
\end{remark}

\begin{example}
\begin{enumerate}
\item
Define $f(t)=(\cos t,\sin t), t\in\mathbb{R}$. Define $t_0=\frac{\pi}{4}$. 
\begin{quotation}
Can we flatten out the curve $f(t)$ near the point $f(t_0)$?
\end{quotation} 
\begin{itemize}
\item
Note that
\[
Df(t_0)=(-\frac{\sqrt{2}}{2},\frac{\sqrt{2}}{2})\ne0,
\]
with rank $1$. Moreover, $\rank(Df(t))=1$ for some $t$ in the neighborhood of $t_0$. Hence the answer is yes.
\end{itemize}
\begin{quotation}
What is the specific way of the change of variables that makes $f(t)$ flat near $f(t_0)$?
\end{quotation}
\begin{itemize}
\item
Choose $\phi(t)=\cos t$ and $\phi^{-1}(u)=t=\cos^{-1}u$, which follows that
\[
g(u)=f(\phi^{-1}(u))=\begin{pmatrix}
\cos(\phi^{-1}u)\\\sin(\phi^{-1}(u))
\end{pmatrix}
=\begin{pmatrix}
u\\
\sin(\cos^{-1}u)
\end{pmatrix}.
\]
\item
Choose $\psi(y)=\begin{pmatrix}
y_1\\y_2-\sin(\cos^{-1}y_1)
\end{pmatrix}$, which follows that
\begin{align*}
\psi\circ f\circ\phi^{-1}(u)
&=
\psi\circ f(\cos^{-1}u)\\
&=\psi\begin{pmatrix}
\cos\cos^{-1}u
\\
\sin\cos^{-1}u
\end{pmatrix}=
\psi\begin{pmatrix}
u
\\
\sin\cos^{-1}u
\end{pmatrix}
=\begin{pmatrix}
u\\0
\end{pmatrix}
\end{align*}
\end{itemize}
\item
$f(x_1,x_2)=(x_1+x_2,x_1-x_2,x_1x_2)$. 
\begin{quotation}
Can we flatten out the curve of $f$ near $(0,0)$?
\end{quotation}
\begin{itemize}
\item
Check that \[
Df(x_1,x_2)=\begin{pmatrix}
1&1\\
1&-1\\
x_2&x_1
\end{pmatrix},
\]
which is of rank $2$ throughout $\mathbb{R}^2$, and therefore the answer is yes.
\end{itemize}
\begin{quotation}
What is the specific way of the change of variables that makes $f(t)$ flat near $f(0,0)$?
\end{quotation}
\begin{itemize}
\item
Define
\[
\phi(x_1,x_2)=\begin{pmatrix}
f_1\\f_2
\end{pmatrix}=\begin{pmatrix}
x_1+x_2\\x_1-x_2
\end{pmatrix}
\]
and therefore
\[
g=f\circ\phi^{-1}(u_1,u_2)=f\begin{pmatrix}
\frac{u_1+u_2}{2}\\\frac{u_1-u_2}{2}
\end{pmatrix}
=
\begin{pmatrix}
u_1\\u_2\\\frac{u_1^2-u_2^2}{4}
\end{pmatrix}
\]
\item
Define 
\[
\psi(y)=\begin{pmatrix}
y_1\\y_2\\y_3-\frac{y_1^2-y_2^2}{4}
\end{pmatrix}.
\]
which follows that
\[
\psi\circ f\circ \phi^{-1}\begin{pmatrix}
u_1\\u_2
\end{pmatrix}=
\psi\begin{pmatrix}
u_1\\u_2\\\frac{u_1^2-u_2^2}{4}
\end{pmatrix}
=
\begin{pmatrix}
u_1\\u_2\\0
\end{pmatrix}
\]
\end{itemize}


\end{enumerate}
\end{example}
\subsection{Functional Dependence}
In linear algebra we have talked about the linear independence:
\begin{definition}[Vector Dependence]
Given $n$ vectors $\bm v_1,\dots,\bm v_n$, they are linear independent if the equation
\[
a_1\bm v_1+\cdots+a_n\bm v_n=0
\]
only has the trivial solution $a_1=a_2=\cdots=a_n$.
\end{definition}

Then we talk about the dependence between functions.
\begin{definition}[Dependence]
A set of \emph{continuous} functions $\{f_1,\dots,f_n: U\to\mathbb{R}\}$, where $U\subseteq\mathbb{R}^m$ is a neighborhood of $\bm x_0\in\mathbb{R}^m$, is said to be \emph{functionally independent} if for the undetermined continuous $n$-argument scalar function $F$, the function equation
\begin{equation}
F(f_1(\bm x),\dots,f_n(\bm x))\equiv0,\label{Eq:12:1}
\end{equation}
only has the trivial solution $F\equiv0$ in some neighborhood $V$ of $\bm y_0:=(f_1,\dots,f_n)(\bm x_0).$
\end{definition}
\begin{proposition}
Let $\{f_1,\dots,f_n\}$ be $\mathcal{C}^1$ and the rank of
\[
Df(\bm x)=
\frac{\partial (f_1,\dots,f_n)}{\partial(x_1,\dots,x_m)}(\bm x)
\]
is $k$ at every $\bm x\in U$, then 
\begin{enumerate}
\item
$k=n$ implies $\{f_1,\dots,f_n\}$ is functionally independent
\item
$k<n$ implies there exists a neighborhood of $\bm x_0$ and $k$ functions $f_1,\dots,f_k$ such that the rest of $(n-k)$ functions can be written as
\[
f_i(\bm x)=g_i(f_1(\bm x),\dots,f_k(\bm x))
\]
for $\forall i=k+1,\dots,n$, where $g_i$ are $\mathcal{C}^1$ functions of $k$ variables.
\end{enumerate}
\end{proposition}

\begin{proof}
\begin{enumerate}
\item
Re-write (\ref{Eq:12:1}) as $F\circ f\equiv0$. Applying rank theorem, there exists some differemorphism $\phi$, $\psi$ such that $\psi\circ f\circ\phi^{-1}$ assigns $(y_1,\dots,y_n)$ into $(y_1,\dots,y_n)$, i.e., it is an identity map $\bm{id}$ around some neighborhood of $\bm y_0:=f(\bm x_0)$. Therefore,
\[
F\circ f=F\circ\psi^{-1}\circ(\psi\circ f\circ\phi^{-1})\circ\phi
=
F\circ\psi^{-1}\circ\bm{id}\circ\phi
=
F\circ\psi^{-1}\circ\phi\equiv0
\]
Or equivalently,
\[
F=0\circ\phi^{-1}\circ\psi\equiv0
\]
\item
Applying rank theorem and IFT gives the desired result.
\end{enumerate}
\end{proof}



















