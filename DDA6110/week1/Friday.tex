% !TEX encoding = UTF-8 Unicode

\section{Wednesday}\index{Friday_lecture}

Convex Programming~(CP):
\[
\begin{array}{ll}
\min&\quad f_0(x)\\
\mbox{s.t.}&\quad x\in X\subseteq\mathbb{R}^n
\end{array}
\]
with $f$ and $X$ being convex.
The standard constraint set $X$ can be written as:
\begin{equation*}
X=\bigg\{
x\bigg|
f_i(x)\le b_i,~~
i=1,2,3,\ldots,m,\quad
Ax=c
\bigg\}
\end{equation*}
with $A\in\mathbb{R}^{p\times n}$, $p<n$, and $f_i$'s are convex.

\begin{proposition}
The constraint set
\[
X=\bigg\{
x\bigg|
f_i(x)\le b_i,~~i=1,\dots,m
\bigg\}
\]
is \emph{convex} if all $f_i$'s are convex.
\end{proposition}
\begin{proof}
For any $x_1,x_2\in X$ and $\theta\in[0,1]$, verify that
\begin{align*}
f_i(\theta x_1 + (1-\theta)x_2)
&\le 
\theta f_i(x_1) + (1-\theta) f_i(x_2)\\
&\le \theta b_i + (1-\theta)b_i=b_i
\end{align*}
where the first inequality is because of the convexity,
and the second inequality is because that $x_i\in X, i=1,2$.
As a result, $\theta x_1 + (1-\theta)x_2\in X$, i.e., $X$ is a convex set.
\end{proof}

\begin{proposition}
If $\{X_i\}_{i\in\mathcal{I}}$ is a collection of convex sets, then the union $X=\bigcap_{i\in \mathcal{I}}$ is convex.
\end{proposition}
\begin{proof}
Consider any $x_1,x_2\in X$, i.e., $x_1,x_2\in X_i,\forall i\in\mathcal{I}$, then by the convexity of $X_i$s, the line segment
\[
\theta x_1 + (1-\theta)x_2\in X_i,\forall i\in\mathcal{I},
\]
i.e., $\theta x_1 + (1-\theta)x_2\in X$.
\end{proof}

\begin{definition}[Epigraph]
Consider a function $f:\mathbb{R}^n\to\mathbb{R}$, 
the epigraph of $f$ is a set defined as
\[
\text{epi}(f)
=
\bigg\{
(x,t)\bigg|
f(x)\le t ,x\in\text{dorm}(f)
\bigg\}\in\mathbb{R}^{n+1}.
\]
\end{definition}

\begin{proposition}
The function $f$ is convex if and only if the epigraph $\text{epi}(f)$ is convex.
\end{proposition}
\begin{proof}
We only talk about the proof on the forward direction.
Consider any $(x_1,t_1), (x_2,t_2)\in \text{epi}(f)$, then we show that $\theta(x_1,t_1)+(1-\theta)(x_2,t_2)\in \text{epi}(f)$ for any $\theta\in[0,1]$:
\begin{align*}
f(\theta x_1 + (1-\theta)x_2)&\le \theta f(x_1) + (1-\theta)f(x_2)\\
&\le \theta t_1 + (1-\theta)t_2,
\end{align*}
which proves the desired result.
\end{proof}

Let's start to introduce some useful convex sets.
\begin{definition}[Combination of two points]
Consider the combination of two points $x_1,x_2$:
\[
\theta_1 x_1 + \theta_2x_2,
\]
\begin{itemize}
\item
When $\theta_1,\theta_2\in\mathbb{R}$, it is called an affine combination;
the resulted space is an \emph{affine space}.
\item
When $\theta_1,\theta_2\in\mathbb{R}_+$, it is called a non-negative combination;
the resulted space is a \emph{convex cone}.
\item
When $\theta_2=1-\theta_1,\theta_1\in[0,1]$, it is called a convex combination;
the resulted space is a \emph{convex set}.
\end{itemize}
\end{definition}

\begin{definition}[Affine Set]
The standard form of an affine space is $\{\bm x\mid\bm{Ax}=\bm b\}$.
\end{definition}

\begin{definition}[Cone]
A set is called a cone with a vertex at the origin if 
for any $\bm x\in X$, $a\bm x\in X$ for any $a\ge0$.
\end{definition}

The standard form of a convex conic programming is the following:
\[
\begin{array}{ll}
\min&\quad\inp{\bm C}{\bm X}\\
\mbox{s.t.}&\quad \inp{\bm A_i}{\bm X}=\bm B_i,~~i=1,2,\dots,m\\
&\quad \bm X\in\mathcal{K}
\end{array}
\]
where $\mathcal{K}$ is a convex cone.

One special form of conic programming is the linear programming:
\[
\begin{array}{ll}
\min&\quad\inp{\bm c}{\bm x}\\
\mbox{s.t.}&\quad \bm{Ax}=\bm b\\
&\quad \bm x\ge\bm0
\end{array}
\]

Another is the second-order cone programming:
\[
\mathcal{K} = \{(\bm x,t)\mid \|\bm x\|_2\le t\}
\]

The semidefinite programming also belongs to the case of conic programming:
\[
\mathcal{K}=\mathcal{S}_+^n
\triangleq
\{
\bm X\in\mathbb{S}^n\mid
\bm X\succeq0,\forall \bm v\in\mathbb{R}^n
\}
\]

Convex hull of a set $S$:

It is the smallest convex set containing $S$, called $\text{conv}(S)$.

\begin{definition}[Polyhedron]
\begin{itemize}
\item
A hyperplane in $\mathbb{R}^n$ can be written as the form $\{\bm x\mid\bm a\trans\bm x=b\}$.
\item
Half space: $\{\bm x\mid\bm a\trans\bm x\le\bm b\}$.
\item
Polyhedron: a intersection of finite hyperplanes and half spaces.
\item
Ellipsoid:
\begin{align*}
\bigg\{
\bm x
\bigg|
(\bm x - \bm x_c)\trans
\bm P^{-1}
(\bm x - \bm x_c)\le 1,~
\bm P\succ0
\bigg\}
&=
\{\bm x\mid \|\bm P^{-1/2}(\bm x- \bm x_c)\|_2^2\le 1\}\\
&=
\{\bm x\mid 
\bm x = \bm x_c + \bm A\bm u,
\|\bm u\|_2\le 1
\}
\end{align*}

\end{itemize}
\end{definition}







