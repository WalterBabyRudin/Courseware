% Use only LaTeX2e, calling the article.cls class and 12-point type.

\documentclass[11pt]{article}

% Users of the {thebibliography} environment or BibTeX should use the
% scicite.sty package, downloadable from *Science* at
% www.sciencemag.org/about/authors/prep/TeX_help/ .
% This package should properly format in-text
% reference calls and reference-list numbers.

\usepackage{scicite}
\usepackage{bm}
\usepackage{mathtools}
\usepackage{amsmath}
\usepackage{amssymb}
\usepackage{amsthm}
	\usepackage[utf8]{inputenc}
\usepackage[english]{babel}
\usepackage{hyperref}
 \DeclarePairedDelimiterX{\inp}[2]{\langle}{\rangle}{#1, #2}
 \newcommand{\trans}{^{\mathrm T}}
\newtheorem{theorem}{Theorem}[section]
\newtheorem{corollary}{Corollary}[theorem]
\newtheorem{lemma}[theorem]{Lemma}

% Use times if you have the font installed; otherwise, comment out the
% following line.

\usepackage{times}

% The preamble here sets up a lot of new/revised commands and
% environments.  It's annoying, but please do *not* try to strip these
% out into a separate .sty file (which could lead to the loss of some
% information when we convert the file to other formats).  Instead, keep
% them in the preamble of your main LaTeX source file.


% The following parameters seem to provide a reasonable page setup.

\topmargin 0.0cm
\oddsidemargin 0.2cm
\textwidth 16cm 
\textheight 21cm
\footskip 1cm


%The next command sets up an environment for the abstract to your paper.

\newenvironment{sciabstract}{%
\begin{quote} \bf}
{\end{quote}}


% If your reference list includes text notes as well as references,
% include the following line; otherwise, comment it out.

\renewcommand\refname{References and Notes}

% The following lines set up an environment for the last note in the
% reference list, which commonly includes acknowledgments of funding,
% help, etc.  It's intended for users of BibTeX or the {thebibliography}
% environment.  Users who are hand-coding their references at the end
% using a list environment such as {enumerate} can simply add another
% item at the end, and it will be numbered automatically.

\newcounter{lastnote}
\newenvironment{scilastnote}{%
\setcounter{lastnote}{\value{enumiv}}%
\addtocounter{lastnote}{+1}%
\begin{list}%
{\arabic{lastnote}.}
{\setlength{\leftmargin}{.22in}}
{\setlength{\labelsep}{.5em}}}
{\end{list}}


% Include your paper's title here

\title{A new first order method} 


% Place the author information here.  Please hand-code the contact
% information and notecalls; do *not* use \footnote commands.  Let the
% author contact information appear immediately below the author names
% as shown.  We would also prefer that you don't change the type-size
% settings shown here.

%\author
%{Jie Wang\\
%\\
%\normalsize{School of Science and Engineering}\\
%\normalsize{The Chinese University of Hong Kong, Shenzhen}
%}

% Include the date command, but leave its argument blank.

\date{}



%%%%%%%%%%%%%%%%% END OF PREAMBLE %%%%%%%%%%%%%%%%



\begin{document} 

% Double-space the manuscript.

%\baselineskip22pt

% Make the title.

\maketitle 



% Place your abstract within the special {sciabstract} environment.




% In setting up this template for *Science* papers, we've used both
% the \section* command and the \paragraph* command for topical
% divisions.  Which you use will of course depend on the type of paper
% you're writing.  Review Articles tend to have displayed headings, for
% which \section* is more appropriate; Research Articles, when they have
% formal topical divisions at all, tend to signal them with bold text
% that runs into the paragraph, for which \paragraph* is the right
% choice.  Either way, use the asterisk (*) modifier, as shown, to
% suppress numbering.

\section{Introduction}
Professor Yin Zhang talked about a new first order method in his lecture. In particular, see the lecture note ``CIE6010'', page 34 in the website
\begin{center}
\url{
https://walterbabyrudin.github.io/information/information.html
}
\end{center}
In general, this first order method is as follows:
\begin{itemize}
\item
Generate two initial guess $x^1,x^2$.
\item
For each iteration $T=1,2,\dots$
\begin{itemize}
\item
Generate $x^3$ as follows:
\begin{align*}
S&\leftarrow x^{2} - x^1\\
y&\leftarrow\nabla f(x^{2}) - \nabla f(x^1)\\
\alpha&\leftarrow1\\
D &\leftarrow \frac{1}{L}I + \frac{(S)\cdot (S)\trans}{(y)\trans(y)}\succ0\\
x^3&\leftarrow x^{2} - \alpha \cdot D \cdot \nabla f(x^{2})
\end{align*}
\item
Update $x^1\leftarrow x^2$ and $x^2\leftarrow x^3$
\item
When the stop criteria is satisfied, return $x^3$.
\end{itemize}
\end{itemize}
This method can be used to efficiently solve the compressive sensing problem~(actually the L1 regularization problem)
\[
\min_{x\in\mathbb{R}^n}f(x)\equiv \phi_{\sigma}(Dx)+\frac{\mu}{2}\|Ax-b\|_2^2,
\]
where $\phi_{\sigma}(y)=\sum_i\sqrt{y_i^2+\sigma}$.
Prof. Yin Zhang also writes his MATLAB script to implement this algorithm in file \textit{yzL1reg2d.p}.
I also write my own script (see the file \textit{myL1reg2d}.)
%
%In this file, we present some tips and sample mark-up to assure your
%\LaTeX\ file of the smoothest possible journey from review manuscript
%to published {\it Science\/} paper.  We focus here particularly on
%issues related to style files, citation, and math, tables, and
%figures, as those tend to be the biggest sticking points.  Please use
%the source file for this document, \texttt{scifile.tex}, as a template
%for your manuscript, cutting and pasting your content into the file at
%the appropriate places.
%
%{\it Science\/}'s publication workflow relies on Microsoft Word.  To
%translate \LaTeX\ files into Word, we use an intermediate MS-DOS
%routine \cite{tth} that converts the \TeX\ source into HTML\@.  The
%routine is generally robust, but it works best if the source document
%is clean \LaTeX\ without a significant freight of local macros or
%\texttt{.sty} files.  Use of the source file \texttt{scifile.tex} as a
%template, and calling {\it only\/} the \texttt{.sty} and \texttt{.bst}
%files specifically mentioned here, will generate a manuscript that
%should be eminently reviewable, and yet will allow your paper to
%proceed quickly into our production flow upon acceptance \cite{use2e}.


\end{document}




















